\chapter{Χρονοσειρές με εποχιακή συμπεριφορά}
\label{chap2}

\section{Εισαγωγή}
\subsection{Γενικά για τις Χρονοσειρές}
Ως χρονοσειρά μπορούμε να ορίσουμε ένα σύνολο τιμών-παρατηρήσεων που περιγράφουν την εξέλιξη της συμπεριφοράς ενός μεγέθους στο πεδίου του χρόνου. Τα μεγέθη που περιγράφονται μπορούν να είναι οποιασδήποτε φύσης αρκεί να μπορούν να ποσοτικοποιηθούν. Έτσι, συναντάμε χρονοσειρές που περιγράφουν φυσικά μεγέθη όπως είναι ο όγκος νερού βροχόπτωσης για μια περιοχή ανά ημέρα αλλά και οικονομικά όπως είναι οι πωλήσεις ενός προϊόντος συναρτήσει του χρόνου. 

Θεωρείται ότι οι διαδοχικές τιμές μιας χρονοσειράς δεν είναι ανεξάρτητες μεταξύ τους. Ωστόσο, αυτό δεν συνεπάγεται ότι πρόκειται για μια ντετερμινιστική διαδικασία που μας επιτρέπει να καθορίσουμε επακριβώς τις μελλοντικές τιμές της από τις προηγούμενες. Αντίθετα, οι χρονοσειρές αποτελούν στοχαστικές διαδικασίες καθότι η εξέλιξη των τιμών τους ενέχουν τυχαιότητα, δεδομένου ότι περιγράφουν την εξέλιξη ενός μεγέθους στον πραγματικό κόσμο.

\subsection{Συνθετικά στοιχεία μια χρονοσειράς}

Σύμφωνα με τις παραδοσιακές μεθόδους ανάλυσης, μια χρονοσειρά αναλύεται σε τέσσερα δομικά χαρακτηριστικά: την τάση, τη κυκλικότητα, την εποχικότητα και την τυχαιότητα. 

Η \textbf{τάση} μιας χρονοσειράς περιγράφει πως μακροπρόθεσμα μεταβάλλεται το μέσο επίπεδο των τιμών μιας χρονοσειράς. Η τάση μπορεί να είναι ανοδική, φθίνουσα ή μηδενική.

H \textbf{κυκλικότητα} περιγράφει την κατά περιόδους μεταβολή της χρονοσειράς εξαιτίας ειδικών εξωγενών συνθηκών. Συναντάμε συχνά μεταβολές στον κυκλικό παράγοντα της σειράς που οφείλονται σε οικονομικές συνθήκες, ενώ χαρακτηριστικά παραδείγματα βρίσκουμε σε χρονοσειρές που περιγράφουν δείκτες παραγωγής, μετοχών, ΑΕΠ και οικονομικών μεγεθών εν γένει.

Η \textbf{εποχιακότητα} ορίζεται ως ένα μοτίβο διακύμανσης που παρουσιάζει η χρονοσειρά και επαναλαμβάνεται ανά τακτές χρονικές περιόδους, συνήθως μικρότερες του έτους. Μερικά παραδείγματα χρονοσειρών που χαρακτηρίζονται από έντονο στοιχείο εποχιακότητας είναι χρονοσειρές θερμοκρασιών, κατανάλωσης πετρελαίου θέρμανσης και γενικά εποχιακών προϊόντων. 

Η \textbf{τυχαιότητα} αποτελεί τον αστάθμητο παράγοντα της τύχης κατά την εξέλιξη μιας χρονοσειράς.

\section{Προσθετικό και Πολλαπλασιαστικό μοντέλο αποσύνθεσης} 

Για να αναλύσουμε μια χρονοσειρά στα επιμέρους στοιχεία της μπορούμε να χρησιμοποιήσουμε απλές μαθηματικές σχέσεις. Αρχικά ορίζουμε την χρονοσειρά ως μία συνάρτηση των δομικών χαρακτηριστικών της:
\[ Y_t = f(S_t, T_t, C_t, R_t)\]
Με:
$Y_t$ να είναι η χρονοσειρά
\\
$S_t$ η συνιστώσα της εποχιακότητας
\\
$T_t$ η συνιστώσα της τάσης
\\
$C_t$ η συνιστώσα της κυκλικότητας
\\
$R_t$ η συνιστώσα της τυxαιότητας.
\\
Δύο απλές μορφές της συνάρτησης $f$ είναι η προσθετική: \\
\[ Y_t = S_t +  T_t +  C_t +  R_t \]
\\ και η πολλαπλασιαστική: \\
\[ Y_t = S_t *  T_t *  C_t *  R_t \]


Αξίζει να σημειώσουμε ότι τα δύο μοντέλα έχουν μία λογαριθμική σχέση μεταξύ τους. Δηλαδή, αν πάρουμε τον λογάριθμο της προσθετικής σχέσης προκύπτει η πολλαπλασιαστική.

\subsection{Κλασική Μέθοδος Αποσύνθεσης με Κινητούς Μέσους Όρους}
Η κλασική μέθοδος αποσύνθεσης αποτελεί μια εύκολη διαδικασία για να διασπάσουμε τη χρονοσειρά στα τέσσερα δομικά της στοιχεία. Συχνά, τη συναντάμε στη βιβλιογραφία με την επωνυμία $X-11$ και αποτελείται από πέντε βασικά βήματα. Μάλιστα, όπως θα δούμε μπορεί να χρησιμοποιηθεί τόσο δεδομένης πολλαπλασιαστικής σχέσης όσο και προσθετικής μεταξύ των συστατικών στοιχείων της χρονοσειράς.

\subsubsection{Βήμα 1ο}
Αρχικά, απομονώνουμε την τάση και τον κύκλο της χρονοσειράς. Αυτό το καταφέρνουμε με το να υπολογίσουμε τον κινητό μέσο όρο της χρονοσειράς στο μήκος της εποχιακότητας. Το αποτέλεσμα δεν μεταφέρει την εποχιακή συμπεριφορά της αρχικής χρονοσειράς, ενώ παράλληλα εξαλείφεται σχεδόν και η τυχαιότητα, αφού οι τυχαίες διακυμάνσεις της χρονοσειράς χάνονται παίρνοντας τον μέσο όρο. Η σειρά που προκύπτει, λοιπόν, θεωρούμε ότι είναι η σειρά τάσης κύκλου και περιγράφεται από τη παρακάτω σχέση για το προσθετικό μοντέλο:
\[ KMO(n)= T + C \]
και την αντίστοιχη, για το πολλαπλασιαστικό:
\[ KMO(n)= T * C \]
όπου και στις δύο περιπτώσεις το $KMO(n)$ είναι ένας κινητός μέσος όρος μήκους $n$ και τα $T$ και $C$ όπως τα ορίσαμε προηγουμένως. Πρέπει να σημειωθεί ότι εν γένει προτιμάται η χρήση του κεντρικού κινητού μέσου όρου μήκους αντίστοιχου της εποχιακότητας, όταν αυτή είναι άρτια σε μία χρονοσειρά, αντί του απλού κινητού μέσου όρου. 

\subsubsection{Βήμα 2ο}
Είναι χρήσιμο σε αυτό το σημείο να παραγάγουμε τη χρονοσειρά των υπόλοιπων στοιχείων της αρχικής χρονοσειράς. Στο προσθετικό μοντέλο αυτό γίνεται με αφαίρεση της σειράς τάσης-κύκλου από τα αρχικά δεδομένα και με διαίρεση στο πολλαπλασιαστικό. Οπότε έχουμε αντίστοιχα τις παρακάτω σχέσεις:
\[ S+R = Y-(T+C)\]
\[ S*R = \frac{Y}{T*C}\]

\subsubsection{Βήμα 3ο}
Σε αυτό το βήμα αφαιρούμε την τυχαιότητα από τους λόγους εποχιακότητας που προέκυψαν στο 2ο Βήμα. Βρίσκοντας, λοιπόν τη μέση τιμή των λόγων που αναφέροναι σε αντίστοιχες περιόδους του εποχιακού κύκλου, δημιουργούμε τους δείκτες εποχιακότητας της αρχικής μας χρονοσειράς. Πρέπει, το άθροισμα τον εν λόγω λόγων να ισούται με το μήκος του εποχιακού μας κύκλου. Λόγου χάρη, σε περίπτωση μηνιαίων δεδομένων αυτό αντιστοιχεί στο ένα έτος, δηλαδή 12.
Αν αυτό δε συμβαίνει, θα χρειαστεί να κανονικοποιήσουμε τα δεδομένα.

Στη πράξη, πολλές φορές, χρειάζεται να διαχειριστούμε χρονοσειρές που παρουσιάζουν μεγάλη τυχαιότητα και ασυνήθιστες τιμές. Σε αυτές τις περιπτώσεις, μπορούμε να παραλείψουμε τη μέγιστη και την ελάχιστη τιμή στον υπολογισμό του μέσου όρου έτσι ώστε να επιτευχθεί μια σταθεροποίηση του αποτελέσματος.

\subsubsection{Βήμα 4ο}
Για να προσδιορίσουμε την αποεποχικοποιημένη σειρά εργαζόμαστε ως εξής:

Για το προσθετικό μοντέλο:
\[Y - S= T\times C\times S\times R - S \]
και την αντίστοιχη, για το πολλαπλασιαστικό:
\[\frac{Y}{S}= \frac{T\times C\times S\times R}{S} \]

Δηλαδή, στη πρώτη περίπτωση, αφαιρούμε τους δείκτες από τις αρχικές τιμές, ενώ στη δεύτερη τους διαιρούμε.

\subsubsection{Βήμα 5ο}
Για να αφαιρέσουμε την τυχαιότητα από το αποτέλεσμα του προηγούμενου βήματος χρησιμοποιούμε τον κινητό μέσο όρο μήκους 3 ή 6, ή διπλού κινητού μέσου όρου $3\times3$ σε αυτό. Η προκύπτουσα χρονοσειρά  είναι μία ομαλή και ακριβής σειρά τάσης-κύκλου. Μπορούμε εύκολα να πάρουμε την σειρά της τυχαιότητας αντίστοιχα με προηγουμένως.

Για το προσθετικό μοντέλο:
\[T\times C\times R - KMO(3\times3) = T\times C\times R - T\times C = R  \]
και την αντίστοιχη, για το πολλαπλασιαστικό:
\[\frac{T\times C\times R}{KMO(3\times3)}= \frac{T\times C\times R}{T\times C} = R  \]

\subsubsection{Βήμα 6ο}
Έχουμε, λοιπόν, καταφέρει ως τώρα να λάβουμε μια χρονοσειρά που περιέχει τάση και κυκλικότητα. Υπάρχουν περιπτώσεις που έχει κάποιο νόημα να διαχωρίσουμε ατά τα στοιχεία. Τότε πρέπει να επιλέξουμε κατάλληλο μοντέλο τάσης και να το απαλείψουμε από τη χρονοσειρά $T\times C$. Έτσι, έστω ότι η χρονοσειρά μας παρουσιάζει γραμμική τάση, εφαρμόζουμε το μοντέλο της απλής γραμμικής παλινδρόμησης. Η προκύπτουσα ευθεία περιγράφει την τάση, ενώ αν την αφαιρέσουμε από την $T \times C$ λαμβάνουμε τον κύκλο. Προφανώς, το παραπάνω ισχύει για το προσθετικό μοντέλο ενώ στο πολλαπλασιαστικό μοντέλο παίρνουμε τον κύκλο με διαίρεση των σειρών. 



\subsection{Διαφορές των δύο μοντέλων και κατάλληλη επιλογή}
Στο προσθετικό μοντέλο αποσύνθεσης οι μεταβολές που οφείλονται σε κάθε δομικό χαρακτηριστικό της χρονοσειράς εφαρμόζονται ανεξάρτητα για να τη συνθέσουν. Αντίθετα, στο πολλαπλασιαστικό μοντέλο τα στοιχεία της σειράς συσχετίζονται μεταξύ τους. Ως αποτέλεσμα, η εποχιακότητα μιας χρονοσειράς που παρουσιάζει έντονο το στοιχείο της τάσης θα έχει αντίστοιχα μεγαλύτερη ένταση σε περίπτωση που χρησιμοποιήσουμε το πολλαπλασιαστικό μοντέλο, καθώς έχουμε αύξηση της εποχιακής διακύμανσης όσο το επίπεδο της χρονοσειράς αλλάζει στον χρόνο. Χρησιμοποιώντας το προσθετικό μοντέλο έχουμε προσθήκη της ίδιας εποχιακής διακύμανσης ανεξάρτητα από το επίπεδο που θα μας οδηγήσει σε ομοιόμορφη εφαρμογή της εποχιακότητας. 

Μπορούμε να χρησιμοποιήσουμε την εξής μέθοδο για να αποφασίσουμε ποια προσέγγιση στην αποσύνθεση ταιριάζει περισσότερο στη χρονοσειρά μας: Αφαιρούμε από την αρχική χρονοσειρά το στοιχείο της εποχιακότητας με χρήση του πολλαπλασιαστικού μοντέλου και προσαρμόζουμε στη προκύπτουσα χρονοσειρά την ευθεία της απλής γραμμικής παλινδρόμησης. Στη περίπτωση που εντοπίσουμε στατιστική σημαντικότητα στη κλίση παραμέτρου της ευθείας του μοντέλου παλινδρόμησης, θεωρούμε ότι η επιλογή μας ήταν σωστή, ειδάλλως εφαρμόζουμε το προσθετικό μοντέλο.


\subsection{Μέθοδοι συρρίκνωσης συντελεστών} 
Είναι μείζονος σημασίας να έχουμε υψηλή ακρίβεια στον υπολογισμό των δεικτών εποχιακότητας, έτσι ώστε να καταφέρουμε να παραγάγουμε και ακριβείς προβλέψεις. Το τελευταίο μπορεί να συμβεί μόνο αν εφαρμόσουμε το μοντέλο πρόβλεψής μας σε μία αποεποχικοποιημένη χρονοσειρά που είναι επαρκώς εξομαλυμένη. Για να το διασφαλίσουμε αυτό χρησιμοποιούμε μεθόδους συρρίκνωσης συντελεστών στους δείκτες εποχιακότητας της χρονοσειράς.

\subsubsection{Μέθοδος Συρρίκνωσης \en{James-Stein}}

Η βασική λειτουργία αυτής της μεθόδου περιγράφεται από την παρακάτω εξίσωση:
\[ S^{JS}_j = W^{JS} + (1 - W^{JS})S_j\]

Το $S$ δηλώνει τις τιμές των δεικτών εποχιακότητας και $S^JS_j$ είναι οι δείκτες της μεθόδου \en{James-Stein}. Για τον υπολογισμό του συντελεστή $W$ χρησιμοποιούμε τον παρακάτω τύπο:
\[W^{JS} = \frac{pos -3}{pos - 1}\frac{V}{V + A}\]

Όπου το μήκος του κύκλου εποχιακότητας δηλώνεται με $pos (periods of seasonality)$. Επίσης τα:

\[ V = \frac{1}{pos} \sum_{j=1}^{pos} \frac{\sum_{k = 1}^{K_j}(S_{jk} - S_j)^2}{K_j(K_j - 1)} \]

\[ A = \frac{\sum_{j = 1}^{K_j}(S_j - 1)^2}{pos - 1)} - V \]

είναι οι διαφορές λόγω δειγματοληπτικών σφαλμάτων. Το $S_{jk}$ δηλώνει τον λόγο της εποχιακότητας για την εποχή $j$ και κατά τον εποχιακό κύκλο $k$ και το $K_j$ δηλώνει το πλήθος των λόγων εποχιακότητας που έχουν υπολογιστή για την εν λόγω εποχή. Στη περίπτωση που βρούμε το $A$ να είναι αρνητικό, το θέτουμε ίσο με το μηδέν. 

\subsubsection{Μέθοδος Συρρίκνωσης \en{Lemon-Krutchkoff}}

Ο εποχιακός δείκτης \en{Lemon-Krutchkoff} βρίσκεται ως εξής:

\[S_{j^*}^{LK} = \sum_{j=1}^{pos}W_{j^*,j} . S_j \]

To $W_{j^*,j}$ υποδηλώνει τα βάρη των σχετικών πιθανοτήτων $L_{j^*,j}$ να παρατηρηθεί ο εκτιμητής $S_{j^*}$, δεδομένου $S_j$ που είναι ο πραγματικός παράγοντας. Έχουμε:


\[W_{j^*,j} = \frac{L_{j^*,j}}{\sum_{j=1}^{pos}L_{j^*,j}} \]

Θεωρούμε ότι η πιθανότητα $L_{j^*,j}$ ακολουθεί κανονική κατανομή με διακύμανση ίση με $\sigma = \sqrt{V}$ .

\section{\en{STL} : Μία μέθοδος αποσύνθεσης Εποχιακότητας/Τάσης βασισμένη στη μέθοδο \en{Loess}}
H \en{STL} είναι μια διαδικασία φιλτραρίσματος που μας επιτρέπει να αποδομήσουμε μια χρονοσειρά σε τρία στοιχεία: την τάση, την εποχιακότητα και τα εναπομείναντα στοιχεία. Εκφραζόμενα σε μία μαθηματική σχέση που η χρονοσειρά συμβολίζεται με Y και τα χαρακτηριστικά της με $T$, $S$ και $R$, αντίστοιχα. 
\[ Y = T + S + R\]

Οι σχεδιαστές της μεθόδου προσπάθησαν να ικανοποιήσουν τα παρακάτω κριτήρια:
\begin{enumerate}
  \item Η μέθοδος \en{STL} να χαρακτηρίζεται από απλό σχεδιασμό και εύκολη χρήση.
  \item Ύπαρξη ελαστικότητας κατά τον προσδιορισμό της ποσότητας διακύμανσης στα στοιχεία της τάσης και της εποχιακότητας.
  \item Προσδιορισμός του αριθμού παρατηρήσεων ανά κύκλο του στοιχείου εποχιακότητας σε οποιονδήποτε ακέραιο μεγαλύτερο του 1. 
  \item Δυνατότητα αποσύνθεσης χρονοσειρών με κενές τιμές.
  \item Ισχυρή τάση και εποχιακότητα που δεν αλλοιώνεται από τη μεταβατική, ανώμαλη συμπεριφορά που μπορεί να χαρακτηρίζει τα δεδομένα
  \item Εύκολη υλοποίηση σε υπολογιστικό περιβάλλον και ταχύ υπολογισμό της, ακόμα και για μεγάλες χρονοσειρές.
\end{enumerate}

\subsection{Ο ορισμός της μεθόδου \en{STL}}

\subsubsection{\en{Loess}}
Έστω ότι $x_i$ και $y_i$, με $i= 1,\dots,n$ είναι οι τιμές μίας ανεξάρτητης και μίας εξαρτημένης μεταβλητής, αντίστοιχα. Η καμπύλης παλινδρόμησης \en{Loess}, $g(x)$ είναι μία εξομάλυνση της $y$ δεδομένης της $x$ που μπορεί να υπολογιστεί για οποιαδήποτε τιμή της $x$ στο πεδίο ορισμού της εξαρτημένης μεταβλητής.  

H $g(x)$ υπολογίζεται ως εξής. Επιλέγουμε έναν θετικό ακέραιο $q$, έστω $ q \leq n$.
Οι τιμές του $q$ που είναι εγγύτερες στο $x$ επιλέγονται και δίνεται στη κάθε μία ένα \textit{βάρος γειτνίασης} βασισμένο στην απόσταση του από το $x$. Έστω $\lambda_q(x)$ η απόσταση του $q$-οστού πιο απομακρυσμένου $x_i$ από το $x$. Έστω, τώρα, ότι $W$ είναι η τρικυβική συνάρτηση βάρους:
\begin{equation}
W(u) =
\left\{
	\begin{array}{ll}
		(1 - u^3)^3  & \mbox{αν } 0 \leq u < 1 \\
		0 & \mbox{αν } u \geq 1
	\end{array}
\right.
\end{equation}

Το βάρος γειτνίασης για οποιοδήποτε $x_i$ είναι:
\[ v_i(x) = W(\frac{|x_i - x|}{ \lambda_q(x)})\]

Συνεπώς όσο πιο κοντά στο $x$ είμαστε τόσο μεγαλύτερο είναι το βάρος. Μάλιστα, αυτό μηδενίζεται για το q-οστό σημείο σε απόσταση. Στο επόμενο βήμα προσαρμόζουμε μια πολυωνυμική καμπύλη βαθμού $d$ στα δεδομένα με βάρος $v_i(x)$ στο σημείο $(x_i, y_i)$. Η τιμή της τοπικά προσαρμοσμένης πολυωνυμικής καμπύλης στο x είναι η $g(x)$. 

Τώρα σε περίπτωση που το $q > n$, η $\lambda_n(x)$ είναι η απόσταση από το x στο $x_i$ που βρίσκεται πιο μακρυά από αυτό και ορίζουμε το $\lambda_q(x)$ ως:
\[\lambda_q(x) = \lambda_n(x)\frac{q}{n}\]
και αντίστοιχα ορίζονται τα βάρη γειτνίασης. 

Για να χρησιμοποιήσουμε τη μέθοδο \en{Loess} πρέπει να επιλεχθούν κατάλληλα $d$ και $q$. Η μέθοδος επιλογής τους ξεφεύγει από τα πλαίσια αυτής της διπλωματικής.

Αξίζει να αναφερθεί βέβαια, ότι όσο το $q$ αυξάνεται η $g(x)$ γίνεται πιο ομαλή, όταν προσεγγίζει το άπειρο η $v_i(x)$ συγκλίνει στο ένα και η $g(x)$ προσεγγίζει το πολυώνυμο ελαχίστων τετραγώνων βαθμού $d$. Στη περίπτωση που η χρονοσειρά έχει ήπια καμπυλότητα είναι βάσιμο να πάρουμε τη διάσταση $d$ ίση με ένα, ενώ όταν παρουσιάζει έντονη καμπυλότητα, με κορυφές και κοιλάδες, η διάσταση βαθμού δύο αποτελεί καλύτερη επιλογή.

Έστω τώρα ότι κάθε παρατήρηση $(x_i(x), y_i(x))$ έχει ένα βάρος $\rho_i$ που εκφράζει την αξιοπιστία της παρατήρησης σε σχέση με τις υπόλοιπες. Ενσωματώνοντας αυτά τα βάρη στη διαδικασία εξομάλυνσης βελτιώνεται η στιβαρότητα(\en{robustness}) της μεθόδου \en{STL}.

\subsubsection{Στοιχεία σχεδίασης: εσωτερικός και εξωτερικός βρόχος}

Η μέθοδος \en{STL} εμπεριέχει δύο αναδρομικές διαδικασίες: έναν εσωτερικό βρόχο εμφωλευμένο μέσα σε έναν εξωτερικό. Σε κάθε πέρασμα του εσωτερικού βρόχου, τα στοιχεία εποχιακότητας και τάσης ανανεώνονται. Κάθε πέρασμα του εξωτερικού βρόχου εμπεριέχει την εκτέλεση του εσωτερικού κόμβου και ακολουθεί με τον υπολογισμό βαρών στιβαρότητας, τα οποία χρησιμοποιούνται στην επόμενη εκτέλεση του εσωτερικού κόμβου για να ελαφρύνουν την επιρροή όποιας μεταβατικής, ανώμαλης συμπεριφοράς στα στοιχεία τάσης και εποχιακότητας. 

Ας θεωρήσουμε, τώρα, τις χρονοσειρές που συνθέτουν οι παρατηρήσεις κάθε περιόδου, ή κύκλου, της εποχιακότητας, έτσι ώστε να έχουμε τόσες στο πλήθος όσο το μήκος της εποχιακότητας. Για παράδειγμα, ο κύκλος της εποχιακότητας είναι το έτος και έχουμε μηνιαία δεδομένα, παίρνουμε 12 τέτοιες χρονοσειρές, τη χρονοσειρά το Ιανουαρίου, του Μαρτίου και ούτω καθεξής. Ονομάζουμε αυτές τις χρονοσειρές \textit{υποσειρές-κύκλου}. 

\subsubsection{Ο εσωτερικός βρόχος}

Κάθε πέρασμα του εσωτερικού βρόχου περιλαμβάνει μία εποχιακή εξομάλυνση, που ανανεώνει το δομικό στοιχείο της εποχιακότητας, ακολουθούμενη από την εξομάλυνση της τάσης, που ανανεώνει το συστατικό της τάσης αντίστοιχα. Τα βήματα που περιγράφουν την διαδικασία είναι τα εξής:
\begin{enumerate}
  \item \textbf{Βήμα 1:} Αφαίρεση τάσης
  \item \textbf{Βήμα 2:} Εξομάλυνση υποσειράς-κύκλου
  \item \textbf{Βήμα 3:} Βαθυπερατό φιλτράρισμα της εξομαλυσμένης υποσειράς-κύκλου
  \item \textbf{Βήμα 4:} Αφαίρεση τάσης από την εξομαλυσμένη υποσειρά-κύκλου
  \item \textbf{Βήμα 5:} Αποεποχικοποίηση
  \item \textbf{Βήμα 6:} Εξομάλυνση τάσης
\end{enumerate}

Έτσι, στα βήματα 2,3 και 4 συναντάμε το κομμάτι εξομάλυνσης εποχιακότητας, ενώ στο βήμα 6 το κομμάτι εξομάλυνσης της τάσης. 

\subsubsection{Ο εξωτερικός βρόχος}

Έχοντας διατρέξει τον εσωτερικό βρόχο έχουμε λάβει προσεγγίσεις για στοιχεία τάσης και εποχιακότητας, τα $T_v$ και $S_v$ αντίστοιχα. Τότε το εναπομείναν στοιχείο βρίσκεται από τη σχέση:
\[ R_v = Y_v - T_v - S_v\]

Ορίζουμε για κάθε παρατήρηση $Y_v$ τα βάρη στιβαρότητας, που μας δείχνουν πόσο ακραία είναι η χρονοσειρά του εναπομείναντος στοιχείου, ως
\[\rho_v = B(\frac{|R_v|}{h})\]
όπου το $h$ ορίζεται ακολούθως
\[ h = 6*median(|R_v)\]
και το B είναι η διτετραγωνική συνάρτηση βάρους:

\begin{equation}
B(u) =
\left\{
	\begin{array}{ll}
		(1 - u^2)^2  & \mbox{αν } 0 \leq u < 1 \\
		0 & \mbox{αν } u > 1
	\end{array}
\right.
\end{equation}

Σε αυτό το σημείο επαναλαμβάνεται ο εσωτερικός βρόχος, αλλά στα βήματα 2 και 6 χρησιμοποιείται το βάρος εποχιακότητας. 

\subsubsection{Μετα-εξομάλυνση της εποχιακότητας}

Η παραπάνω εξομάλυνση δεν μας εγγυάται ότι η μετάβαση από ένα σημείο στο επόμενο στην υπολογισμένη εποχιακότητα θα είναι ομαλή. Υπάρχουν περιπτώσεις, όμως, που θέλουμε το στοιχείο εποχιακότητας της χρονοσειράς μας να χαρακτηρίζεται από ομαλές μεταβάσεις. Μία απλή αντιμετώπιση στο παραπάνω ζήτημα είναι, επιπρόσθετα, να εξομαλύνουμε το στοιχείο της εποχιακότητας με τη μέθοδο \en{Loess}. Οι εξομαλυμένες, πλέον, τιμές αποτελούν τη τελική μορφή της εποχιακότητας.

\subsection{Σχόλια επί της μεθόδου}

Η μέθοδος \en{STL} είναι μία πολύπλευρη και στιβαρή μέθοδος αποσύνθεσης. Μερικά πλεονεκτήματα που παρουσιάζει σε σχέση με άλλες μεθόδους αποσύνθεσης είναι:
\begin{itemize}
  \item Μπορεί να διαχειριστεί οποιαδήποτε τύπο εποχιακότητας.
  \item Το στοιχείο εποχιακότητας δύναται να αλλάζει όσο προχωράει ο παράγοντας του χρόνου και ο ρυθμός μεταβολής μπορεί να καθοριστεί από το χρήστη.
  \item Η ομαλότητα της σειράς τάσης-κύκλου μπορεί να καθοριστεί επίσης από τον χρήστη.
  \item Παρουσιάζει στιβαρότητα σε ακραίες τιμές με αποτέλεσμα περιστασιακές ιδιαίτερες παρατηρήσεις δεν επηρεάζουν την εκτίμηση της τάσης-κύκλου και της εποχιακότητας. 
\end{itemize}

Από την άλλη η μέθοδος δε μπορεί από μόνη της να διαχειριστεί μέρες διαπραγματεύσεων ή ημερολογιακές παραλλαγές.

Επίσης, αν και ο ορισμός που δώσαμε περιγράφει μία προσθετική αντιμετώπιση της αποσύνθεσης, η μέθοδος μπορεί εύκολα να τροποποιηθεί για πολλαπλασιαστικές χρονοσειρές.

\section{Μέθοδοι Πρόβλεψης με ενσωματωμένη την αποεποχικοποίηση}
	
Ενώ οι μέθοδοι που είδαμε ως τώρα είναι ανεξάρτητοι της διαδικασίας πρόβλεψης της χρονοσειράς, συνήθως τις χρησιμοποιούμε για να προετοιμάσουμε τη χρονοσειρά μας για το τελικό μας στόχο, τη πρόβλεψη. Αναλυτικά, διάφορες μέθοδοι πρόβλεψης χρονοσειρών θα μελετηθούν στο επόμενο κεφάλαιο, αλλά προς το παρόν θα εξετάσουμε δύο μεθόδους που έχουν ενσωματωμένη τη διαχείριση της εποχιακής συμπεριφοράς της χρονοσειράς.

\subsection{Η εποχιακή μέθοδος \en{Holt-Winters}}

Η μέθοδος \en{Holt} αποτελεί μία μέθοδο πρόβλεψης εκθετικής εξομάλυνσης που έχει χτιστεί ώστε να διαχειρίζεται χρονοσειρές που παρουσιάζουν γραμμική τάση. Στηριζόμενοι πάνω στην κλασική μέθοδο οι \en{Holt} και \en{Winters} την επέκτειναν έτσι ώστε να είναι ικανή να διαχειριστεί και την εποχιακότητα. Η νέα μορφή της μεθόδου αποτελείται συνολικά από τέσσερις εξισώσεις, μία για την πρόβλεψη, μία για το επίπεδο της χρονοσειράς, μία για την τάση και μία για την εποχιακότητα.

Παρόμοια με τις μεθόδους αποσύνθεσης που είδαμε στις προηγούμενες ενότητες η εποχιακή μέθοδος \en{Holt-Winters} δύναται να χρησιμοποιηθεί τόσο με προσθετική προσέγγιση όσο και με πολλαπλασιαστική. 

\subsubsection{Προσθετικό μοντέλο \en{Holt-Winters}}

Χρησιμοποιούμε την προσθετική μέθοδο όταν οι εποχιακές διακυμάνσεις παραμένουν επί το πλείστον σταθερές καθ' όλη τη χρονική έκταση της χρονοσειράς. Σε αυτή τη περίπτωση η η εποχιακότητα εκφράζεται κατά απόλυτη τιμή στη κλίμακα των παρατηρήσεων, και μπορούμε να λάβουμε την αποεποχικοποιημένη  χρονοσειρά αφαιρώντας το στοιχείο της εποχιακότητας.
Προσθέτοντας τις τιμές της εποχιακότητας για έναν κύκλο της το αποτέλεσμα πρέπει να είναι μηδέν.

Συμβολίζουμε το μήκος του κύκλου εποχιακότητας ως $m$ και τον ορίζοντα πρόβλεψης ως $h$. Ο ορίζοντας πρόβλεψης, μία έννοια που θα αναλύσουμε λεπτομερέστερα στο επόμενο κεφάλαιο, μας δείχνει πόσο βήματα θα προχωρήσει στο χρόνο το μοντέλο πρόβλεψης μας σε σχέση με τα δεδομένα της αρχικής χρονοσειράς. Έτσι, έχουμε τη προσθετική μέθοδο \en{Holt-Winters} να περιγράφεται από τις ακόλουθες εξισώσεις:
\[ y_{t+h|t} = l_t + hb_t + s_{t-m+h^+_m}\]
\[l_t = \alpha(y_t - s_{t-m} + (1 - \alpha)(l_{t-1} + b_{t-1}\]
\[b_t = \beta^*(l_t - l_{t-1}) + (1-\beta^*)b_{t-1}\]
\[s_t = \gamma(y_t - l_{t-1} - b_{t-1}) + (1-\gamma)s_{t-m}\]

Με $h^+_m$ να η θέση της παρατήρησης $h$ στον κύκλο εποχιακότητας m που ξεκινάει στο $h=1$. 
Έτσι εξασφαλίζουμε ότι κατά την διαδικασία πρόβλεψης χρησιμοποιούμε δείκτες εποχιακότητας που προέρχονται από τον τελευταίο εποχιακό κύκλο των παρατηρήσεων. Επίσης τα $\alpha, \beta^*$ και $ \gamma$ είναι οι παράμετροι εξομάλυνσης του μοντέλου μας. 
Στην εξίσωση επιπέδου της χρονοσειράς μας συναντάμε ένα σταθμισμένο μέσο μεταξύ την αποεποχικοποιημένης παρατήρησής μας και της μη-εποχιακής πρόβλεψης για τη χρονική στιγμή $t$. Όπως θα δούμε παρακάτω η εξίσωση της τάσης είναι παρόμοια με αυτή της γραμμικής μεθόδου \en{Holt} και τελικά η εξίσωση της εποχιακότητας εμπεριέχει ένα σταθμισμένο μέσο μεταξύ της παρούσας χρονικής στιγμής και του δείκτη εποχιακότητας της ίδιας περιόδου του προηγούμενου κύκλου εποχιακότητας. 

Κάνοντας πρόβλεψη ενός βήματος θεωρούμε το σφάλμα να είναι ίσο με:
\[e_t = y_t - (L_{t-1} + b_{t-1} + s_{t-m}) = y_t - y_{t|t-1} \]
και αντίστοιχα η μορφή της διόρθωσης σφάλματος των εξισώσεων εξομάλυνσης είναι:

\[l_t = l_{t-1} + b_{t-1} + \alpha e_t\]
\[b_t = b_{t-1} + \alpha \beta^* e_t\]
\[s_t = s_{t-m} + \gamma e_t\]

\subsubsection{Πολλαπλασιαστικό μοντέλο \en{Holt-Winters}}

Τη πολλαπλασιαστική μέθοδο τη χρησιμοποιούμε όταν οι διακυμάνσεις της εποχιακής συμπεριφοράς είναι ανάλογες του επιπέδου της χρονοσειράς. 
Εδώ, η εποχιακότητα εκφράζεται σε σχετική (ποσοστιαία) μορφή και μπορούμε να λάβουμε την αποεποχικοποιημένη χρονοσειρά διαιρώντας την αρχική μας σειρά με την εποχιακότητα.
Το άθροισμα των τιμών ενός κύκλου εποχιακότητας της πολλαπλασιαστικής προσέγγισης αθροίζει στο μήκος του κύκλου. Για παράδειγμα, σε ετήσιες χρονοσειρές με μηνιαία δεδομένα το άθροισμα των τιμών για όλους τους μήνες ενός χρόνου πρέπει να είναι 12.

Το σύστημα εξισώσεων της πολλαπλασιαστικής μεθόδου είναι:
\[ y_{t+h|t} = (l_t + hb_t)  s_{t-m+h^+_m}\]
\[l_t = \alpha\frac{y_t}{s_{t-m}} + (1 - \alpha)(l_{t-1} + b_{t-1}\]
\[b_t = \beta^*(l_t - l_{t-1}) + (1-\beta^*)b_{t-1}\]
\[s_t = \gamma\frac{y_t }{ l_{t-1} - b_{t-1}} + (1-\gamma)s_{t-m}\]

και το αντίστοιχο διόρθωσης σφάλματος:

\[l_t = l_{t-1} + b_{t-1} + \alpha \frac{e_t}{s_{t-m}}\]
\[b_t = b_{t-1} + \alpha \beta^* \frac{e_t}{s_{t-m}}\]
\[s_t = s_{t-m} + \gamma \frac{e_t}{l_{t-1}+b_{t-1}}\]

με το σφάλμα να είναι:
\[e_t = y_t - (L_{t-1} + b_{t-1}) s_{t-m} \]

\subsection{Εποχιακά Μοντέλα \en{ARIMA}}

Τα ολοκληρωμένα αυτοπαλινδρομικά μοντέλα κινητών μέσων όρν \en{ARIMA (Auto Regressive Integrated Moving Average)} είναι στοχαστικά μοντέλα ανάλυσης και πρόβλεψης της εξέλιξης μεγεθών. Σε αντίθεση με ντετερμενιστικά μοντέλα που θα δούμε στο επόμενο κεφάλαιο, τα μοντέλα \en{ARIMA} βασίζονται στον υπολογισμό της πιθανότητας που περιγράφει πως η τιμή ενός μεγέθους παίρνει τιμές σε ένα διάστημα. Οι \en{Box} και \en{Jenking} μελέτησαν εκτενώς αυτά τα μοντέλα και έτσι πολλές φορές τα συναντάμε στη βιβλιογραφία με τα ονόματα τους.

Ενσωματωμένο στα μοντέλα \en{ARIMA} είναι και το σφάλμα πρόβλεψης, δηλαδή ο τυχαίος παράγοντας,οι τιμές του μεγέθους που βρήκαμε στις προηγούμενες περιόδους και σχετικούς στοχαστικού παράγοντες. Μπορούμε να εκφράσουμε ένα μοντέλο \en{ARIMA} σαν γραμμικό συνδυασμό των παραπάνω παραγόντων, που ο βέλτιστος συνδυασμός μπορεί να παράξει τις καλύτερες προβλέψεις. Σε πραγματικά δεδομένα, δεν είναι εύκολο να τον εντοπίσουμε πάντα, αλλά μπορούμε να τον προσεγγίσουμε σε ικανοποιητικό βαθμό.

Βέβαια πρέπει να πληρούνται κάποιες προϋποθέσεις για να μπορεί να εφαρμοστεί ένα μοντέλου \en{ARIMA}.
\begin{itemize}
  \item Η χρονοσειρά πρέπει να είναι διακριτή
  \item Η χρονοσειρά πρέπει να είναι στάσιμη 
  \item Στόχος μας να είναι η βραχυπρόθεσμη πρόβλεψη 
\end{itemize}

\subsubsection{Μοντέλα \en{ARIMA} για εποχιακή πρόβλεψη}
Τα μοντέλα \en{ARIMA} δύνανται να μοντελοποιήσουν ένα μεγάλο εύρος εποχιακών δεδομένων και αυτό επιτυγχάνεται ενσωματώνοντας στο μοντέλο επιπρόσθετους εποχιακούς όρους. Εκφράζουμε τα παραπάνω ως εξής:
\[ARIMA (p, d, q) (P, D, Q)_m \]

Τα μικρά γράμματα δηλώνουν τα κλασικά μη εποχικά κομμάτια του μοντέλου, ενώ τα κεφαλαία δηλώνουν τα εποχιακά κομμάτια. Όπως και πριν το $m$ δηλώνει το μέγεθος του κύκλου εποχιακότητας. Τα εποχιακά κομμάτια του μοντέλου συμπεριλαμβάνουν όρους που είναι συναφείς με τους μη εποχιακούς, αλλά εμπεριέχουν backshifts του εποχιακού κύκλου. Οι εν λόγω όροι απλά πολλαπλασιάζονται με τους μη εποχιακούς.

\subsubsection{Συναρτήσεις αυτοσυσχέτισης και μερικής αυτοσυσχέτισης: \en{ACF/PACF}}

Τα μοντέλα \en{ARIMA} απαρτίζονται από ένα μοντέλο \en{autoregression (AR)} και ένα μοντέλο \en{moving averages (MA)}. 
Το εποχιακό κομμάτι οποιουδήποτε από τα δύο αυτά  μοντέλα θα γίνει εμφανές στις εποχιακές καθυστερήσεις των \en{PACF} και \en{ACF}. 
Έτσι, για να χρησιμοποιήσουμε τη σωστή εποχιακή σειρά για ένα μοντέλο, πρέπει να εστιάσουμε τη προσοχή μας στις εποχιακές καθυστερήσεις. Η μοντελοποίηση είναι παρόμοια με αυτή των μη εποχιακών δεδομένων, παρά το γεγονός ότι πρέπει να προσδιορίσουμε και τους εποχιακούς όρους που περιγράψαμε παραπάνω.


