\chapter{Εκτενής Περίληψη}
\label{chapabst}

Χρονοσειρά ονομάζουμε ένα σύνολο παρατηρήσεων που περιγράφουν την εξέλιξη της συμπεριφοράς ενός μεγέθους στο χρόνο. Θεωρούμε, μάλιστα, ότι αυτές οι παρατηρήσεις δεν είναι ανεξάρτητες μεταξύ τους χωρίς βέβαια αυτό να υπονοεί μια ντετερμινιστική σύνδεση μεταξύ αυτών. 

Οι χρονοσειρές μπορούν να αναλυθούν σε τέσσερις χαρακτηριστικές υποσειρές. Την τάση, που περιγράφει πως μεταβάλλεται χρονικά το επίπεδο της χρονοσειράς. Τη κυκλικότητα που μετράει την κατά περιόδους μεταβολή της χρονοσειράς που οφείλεται σε εξωγενείς συνθήκες. Την εποχιακότητα που αποτελεί το μοτίβο διακύμανσης που παρουσιάζει η χρονοσειρά και επαναλαμβάνεται ανά σταθερές χρονικές περιόδους. Ο αστάθμητος παράγοντας της τύχης μιας χρονοσειράς καλείται τυχαιότητα.

Υπάρχει ένα σύνολο μεθόδων που μας επιτρέπει να αποσυνθέσουμε τη χρονοσειρά στα επιμέρους της στοιχεία. Εν γένει αντιμετωπίζουμε τη χρονοσειρά είτε ως άθροισμα των συνθετικών της μονάδων, είτε ως γινόμενο. Η πρώτη προσέγγιση είναι η προσθετική, ενώ η δεύτερη η πολλαπλασιαστική. Χρησιμοποιώντας το στατιστικό εργαλείο των κινητών μέσων όρων μπορούμε να εφαρμόσουμε τη κλασική μέθοδο αποσύνθεσης είτε με την προσθετική είτε με την πολλαπλασιαστική προσέγγιση. Συμπληρωματικά, για να εξασφαλίσουμε υψηλή ακρίβεια στον υπολογισμό των δεικτών εποχιακότητας μπορούμε να ενσωματώσουμε στη διαδικασία υπολογισμού της εποχιακότητας της χρονοσειράς μία μέθοδο συρρίκνωσης συντελεστών όπως είναι η μέθοδος \en{James-Stein} ή η μέθοδος \en{Lemon-Krutchkoff}.

Μία άλλη μέθοδος αποσύνθεσης είναι η μέθοδος \en{STL}. Αποτελεί ουσιαστικά μία διαδικασία φιλτραρίσματος της χρονοσειράς που μας επιτρέπει να την αποδομήσουμε σε τρία βασικά χαρακτηριστικά: την τάση, την εποχιακότητα και τα εναπομείνοντα στοιχεία. Η διαδικασία βασίζεται στη μέθοδο \en{Loess}.

Εκτός των παραπάνω μεθόδων που είναι ανεξάρτητες της διαδικασίας της πρόβλεψης, αν και αποτελεί συνήθως τον τελικό στόχο, υπάρχουν μοντέλα που ενσωματώνουν την εποχιακή ανάλυση. Ένα τέτοιο παράδειγμα είναι η μέθοδος \en{Holt-Winters} που στηρίχθηκε στη μέθοδο εκθετικής εξομάλυνσης για χρονοσειρές γραμμικής τάσης και είναι ικανή να διαχειριστεί και την εποχιακότητα. Επίσης, τα ολοκληρωμένα αυτοπαλινδρομικά μοντέλα κινητών μέσω όρων \en{(ARIMA)} δύνανται να μοντελοποιήσουν ένα μεγάλο εύρος εποχιακών δεδομένων.

Γενικά η διαδικασία της πρόβλεψης είναι μια πολυβηματική διαδικασία που αποσκοπεί στη χρήση της γνώσης των παρελθοντικών παρατηρήσεων μιας χρονοσειράς για να εκτιμήσει πως αυτή θα εξελιχθεί στο μέλλον.

Αρχικά, πρέπει η χρονοσειρά να έρθει σε κατάλληλη μορφή για να μπορέσουμε να εφαρμόσουμε τις στατιστικές μεθόδους που την προεκτείνουν στο μέλλον. Έτσι, το πρώτο βήμα είναι να αναπαραστήσουμε γραφικά τα δεδομένα έτσι ώστε να αποκτήσουμε μια εποπτεία στα ποιοτικά της χαρακτηριστικά. Βάσει αυτών και της γνώσης που έχουμε εν γένει για τα δεδομένα, καλούμαστε στη συνέχεια να διαχειριστούμε τις ιδιομορφίες της χρονοσειράς. Αυτές μπορεί να είναι μη ιδανική δειγματοληψία, κενές τιμές ή μηδενικές. Επίσης, πολλές φορές η χρονοσειρά χρήζει ημερολογιακών προσαρμογών αφότου οι παρατηρήσεις της μπορούν να επηρεάζονται από τις ημέρες ανθρώπινης εργασίας και συνεπώς από τα σαββατοκύριακα ή και τις αργίες. 

Αφότου τα δεδομένα έχουν έρθει σε κατάλληλη μορφή και μπορούμε να τα αποσυνθέσουμε όπως περιγράψαμε προηγουμένως, χρειάζεται να τα προεκτείνουμε στο μέλλον. Μία πρώτη προσέγγιση είναι να θεωρήσουμε ότι κάθε μελλοντική στιγμή είναι ταυτόσημη με αυτή που προηγείται. Αυτή προσέγγιση περιγράφει τη μέθοδο \en{Naive} που συνήθως χρησιμοποιείται ως βάση σύγκρισης.

Η ανάλυση της παλινδρόμησης θεωρώντας τον χρόνο ανεξάρτητη μεταβλητή, αποτελεί μία άλλη μέθοδο προέκτασης της χρονοσειράς. Εν γένει αποσκοπεί στην εύρεση συσχετίσεων μεταξύ μιας εξαρτημένης μεταβλητής και μίας ή περισσότερων ανεξάρτητων μεταβλητών και γι' αυτό το λόγο εκτός από μοντέλο πρόβλεψης αυτή καθ' αυτή μπορεί να χρησιμοποιηθεί και ως υποβοήθημα για άλλες μεθόδους.

Τα μοντέλα εκθετικής εξομάλυνσης είναι απλά μοντέλα, εύκολα στη χρήση, με μικρές υπολογιστικές απαιτήσεις που έχουν την δυνατότητα να παράξουν ακριβείς προβλέψεις ακόμα και με σχετικά μικρό ιστορικό παρατηρήσεων. Τα συγκεκριμένα μοντέλα εξαρτώνται από τη μορφή της τάσης (Σταθερού επιπέδου, Γραμμικής, Εκθετικής ή Φθίνουσας τάσης) και από το πρότυπο εποχιακότητας (Χωρίς, Προσθετική, Πολλαπλασιαστική).

Η μέθοδος Θ βασίζεται στη μεταβολή των τοπικών καμπυλοτήτων μιας χρονοσειράς για να παράξει προβλέψεις. Η χρονοσειρά αναλύεται σε δύο ή περισσότερες γραμμές \en{Theta} και κάθε μία από αυτές προβλέπεται ξεχωριστά, είτε με το ίδιο είτε με διαφορετικό μοντέλο πρόβλεψης.

Αφότου έχουμε ολοκληρώσει τη διαδικασία της πρόβλεψης πρέπει να αξιολογήσουμε κατά πόσο το μοντέλο μας παρήγαγε ακριβείς προβλέψεις ή ποιο από τα μοντέλα που εφαρμόσαμε είναι καλύτερο. Για να το πετύχουμε αυτό χρησιμοποιούμε ένα σύνολο στατιστικών δεικτών αξιολόγησης της ακρίβειας. Αυτοί οι δείκτες μπορεί να εξαρτώνται από τη κλίμακα των δεδομένων ή να εκφράζονται από ποσοστιαία σφάλματα. Μπορούν να αντικατοπτρίζουν πραγματικά ή σχετικά μεγέθη, ή να είναι αποτέλεσμα κανονικοποίησης.

Στη παραπάνω διαδικασία, όμως, έχουμε πρόβλημα στη περίπτωση που η χρονοσειρά μας δεν διαθέτει αρκετά ιστορικά δεδομένα και συνεπώς δε μπορούμε να εξάγουμε το στοιχείο της εποχιακότητας. Η συνηθισμένη αντιμετώπιση είναι να θεωρήσουμε τη χρονοσειρά ως μη εποχιακή και να την προβλέψουμε ως τέτοια. Αλλά, έτσι ουσιαστικά δεν έχουμε απαλλάξει τη χρονοσειρά από τις εποχιακές της διακυμάνσεις και αυτές θα επηρεάσουν την ακρίβεια της πρόβλεψής μας. 

Ένας τρόπος να αντιμετωπίσουμε αυτό το ζήτημα είναι αν έχουμε ένα σύνολο χρονοσειρών που περιγράφουν συναφή μεγέθη. Έτσι, εκμαιεύουμε την πληροφορία για την εποχικότητα παρόμοιων χρονοσειρών και την χρησιμοποιούμε για να προβλέψουμε τις μικρότερες χρονοσειρές που φαίνεται να έχουν αντίστοιχη εποχιακή συμπεριφορά.

Για να το καταφέρουμε αυτό χρησιμοποιήθηκε η ακόλουθη μεθοδολογία. Αρχικά, φέρνουμε τα δεδομένα σε κατάλληλη μορφή για τις στατιστικές μεθόδους που θέλουμε να εφαρμόσουμε με τις συνηθισμένες τεχνικές διαχείρισης των ιδιομορφιών των χρονοσειρών και με χρήση της πληροφορίας που έχουμε για τα δεδομένα. Κατόπιν, χωρίζουμε τις χρονοσειρές σε δύο ομάδες: αυτές με επαρκή ιστορικά δεδομένα για να εξάγουμε την εποχιακή τους συμπεριφορά και τις υπόλοιπες.

Στη πρώτη κατηγορία, βρίσκουμε τους δείκτες εποχιακότητας με το κλασικό πολλαπλασιαστικό μοντέλο αποσύνθεσης. Στη συνέχεια, εφαρμόζουμε τον αλγόριθμο συσταδοποίησης \en{DBSCAN} στους υπολογισμένους δείκτες για να εντοπίσουμε αν υπάρχουν ομάδες με συνάφεια μεταξύ τους και να τις εντοπίσουμε. Για κάθε συστάδα υπολογίζουμε τους μέσους δείκτες εποχιακότητας.

Στη δεύτερη κατηγορία, υπολογίζουμε μία ψευδο-εποχιακότητα που θα χρησιμοποιήσουμε σαν κριτήριο συνάφειας για τις συστάδες που προέκυψαν προηγουμένως. Αν, λοιπόν, ένα διάνυσμα δεικτών έχει μικρότερη μέση τετραγωνική απόσταση από το μέσο όρο των εποχιακών δεικτών μίας συστάδας από την μέγιστη απόσταση που υπολογίζουμε από τις χρονοσειρές τις συστάδας και του μέσου όρου τους, τότε κατατάσσουμε την μικρή χρονοσειρά στην συστάδα.

Για τις μικρές χρονοσειρές που βρέθηκε να παρουσιάζουν κοντινή εποχιακή συμπεριφορά με κάποια συστάδα, χρησιμοποιούμε τους μέσους εποχιακούς δείκτες της συστάδας που ανήκουν ως την χαρακτηριστική τους εποχιακότητα. Κατόπιν, προβλέπουμε βάσει αυτής. Συγχρόνως, κάνουμε πρόβλεψη στα αρχικά δεδομένα και τελικώς συγκρίνουμε την ακρίβεια των δύο προσεγγίσεων σύμφωνα με το κανονικοποιημένο δείκτη μέσου απόλυτου σφάλματος.

Τα δεδομένα που έχουμε στη διάθεσή μας περιγράφουν τη στάθμη υγρού φυσικού αερίου σε ένα σύνολο δεξαμενών στη Γαλλία. Ο αρχικός σκοπός της ανάλυσης ήταν να εκτιμήσουμε πότε οι δεξαμενές αυτές θα αδειάσουν, έτσι ώστε να μπορεί η εταιρία διανομής να το αποτρέψει αλλά και να σχεδιάσει βέλτιστα τον ανεφοδιασμό τους. 

Για τη προετοιμασία τους ακολουθήσαμε τις εξής ενέργειες. Μετατρέψαμε τις αρνητικές και μηδενικές τιμές σε κενές τιμές. Μετατρέψαμε τα δεδομένα σε ημερήσια κρατώντας, στις περιπτώσεις πολλαπλών παρατηρήσεων εντός μίας ημέρας, τη μικρότερη από αυτές. Μετά συμπληρώσαμε τις κενές τιμές με τη μέθοδο της γραμμικής παρεμβολής βάσει του χρόνου.

Λόγω του ότι θέλουμε να προβλέψουμε πότε θα αδειάσει η κάθε δεξαμενή χρειαζόμαστε την χρονοσειρά ζήτησης. Γι' αυτό εφαρμόζοντας πρώτες διαφορές λαμβάνουμε την ημερήσια ζήτηση. Αφαιρέσαμε τους ανεφοδιασμούς και τους διαχειριστήκαμε ως κενές τιμές. Τέλος, έχοντας ως στόχο την μεσοπρόθεσμη ζήτηση, μετατρέψαμε τα δεδομένα μας σε μηνιαία, υπολογίζοντας τον μέσο όρο της ημερήσιας ζήτησης κάθε μήνα. 

Αφότου εφαρμόσαμε την αποεποχικοποίηση όπως περιγράφηκε προηγουμένως, εφαρμόσαμε τον αλγόριθμο \en{DBSCAN} που μας επέστρεψε μία συστάδα. Υπολογίσαμε τις μικρές χρονοσειρές που ανήκουν σε αυτή και εφαρμόσαμε τις δύο προσεγγίσεις για τη παραγωγή προβλέψεων. Για κάθε μία προσέγγιση, χρησιμοποιήσαμε έξι μοντέλα πρόβλεψης

Βρήκαμε ότι πράγματι βελτιώθηκε σημαντικά η ακρίβεια των προβλέψεών μας. Όλα τα μοντέλα πρόβλεψης είχαν μεγαλύτερη ακρίβεια, ενώ στο μεγαλύτερο ποσοστό των χρονοσειρών έδειξαν καλύτερα αποτελέσματα με την προτεινόμενη προσέγγιση. 


