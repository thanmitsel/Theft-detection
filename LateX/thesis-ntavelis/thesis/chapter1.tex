\chapter{Εισαγωγή}
\label{chap1}

Στις κλασικές στατιστικές μεθόδους πρόβλεψης χρονοσειρών η συνηθισμένη διαδικασία ακολουθεί μια συγκεκριμένη διαδικασία βημάτων. Αρχικά, προετοιμάζουμε τη χρονοσειρά για ανάλυση. Έπειτα, την αναλύουμε στις τέσσερις της συνιστώσες: την τάση, την εποχιακότητα, την κυκλικότητα και την τυχαιότητα. Η πρόβλεψη γίνεται στην αποεποχικοποιημένη χρονοσειρά και μετά ενσωματώνεται σε αυτή το στοιχείο της εποχιακότητας.

Όμως, τα εργαλεία για αποσύνθεση της χρονοσειράς που έχουμε στη διάθεσή μας απαιτούν η χρονοσειρά να έχει τουλάχιστον ένα πλήθος παρατηρήσεων. Σε αντίθετη περίπτωση, η χρονοσειρά προβλέπεται σαν να μην χαρακτηρίζεται από εποχιακή συμπεριφορά. Το πρόβλημα προκύπτει λοιπόν στην αδυναμία να συνυπολογίσουμε την επιρροή της εποχιακότητας στην πρόβλεψη χρονοσειρών με μικρό ιστορικό.

Παράλληλα, ζούμε σε μια εποχή που την χαρακτηρίζει αφθονία δεδομένων. Έτσι συναντάμε όλο και περισσότερο χρονοσειρές που είναι μέρος ενός ευρύτερου συνόλου που περιγράφει παρόμοια μεγέθη.


\section{Αντικείμενο της διπλωματικής}

Στη παρούσα διπλωματική θα προσπαθήσουμε να χρησιμοποιήσουμε τη πληροφορία που μας δίνεται από ένα μεγάλο σύνολο δεδομένων για να προβλέψουμε με μεγαλύτερη ακρίβεια μικρές χρονοσειρές.

Θα χωρίσουμε, λοιπόν, τις χρονοσειρές σε μικρές και μεγάλες, δηλαδή με επαρκές ιστορικό για αποεποχικοποίηση. Στις μεγάλες χρονοσειρές θα χρησιμοποιήσουμε τις γνωστές μεθόδους αποσύνθεσης για να παραγάγουμε τους δείκτες εποχιακότητας για κάθε μία από αυτές. Στις μικρές, θα υπολογίσουμε ένα σύνολο δεικτών ψευδο-εποχιακότητας.

Χρησιμοποιώντας τεχνικές συσταδοποίησης στη πρώτη ομάδα των χρονοσειρών θα ελέγξουμε αν υπάρχουν πράγματι μοτίβα εποχιακότητας που χαρακτηρίζουν υποσύνολα των δεδομένων και θα τα εντοπίσουμε. Μετά θα εξετάσουμε αν οι μικρές χρονοσειρές δύνανται να καταταχθούν σε κάποια από τις συστάδες που υπολογίσαμε βάσει των δεικτών ψευδο-εποχιακότητας τους.

Στη συνέχεια, θα προεκτείνουμε τις μικρές χρονοσειρές που βρήκαμε να ανήκουν σε κάποια συστάδα στο μέλλον με δύο τρόπους. Πρώτα, θα τις προβλέψουμε όπως γίνεται συνήθως. Έπειτα, θα τις αποεποχικοποιήσουμε βάσει της μέσης εποχιακότητας των μεγάλων χρονοσειρών που ανήκουν στην ίδια συστάδα με αυτές, θα τις προβλέψουμε και θα τις επαναεποχικοποιήσουμε.

Τελικά θα εκτιμήσουμε αν η ακρίβεια της προτεινόμενης μεθόδου είναι μεγαλύτερη από τη κλασική προσέγγιση.


\subsection{Συνεισφορά}

Η συνεισφορά της διπλωματικής συνοψίζεται ως εξής:

\begin{enumerate}
\item Μελετήθηκε ένα σύνολο χρονοσειρών που περιγράφει το επίπεδο φυσικού αερίου σε δεξαμενές
\item Έγινε ανάλυση τους και μετατράπηκαν σε μορφή για μεσοπρόθεσμη πρόβλεψη
\item Υπολογίστηκαν οι ομάδες συνάφειας των μοτίβων εποχιακότητας
\item Βάσει αυτών αποεποχικοποιήσαμε χρονοσειρές μικρού ιστορικού
\item Μετρήσαμε ότι η ακρίβεια πρόβλεψης βελτιώνεται με την προτεινόμενη μέθοδο αποεποχικοποίησης
\end{enumerate}


\section{Οργάνωση του τόμου}

Ακολουθεί εκτενής περίληψη της διπλωματικής στο Κεφάλαιο \ref{chapabst}. Στο Κεφάλαιο \ref{chap2} θα περιγράψουμε τι είναι μια χρονοσειρά, τα δομικά της στοιχεία και θα δωθεί ιδιαίτερη προσοχή στην εποχιακή της συμπεριφορά, πώς μπορούμε να την απομονώσουμε ή να την ενσωματώσουμε στο μοντέλο της πρόβλεψης. Η μεθοδολογία που ακολουθείται για τη πρόβλεψη μιας χρονοσειράς, δηλαδή η προετοιμασία της, η προέκτασή της στο μέλλον και τέλος η αξιολόγησή ακρίβειας αναλύονται στο Κεφάλαιο \ref{chap3}. Στο Κεφάλαιο \ref{chap4} παραθέτουμε αναλυτικά τα βήματα που ακολουθούμε για να αντιμετωπίσουμε το πρόβλημα των εποχιακών μικρών χρονοσειρών. Συγκεκριμένα, πώς εφαρμόζονται αυτά τα βήματα στο υπό εξέταση σύνολο δεδομένων φαίνεται στο Κεφάλαιο \ref{chap5}.  Στο Κεφάλαιο \ref{chap8} συνοψίζουμε τα αποτελέσματα και συζητάμε μελλοντικές επεκτάσεις.
Τέλος, στο παράρτημα, ακολουθούν οι γλώσσες προγραμματισμού, οι βιβλιοθήκες και τα περιβάλλοντα ανάπτυξης που χρησιμοποιήθηκαν για την υλοποίηση του πειράματος της διπλωματικής και η παράθεση των λεπτομερών αποτελεσμάτων μέτρησης της ακρίβειας.

