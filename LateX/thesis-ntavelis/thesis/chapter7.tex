\chapter{Τεχνικές λεπτομέρειες}
\label{chap7}

Σε αυτό το κεφάλαιο θα περιγράψουμε τα εργαλεία προγραμματισμού και στατιστικής ανάλυσης που χρησιμοποιήθηκαν για την υλοποίηση της παρούσας διπλωματικής εργασίας. Χρησιμοποιήθηκαν δύο γλώσσες προγραμματισμού, με ένα μεγάλο πλήθος διαφορετικών βιβλιοθηκών και δύο προγραμματιστικά περιβάλλοντα.

\section{Γλώσσες προγραμματισμού}

Για να υλοποιηθεί ο κώδικας επίλυσης του προβλήματος που άπτεται η εργασία χρησιμοποιήθηκαν δύο γλώσσες προγραμματισμού: η \en{Python} και η \en{VB.NET}.

\subsection{\en{Python}}

Στη παρούσα εργασία χρησιμοποιήθηκε η γλώσσα \en{Python} καθότι προσφέρει μία πληθώρα εργαλείων για στατιστική ανάλυση και μηχανική μάθηση. Άλλωστε είναι από τις πιο διαδεδομένες γλώσσες γι' αυτό το σκοπό. Συγκεκριμένα χρησιμοποιήθηκαν οι βιβλιοθήκες που περιγράφονται ακολούθως, που είναι όλες ανοικτού κώδικα.

\subsubsection{\en{NumPy}}

Το πακέτο \en{NumPy} είναι βασικό πακέτο της γλώσσας \en{Python} που επιτρέπει προγραμματισμό για επιστημονική χρήση. Επιτρέπει τη διαχείριση δεδομένων μεγάλων διαστάσεων και προσφέρει μεγάλο πλήθος συναρτήσεων για την επεξεργασία τους.

\subsubsection{\en{Pandas}}

Το πακέτο \en{Pandas} είναι ιδανικό για την ανάλυση χρονοσειρών. Προσφέρει εύκολες στη χρήση δομές δεδομένων, και επιτρέπει την άμεση οπτικοποίηση και επεξεργασία τους.

\subsubsection{\en{Scikit-learn}}

Το \en{Scikit-learn} προσφέρει ένας πλήθος αλγορίθμων Μηχανικής Μάθησης. Συγχρόνως, παρέχει απλά και αποδοτικά εργαλεία για εξόρυξη και ανάλυση δεδομένων.

\subsubsection{\en{Matplotlib}}

Το \en{Matplotlib} είναι μία βιβλιοθήκη της \en{Python} που παράγει υψηλής ποιότητας γραφήματα. Βρίσκεται σε πλήρη αρμονία με τις προηγούμενες βιβλιοθήκες-πακέτα, ενώ δίνει στο χρήστη πολλές επιλογές για παραμετροποίηση.

\subsection{\en{VB.NET}}

Η γλώσσα \en{VB.NET} είναι μία αντικειμενοστραφής γλώσσα προγραμματισμού που μας επέτρεψε να χτίσουμε μία βιβλιοθήκη που διαχειρίζεται πλήρως τις χρονοσειρές και τα μοντέλα πρόβλεψης ως αντικείμενα που απορρέει το ένα από το άλλο. Έτσι, η εφαρμογή των μοντέλων πρόβλεψης της παρούσας εργασίας έγινε αποκλειστικά μέσω της \en{VB.NET}.

\section{Πλατφόρμες και προγραμματιστικά εργαλεία}

Χρησιμοποιήθηκαν δύο διαφορετικά περιβάλλοντα για να αναπτύξουμε τον κώδικα της διπλωματικής, ένα για το κομμάτι που αναπτύχθηκε σε \en{Python} και ένα για αυτό σε \en{VB.NET}.

\subsection{\en{Jupyter Notebook}}

Το \en{Jupyter Notebook} αποτελεί μία πλατφόρμα ανοικτού κώδικα που τρέχει στο πρόγραμμα περιηγητή (\en{browser}) και επιτρέπει τη δημιουργία και διαμοιρασμό αρχείων που περιέχουν κώδικα, εξισώσεις, οπτικοποίηση δεδομένων και επεξηγηματικό κείμενο. Κάνει εύκολη τη επεξεργασία δεδομένων, την αριθμητική προσομοίωση, τη στατιστική μοντελοποίηση και την εφαρμογή μεθόδων μηχανικής μάθησης μεταξύ άλλων. Όλα αυτά, δίνει τη δυνατότητα να γίνουν με έναν διαδραστικό τρόπο.

\subsection{\en{Visual Studio 2013}}

Η ανάπτυξη της βιβλιοθήκης προβλέψεων έγινε στη πλατφόρμα του \en{Visual Studio 2013}. Η πλατφόρμα αποτελεί ένα ολοκληρωμένο περιβάλλον ανάπτυξης (\en{integrated development environment (IDE)}) που προσφέρει εργαλεία για

\begin{itemize}
    \item Σχεδιασμό
    \item Ανάπτυξη Κώδικα
    \item Χτίσιμο και Ανάπτυξη Εφαρμογών
    \item Αποσφαλμάτωση \en{(Debugging)}
    \item Συνεργασία Προγραμματιστών
\end{itemize}



