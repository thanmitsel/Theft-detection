\begin{acknowledgements}

Θα ήθελα να ευχαριστήσω τον επιβλέποντα καθηγητή κ. Βασίλειο Ασημακόπουλο για την ευκαιρία που μου έδωσε να εκπονήσω τη παρούσα διπλωματική και την υποστήριξή του σε όλη την πορεία της.

Επίσης, θα ήθελα να ευχαριστήσω  τους καθηγητές κ. Ιωάννη Ψαρρά και κ. Δημήτριο Ασκούνη για την τιμή που μου έκαναν να συμμετάσχουν στην επιτροπή εξέτασης της διπλωματικής.

Eυχαριστώ ιδιαίτερα τον υποψήφιο διδάκτορα Ευάγγελο Σπηλιώτη για την καθοδήγηση, στήριξη και καθοριστική βοήθεια που μου παρείχε, όπως και τα υπόλοιπα μέλη της Μονάδας Προβλέψεων και Στρατηγικής.

Θερμές ευχαριστίες θα ήθελα να απευθύνω στον Δρ Χριστόφορο Αναγνωστόπουλο και την εταιρία \en{Mentat Innovations} για την καθοδήγησή τους στα πρώτα βήματα αυτής της εργασίας.

Τέλος, θα ήθελα να ευχαριστήσω την οικογένειά μου και τους φίλους μου Γιώργο, Γρηγόρη, Κατερίνα και Μαρία.

\end{acknowledgements}


\begin{abstract}
Αντικείμενο της διπλωματικής εργασίας είναι η ανάπτυξη μεθοδολογίας για τη βελτίωση της ακρίβειας στατιστικών μεθόδων πρόβλεψης σε χρονοσειρές που έχουν μικρό ιστορικό παρατηρήσεων μέσω τεχνικών συσταδοποίησης εποχιακών δεικτών από συναφείς χρονοσειρές.

Οι κλασικές μέθοδοι αποσύνθεσης απαιτούν ένα ελάχιστο πλήθος παρατηρήσεων για να μπορέσουν να εξάγουν το μοτίβο της εποχιακότητας μιας χρονοσειράς. Στη πράξη, όμως, συναντάμε συχνά χρονοσειρές που αποτελούνται από μικρό πλήθος τιμών, ενώ συγχρόνως περιγράφουν εποχιακά μεγέθη.

Παράλληλα, τα τελευταία χρόνια υπάρχει αφθονία στα δεδομένα που έχουμε στη διάθεσή μας. Η παρούσα εργασία βασίζεται στην υπόθεση ότι μπορούμε να χρησιμοποιήσουμε τη διαθέσιμη πληροφορία για να εξάγουμε αντιπροσωπευτικούς δείκτες εποχιακότητας που μπορούμε να χρησιμοποιήσουμε για να αναλύσουμε και να προεκτείνουμε χρονοσειρές που χαρακτηρίζονται από μικρό ιστορικό.

Για να το κάνουμε αυτό πρέπει αρχικά να συγκεντρώσουμε ένα πλήθος χρονοσειρών που περιγράφει παρόμοια φυσικά μεγέθη. Έπειτα, χρησιμοποιώντας τεχνικές συσταδοποίησης στους δείκτες εποχιακότητας αυτών που έχουν επαρκή δεδομένα για να εφαρμόσουμε τις κλασικές μεθόδους αποσύνθεσης, δημιουργούμε συστάδες παρόμοιας εποχιακής συμπεριφοράς. Ελέγχουμε, κατόπιν, αν οι μικρές χρονοσειρές μπορούν να υπαχθούν σε αυτές τις συστάδες και αν ναι, τις προβλέπουμε με δεδομένο ότι οι δείκτες εποχιακότητας τους είναι οι ίδιοι με τους μέσους δείκτες των συστάδων.

Για να ελέγξουμε την υπόθεση, εφαρμόσαμε την μεθοδολογία που περιγράφηκε σε ένα σύνολο χρονοσειρών ζήτησης φυσικού αερίου και λάβαμε θετικά αποτελέσματα. Συγκεκριμένα συγκρίναμε τη προτεινόμενη προσέγγιση με τη κλασική, που προεκτείνει τις μικρές χρονοσειρές βάσει των αρχικών τους δεδομένων και παρατηρήσαμε σημαντική βελτίωση της ακρίβειας.

\begin{keywords}
  Χρονοσειρές, Τεχνικές Προβλέψεων, Εποχιακότητα, Συσταδοποίηση, Μικρό ιστορικό, Φυσικό Αέριο.
 
\end{keywords}

\end{abstract}



\begin{abstracteng}
\tl{The purpose of this diploma thesis is to develop a methodology for improving the accuracy of statistical forecasting methods on timeseries with short history through the use of clustering techniques on the seasonal indices of other similar timeseries.
}

\tl{Classical decomposition methods require a minimum number of observations to be able to detect the seasonality pattern of a timeseries. In practice, however, we often encounter timeseries lacking enough data, while at the same time describing seasonal values. }

\tl{Meanwhile, in recent years, there is an abundance of accessible data. This thesis draws upon the hypothesis that we can utilise the available infomation to extract representative seasonality indices that we can use in order to analyse and extend timeseries that are characterised by short history.}

\tl{ In order to achieve this, we initially have to gather a large number of timeseries describing similar values.  Afterwards, we create clusters of similar seasonal behaviour by using clustering techniques on the seasonality indices of series with sufficient data. Then, we check if the shorter timeseries qualify to be a part of these clusters and if so, we predict their future values as they were characterised by the seasonal behaviour of the mean indices of the cluster members.}

\tl{ To test our hypothesis, we applied the described methodology to a set of natural gas demand timeseries and received positive results. In particular, we compared the proposed approach to the classical one, which forecasts short timeseries based on their original data, and we have measured a significant overall improvement in accuracy.}
\begin{keywordseng}
  \tl{Timeseries, Forecasting Techniques, Seasonality, Clustering, Short history, Natural Gas.}
\end{keywordseng}

\end{abstracteng}
