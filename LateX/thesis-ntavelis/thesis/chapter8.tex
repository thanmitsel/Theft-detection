\chapter{Επίλογος}
\label{chap8}

Σε αυτό το κεφάλαιο θα σκιαγραφήσουμε την πορεία ανάλυσης των δεδομένων που μας οδήγησαν εν τέλει στη διαδικασία της πρόβλεψης και επικύρωσης της υπόθεσής μας. Θα συζητήσουμε τα αποτελέσματα που προέκυψαν και τι περαιτέρω βήματα μπορούν να γίνουν για να ισχυροποιήσουν αυτά τα αποτελέσματα. 

\section{Σύνοψη και συμπεράσματα}

Στις κλασική μεθοδολογία πρόβλεψης μιας χρονοσειράς, αφού την προετοιμάσουμε την αποσυνθέτουμε στα βασικά της χαρακτηριστικά: τη τάση, τον κύκλο, την τυχαιότητα και την εποχιακότητα. Αφότου έχουμε υπολογίσει τους εποχιακούς δείκτες, τους χρησιμοποιήσουμε για να απαλλάξουμε τα αρχικά δεδομένα από τις αλλαγές που υφίστανται λόγω της περιόδου που βρίσκονται μέσα στο μήκος του μοτίβου της εποχιακότητας. Έτσι προκύπτει μία πιο σταθερή πλέον χρονοσειρά, αφού της έχουμε αφαιρέσει τις προβλεπόμενες διακυμάνσεις. Αυτή τη χρονοσειρά, λοιπόν, χρησιμοποιούμε ως είσοδο στο μοντέλο της πρόβλεψής μας. Κατόπιν, λαμβάνουμε το αποτέλεσμα του μοντέλου και ενσωματώνουμε την εποχιακή συμπεριφορά για να προκύψει η τελική πρόβλεψη. 

Το πρόβλημα εμφανίζεται όταν η υπό εξέταση χρονοσειρά δεν χαρακτηρίζεται από επαρκώς μεγάλο ιστορικό για να μας επιτρέψει να εξάγουμε το μοτίβο της εποχιακότητάς της. Σε αυτή τη περίπτωση, η συνηθισμένη προσέγγιση είναι να αντιμετωπίζουμε τη χρονοσειρά ως μη εποχιακή και να τη προβλέπουμε χωρίς να κάνουμε ανάλυση εποχιακότητας. Όμως, οι εποχιακές διακυμάνσεις, σε περίπτωση που η σειρά περιέγραφε πράγματι ένα εποχιακό μέγεθος, συνεχίζουν να χαρακτηρίζουν τα δεδομένα και τελικώς επηρεάζουν την ακρίβεια των μοντέλων.

Παράλληλα, όσο περνάνε τα χρόνια το πλήθος των δεδομένων που εν γένει έχουμε στη διάθεσή μας αυξάνεται εκθετικά. Αυτό το γεγονός ανοίγει δρόμους σε νέες προσεγγίσεις για να προσπαθήσουμε να αυξήσουμε την ακρίβεια της διαδικασίας της πρόβλεψης. Έτσι, θεωρώντας ότι έχουμε ένα μεγάλο πλήθος χρονοσειρών που περιγράφουν παρόμοιας φύσης δεδομένα, υποθέσαμε ότι μπορούμε να εκμαιεύσουμε πληροφορία για την εποχιακή συμπεριφορά των δεδομένων της συγκεκριμένης φύσης.

Δοκιμάσαμε την υπόθεσή μας σε ένα σύνολο δεδομένων χρονοσειρών από τα δεδομένα για τον όγκο υγρού φυσικού αερίου σε δεξαμενές. Στόχος ήταν να μπορέσουμε να εκτιμήσουμε σε ποια χρονική στιγμή στο μέλλον κάθε δεξαμενή θα αδειάσει με σκοπό να το αποτρέψουμε, έτσι ώστε να μην υπάρξει πρόβλημα στους πελάτες της εταιρίας που παρέχει το φυσικό αέριο αλλά και να της δώσουμε τη γνώση που χρειάζεται για να σχεδιάσει βέλτιστα τον ανεφοδιασμό των δεξαμενών.

Πρώτο βήμα στην ανάλυση ήταν να καθαρίσουμε τα δεδομένα μας και να τα φέρουμε στη μορφή που θα μας βοηθήσει να πετύχουμε τον στόχο μας. Φροντίσαμε, λοιπόν, να τα φέρουμε όλα σε ημερήσια συχνότητα και να διαχειριστούμε τις κενές και μηδενικές τιμές. Κατόπιν, είδαμε ότι για να προβλέψουμε πότε θα καταναλωθεί όλο το υγρό φυσικό αέριο στις δεξαμενές μας ενδιαφέρει ο ρυθμός κατανάλωσης σε κάθε μία από αυτές. Έτσι, παίρνοντας τις διαφορές μεταξύ δύο διαδοχικών ημερών στον όγκο του υγρού, λάβαμε την ημερήσια κατανάλωση. Σκοπεύοντας να προβλέψουμε μεσοπρόθεσμα το πότε θα αδειάσει μία δεξαμενή, υπολογίσαμε τον μέσο όρο της ημερήσια κατανάλωσης για κάθε μήνα και προέκυψε η μηνιαία χρονοσειρά της μέσης κατανάλωσης. 

Τελειώνοντας, λοιπόν, το πρώτο βήμα της ανάλυσης καταλήξαμε να έχουμε ένα σύνολο χρονοσειρών που περιγράφουν την μέση μηνιαία κατανάλωση ανά δεξαμενή. Όλες αυτές η χρονοσειρές μετράνε παρόμοια μεγέθη, ενώ ανάμεσά τους υπάρχει ένα ποσοστό που δεν έχουν επαρκή δεδομένα για να εξάγουμε την εποχιακή τους συμπεριφορά. Από την άλλη, η κατανάλωση φυσικού αερίου, που είναι συνδεδεμένη άμεσα με την θερμοκρασία αφότου μία κύρια χρήση του είναι η θέρμανση, περιμένουμε να χαρακτηρίζεται από εποχιακές διακυμάνσεις στο μήκος του χρόνου. 

Έτσι, διαμερίσαμε τις χρονοσειρές σε αυτές που μπορούμε να χρησιμοποιήσουμε τις κλασικές μεθόδους αποσύνθεσης και σε αυτές που δεν μπορούμε. Στη πρώτη κατηγορία τις εφαρμόσαμε κανονικά και λάβαμε τους δώδεκα ετήσιους δείκτες εποχιακότητας για κάθε μία εξ' αυτών. Για τη δεύτερη κατηγορία αφαιρέσαμε από την χρονοσειρά τις τιμές της ευθείας γραμμικής παλινδρόμησης που αντιστοιχούσαν σε κάθε παρατήρηση, και από το αποτέλεσμα πήραμε τον μέσο όρο των τιμών για κάθε περίοδο. Προέκυψαν, συνεπώς, δώδεκα τιμές που ονομάσαμε δείκτες ψευδο-εποχιακότητας.

Επόμενο βήμα ήταν να ελέγξουμε αν πράγματι υπήρχαν κοινά μοτίβα στην εποχιακότητα των χρονοσειρών της πρώτης κατηγορία και να τα εντοπίσουμε. Για αυτό το λόγο χρησιμοποιήσαμε τον αλγόριθμο συσταδοποίησης \en{DBSCAN}. Βρήκαμε, ότι αν και πολλές χρονοσειρές δεν ακολουθούσαν συγκεκριμένο μοτίβο, υπήρχε ένα σύνολο εξ αυτών που παρουσίαζε συνάφεια στους εποχιακούς του δείκτες. Γι' αυτό το μοτίβο υπολογίσαμε την μέση εποχιακότητα.

Για να ελέγξουμε αν οι χρονοσειρές με μικρό ιστορικό ανήκουν στην συστάδα που προέκυψε, χρησιμοποιήσαμε το κριτήριο με το οποίο ο αλγόριθμος \en{k-means} κατατάσσει τα δεδομένα που πραγματεύεται σε συστάδες. Έτσι, για το σύνολο των χρονοσειρών με συνάφεια στους δείκτες εποχιακότητας, υπολογίσαμε τη μέγιστη μέση τετραγωνική διαφορά του κέντρου από τους δείκτες των υπόλοιπων του χρονοσειρών. Αν η μέση τετραγωνική διαφορά μεταξύ των δεικτών ψευδο-εποχιακότητας και του κέντρου ήταν μικρότερη από το αποτέλεσμα, θεωρήσαμε ότι μπορεί να αντιπροσωπευθεί η εποχιακότητα των χρονοσειρών τους από αυτή του κέντρου.

Έχοντας στη διάθεση μας ένα σύνολο χρονοσειρών που πληρούσε πλήρως τα κριτήρια που θέσαμε για την υπόθεσή μας περάσαμε στη διαδικασία της πρόβλεψης. Αρχικά, προβλέψαμε τις χρονοσειρές κατά την κλασική προσέγγιση: τις προεκτείναμε στο μέλλον βάσει των αρχικών τους δεδομένων. Έπειτα, αποεποχικοποιήσαμε αυτές τις χρονοσειρές, με τους μέσους δείκτες εποχιακότητας της συστάδας, και εφαρμόσαμε τα μοντέλα πρόβλεψης κατόπιν. Στη δεύτερη περίπτωση, πολλαπλασιάσαμε πάλι τους δείκτες εποχιακότητας με το προϊόν των μοντέλων.

Σε ποσοστό μεγαλύτερο του 70\% για κάθε μέθοδο πρόβλεψης είχαμε η προτεινόμενη μέθοδος να έχει μεγαλύτερη ακρίβεια, σημειώνοντας χαμηλότερη τιμή σφάλματος σε κάθε μία από τις μεθόδους που εξετάσαμε. Το αποτέλεσμα αυτό επαλήθευσε την υπόθεσή μας και έδειξε τη σημασία της διαχείρισης της εποχιακότητας των χρονοσειρών ακόμα και όταν αυτές έχουν μικρό ιστορικό.

Από τις μεθόδους που χρησιμοποιήσαμε μεγαλύτερη βελτίωση είδαμε στην ακρίβεια της μεθόδου της γραμμικής παλινδρόμησης. Παράλληλα, είχαμε το μεγαλύτερο ποσοστό των χρονοσειρών που πήγε καλύτερα η προτεινόμενη μέθοδος.  Αυτό είναι λογικό, καθότι η μέθοδος υπολογίζει την τάση της χρονοσειράς και την προεκτείνει στο μέλλον. Δεδομένου ότι οι υπό εξέταση χρονοσειρές έχουν μικρό ιστορικό ενώ χαρακτηρίζονται από εποχιακή συμπεριφορά, η γραμμή ελαχίστων τετραγώνων επηρεάζεται από τις κορυφές τις χρονοσειράς και τις βλέπει σαν αύξηση της τάσης. 

Τόσο η προτεινόμενη όσο και η κλασική προσέγγιση είχαν μεγαλύτερη ακρίβεια για τη μέθοδο \en{Naive}. Αυτό συμβαίνει διότι σε τόσο μικρό αριθμό δεδομένων η επιρροή της τυχαιότητας είναι πολύ μεγάλη κατά την πρόβλεψη.

\section{Μελλοντικές επεκτάσεις}

Στη παρούσα εργασία είδαμε ότι εκμεταλλευόμενοι την πληροφορία που μας δίνεται από έναν μεγάλο όγκο συναφών δεδομένων μπορούμε να εκτιμήσουμε την εποχιακή συμπεριφορά μικρότερων χρονοσειρών εξ αυτών.

Βέβαια, η μελέτη έγινε σε ένα συγκεκριμένο σύνολο χρονοσειρών ενεργειακών χρονοσειρών. Το πρώτο βήμα, λοιπόν, είναι να εξεταστούν και άλλα σύνολα δεδομένων. Ιδιαίτερο ενδιαφέρον, δε, θα έχει η μελέτη οικονομικών δεδομένων και η εφαρμογή στη διαχείριση αποθήκης.

Ένας άλλος παράγοντας που αξίζει να εξεταστεί είναι εναλλακτικές μέθοδοι αποσύνθεσης της χρονοσειράς. Αντί της κλασικής πολλαπλασιαστικής μεθόδου αποσύνθεσης, μπορεί να γίνει ανάλυση με το προσθετικό μοντέλο ή το μοντέλο \en{STL}, που είδαμε επίσης σε προηγούμενο κεφάλαιο. Επίσης, μπορεί να χρησιμοποιηθεί διαφορετική μέθοδος συρρίκνωσης συντελεστών για τους δείκτες εποχιακότητας, ενώ θα μπορούσε να εφαρμοστεί συρρίκνωση και στους δείκτες ψευδο-εποχιακότητας.

Όσον αναφορά την ανάλυση συστάδων, εκτός των \en{DBSCAN} και \en{k-means} θα μπορούσαν να χρησιμοποιηθούν και άλλοι αλγόριθμοι συσταδοποίησης. Επιπρόσθετα, στη περίπτωση που το σύνολο δεδομένων περιέχει και άλλες πληροφορίες που να δίνουν μεγαλύτερη διάσταση σε αυτά, όπως λόγου χάρη τοποθεσία ή είδος πελάτη, θα μπορούσε να χρησιμοποιηθεί αυτή σαν παράμετρος στην ανάλυση συστάδων.

Μεγάλο ενδιαφέρον θα παρουσίαζε επίσης, η σύγκριση της προτεινόμενης μεθόδου με άλλες προσεγγίσεις προέκτασης χρονοσειρών. Μπορεί να γίνει σύγκριση με μεθόδους πρόβλεψης με χρήση νευρωνικών δικτύων, μοντέλων \en{ARIMA} και με μπεϋζιανές τεχνικές προβλέψεων. 
