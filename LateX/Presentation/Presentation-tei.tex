%%%%%%%%%%%%%%%%%%%%%%%%%%%%%%%%%%%%%%%%%%%%%%%%%%
%%%%%%%%%%%%%%%%%%%%%%%%%%%%%%%%%%%%%%%%%%%%%%%%%%
%%
%% Based one the "beamer-greek-two" template provided 
%% by the Laboratory of Computational Mathematics, 
%% Mathematical Software and Digital Typography, 
%% Department of Mathematics, University of the Aegean
%% (http://myria.math.aegean.gr/labs/dt/)
%%
%% Adapted by John Liaperdos, October-November 2014
%% (ioannis.liaperdos@gmail.com)
%%
%% Last update: 26/11/2014
%%
%%%%%%%%%%%%%%%%%%%%%%%%%%%%%%%%%%%%%%%%%%%%%%%%%%
%%%%%%%%%%%%%%%%%%%%%%%%%%%%%%%%%%%%%%%%%%%%%%%%%%
%%
\PassOptionsToPackage{unicode}{hyperref}
\PassOptionsToPackage{naturalnames}{hyperref}
\documentclass[greek]{beamer} 
\usepackage{babel}
\usepackage[utf8]{inputenc}


%%% FONT SELECTION %%%%%%%%%%%%%%%%%
%%% we choose a sans font %%%%%%%%%%
\usepackage{kmath,kerkis} 
%\usepackage[default]{gfsneohellenic} 
%%%%%%%%%%%%%%%%%%%%%%%%%%%%%%%%%%%%

\usepackage{color}
\usepackage{amsmath}
\usepackage{amssymb}

\usepackage{epstopdf}
\usepackage{graphicx}
\graphicspath{{./images/}}

%%
% load TEI-Pel - specific layout
\usepackage{TeiPel_Beamer_Layout}
\setTeipelLayout{draft,newlogo}% options: "draft", "newlogo"

%%%%%%%%%%%%%%%%%%%%%%%%%%%%%%%%%%%%%%%%%%%%%%%%%%%%%%%%%%%%
% Thesis Info %%%%%%%%%%%%%%%%%%%%%%%%%%%%%%%%%%%%%%%%%%%%%%
%%%%%%%%%%%%%%%%%%%%%%%%%%%%%%%%%%%%%%%%%%%%%%%%%%%%%%%%%%%%
	% title
		\title[Πρότυπο Σύστημα Ομότιμων Κόμβων Βασισμένο σε Σχήματα \textlatin{RDF}]{Πρότυπο Σύστημα Ομότιμων Κόμβων\\ Βασισμένο σε Σχήματα \textlatin{RDF}}	
	% author 
    % (In the mandatory argument "{}", separate multiple
    % authors with "\and" - use "\\" for better author name formatting
    % in the title page. Names in latin should be formatted using the
    % \textlatin{name} macro. In the optional argument "[]" include all
	% author names, with no "\and" or text formatting macros.)
	% Example: 
    %\author[Κ. Δημητρίου Albert Einstein]{Κωνσταντίνος Δημητρίου \and \textlatin{Albert Einstein}}
		\author[Κ. Δημητρίου]{Κωνσταντίνος Δημητρίου}
	% supervisor	
		\supervisor{Επιβλέπων}{Γιάννης Λιαπέρδος}{Καθηγητής Εφαρμογών}
	% date
		\presentationDate{22 Οκτωβρίου 2014}
%%%%%%%%%%%%%%%%

\begin{document}

% typeset front slides
	\typesetFrontSlides

%%%%%%%%%%%%%%%%
% Your Slides Start here:

%%%%
\section{Βασικό κίνητρο}

%%
\subsection[Βασικό Πρόβλημα]{Το βασικό πρόβλημα που μελετήσαμε}

\begin{frame}{Τίτλος διαφάνειας \textlatin{\#}1}
	\framesubtitle{Υπότιτλος διαφάνειας \textlatin{\#}1}
	\begin{itemize}
		\item Χρησιμοποιείτε συχνά το περιβάλλον \textlatin{itemize}.
		\pause
		\item Χρησιμοποιείτε σύντομες προτάσεις και φράσεις.
		\pause
		\item Σε αυτή την παρουσίαση χρησιμοποιούμε την εντολή \textlatin{\textbackslash{}pause}.
	\end{itemize}
\end{frame}

\begin{frame}{Τίτλος διαφάνειας \textlatin{\#}2}
	\begin{itemize}
		\item <1->Μπορείτε να περιγράψετε με ποια σειρά θα εμφανίζονται τα αντικείμενα.
		\item <3->Όπως εδώ.
		\item <2->Αυτό είναι το δεύτερο.
	\end{itemize}
\end{frame}

\begin{frame}{Τίτλος διαφάνειας \textlatin{\#}3}
	\begin{block}
		<1->{}
		\begin{itemize}
			\item Ομάδα χωρίς τίτλο.
			\item Φαίνεται σε όλες τις διαφάνειες αυτής της σελίδας.
		\end{itemize}
	\end{block}
	\begin{exampleblock}
		<2->{Παράδειγμα τίτλου ομάδας}
		\begin{itemize}
			\item $e^{i\pi}=-1$.
			\item $e^{i\pi/2}=i$.
		\end{itemize}
	\end{exampleblock}
\end{frame}

%%
\subsection{Προηγούμενες εργασίες}

\begin{frame}{Τίτλος διαφάνειας \textlatin{\#}4}
	\begin{example}
		<1->Πρώτο παράδειγμα. 
	\end{example}
	\begin{example}
		<2->Δεύτερο παράδειγμα.
	\end{example}
\end{frame}

\begin{frame}{Τίτλος διαφάνειας \textlatin{\#}5}
	\begin{center}
		Παράδειγμα Πίνακα \\[12pt]
		\begin{tabular}{c||c|c|c|}
			& \textbf{στήλη 1} & \textbf{στήλη  2} & \textbf{στήλη 3} \\
			\hline
			\hline
			\textbf{γραμμή 1} & 11 & 12 & 13 \\
			\hline
			\textbf{γραμμή 2} & 21 & 22 & 23 \\
		\end{tabular}
    \end{center}
\end{frame}

\begin{frame}{Τίτλος διαφάνειας \textlatin{\#}6}
	\begin{center}
		Παράδειγμα Σχήματος \\[12pt]
		\includegraphics[width=0.35\textwidth,keepaspectratio]{LampFlowchart.png}
		\\
		\footnotesize(πηγή: \textlatin{Wikipedia})
    \end{center}
\end{frame}

\begin{frame}{Τίτλος διαφάνειας \textlatin{\#}7}
	\centering
	Παράδειγμα Μαθηματικών Σχέσεων \\[12pt]
	\begin{equation}
        	B'=-\nabla \times E
	\end{equation}
	\begin{equation*}
        	E'=\nabla \times B - 4\pi j
	\end{equation*}
\end{frame}

%%%%
\section{Τα αποτελέσματά μας/η συμβολή μας}

%%
\subsection{Κύρια αποτελέσματα}

\begin{frame}{Περίληψη}
   	\begin{alertblock}{Προσοχή}
   		\textlatin{This is an important alert}
   	\end{alertblock}
\end{frame}

%%
\subsection{Τίτλος υποενότητας}

\begin{frame}{Περίληψη}
	\begin{itemize}
		\item Το \textcolor{red}{πρώτο κύριο μήνυμα} της ομιλίας σας σε μια ή δυο γραμμές.
		\item Το \textcolor{red}{δεύτερο κύριο μήνυμα} της ομιλίας σας σε μια ή δυο γραμμές.
		\item Ίσως ένα \textcolor{red}{τρίτο μήνυμα}, αλλά μέχρι εδώ; όχι άλλο.
	\end{itemize}
	\vskip0pt plus.5fill
	\begin{itemize}
		\item Επισκόπηση.
	\end{itemize}
	\begin{itemize}
		\item Τι δεν έχουμε κάνει ακόμα.
		\item Επιπλέον θέματα.
	\end{itemize}
\end{frame}

\begin{frame}{Βιβλιογραφία}
	\begin{thebibliography}{2}
		\beamertemplatebookbibitems
		\bibitem{Author1990}Α.\ Συγγραφέας. \newblock\textlatin{\emph{Handbook of Everything}}.\newblock
\textlatin{Some Press, \oldstylenums{1990}}.

		\beamertemplatearticlebibitems
		\bibitem{Someone2002}Β.\ Συγγραφέας.\newblock \textlatin{On this and that}\emph{.}
\newblock\textlatin{\emph{Journal on This and That}}. 
\oldstylenums{2}(\oldstylenums{1}):\oldstylenums{50}--\oldstylenums{100}, 
\oldstylenums{2000}.
	\end{thebibliography}
\end{frame}

%%%
\end{document}