\chapter{Τεχνικές λεπτομέρειες}
\label{chap7}

Εδώ λέμε ότι θα ακολουθήσουν τεχνικές λεπτομέρειες της διπλωματικής.

\section{Λεπτομέρειες υλοποίησης}

Εδώ περιγράφουμε λεπτομερώς θέματα της διπλωματικής που έχουν τεχνικό ενδιαφέρον. Προσδιορίστε επομένως τα θέματα αυτά, βάλτε μια ενότητα για κάθε ένα και περιγράψτε τα αναλυτικά. Η περιγραφή μπορεί να γίνει βάζοντας κομμάτια κώδικα ή ψευδοκώδικα, και περιγράφοντάς τα με λόγια. Μην ξεχνάτε να δίνετε πάντα παραδείγματα για το πώς τρέχει ένα κομμάτι κώδικα π.χ. για έναν αλγόριθμο.

\subsection{<Τίτλος θέματος 1>}
Γράψτε το κείμενό σας εδώ ...

\subsection{<Τίτλος θέματος 1>}
Γράψτε το κείμενό σας εδώ ...

\section{Πλατφόρμες και προγραμματιστικά εργαλεία}

Εδώ περιγράφονται τα χαρακτηριστικά της συγκεκριμένης υλοποίησης, όπως η πλατφόρμα ανάπτυξης και εκτέλεσης, τα προγραμματιστικά εργαλεία, οι απαιτήσεις της εφαρμογής σε hardware, κ.λ.π. Επίσης, περιγράφεται λεπτομερώς η διαδικασία εγκατάστασης της διπλωματικής σε υπολογιστή. Προσέξτε να δίνονται όλες οι λεπτομέρειες, το απαραίτητο λογισμικό και οι αναγκαίες ρυθμίσεις.
