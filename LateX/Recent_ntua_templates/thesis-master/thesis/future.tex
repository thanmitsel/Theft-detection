\chapter{Conclusion}\label{ch:future}

\section{Concluding remarks}

This thesis is a record of over a year long process that has ultimately lead to 
the creation of two Archipelago peers, cached and synapsed. It documents the 
evaluation of third-party solutions, the reasoning behind our choice to create 
a custom cache mechanism, as well as the design decisions and technical issues 
that we have encountered.

Looking back at what we have created, we can safely state that this thesis has 
successfully covered two basic needs of Archipelago, caching and networking.  
The extend at which these needs have been satisfied can be considered as more 
than adequate, since cached has been successfully tested under an actual VM, 
while synapsed has managed to bridge two peers over network with minimum 
latency.

Most importantly however, cached provides a concrete solution to the task that 
was described in the very first chapter of this thesis; substantially improve 
the current performance of Archipelago.  To be more specific, the results of 
synthetic benchmarks show that cached can improve up to 200x the current 
performance.  Moreover, when tested with an actual VM, the performance speedup 
was able to reach 400\%, even for small cache sizes.

Finally, the future looks very bright for our caching mechanism. The impending 
integration of cached in the demo environment\footnote{demo.synnefo.org}
of Synnefo, will help it gain the necessary exposure that will provide us with 
the needed feedback for more improvements and the test-bed for new ideas.

\section{Future work}

The future work for cached is happening as of writing this very chapter. We are 
currently working to add the following:

\begin{itemize}
	\item Full CoW support.
	\item Support for different namespaces (mappings, volumes, objects) so 
		that cached can be used as a generic caching peer for all 
		Archipelago needs.
	\item Support for different policies and limits per volume.
\end{itemize}

Moreover, the long term goal for cached is to be able to be used with synapsed 
effectively, in order to create a fast distributed and replicated cache.
