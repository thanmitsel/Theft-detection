\begin{acknowledgements}

Θα ήθελα να ευχαριστήσω τον επιβλέποντα καθηγητή κ. Νικόλαο Χατζηαργυρίου για την ευκαιρία που μου έδωσε να εκπονήσω τη παρούσα διπλωματική και την υποστήριξή του σε όλη την πορεία της.

Επίσης, θα ήθελα να ευχαριστήσω  τους καθηγητές κ. Σταύρο Παπαθανασίου και κ. Παύλο Γεωργιλάκη για την τιμή που μου έκαναν να συμμετάσχουν στην επιτροπή εξέτασης της διπλωματικής.

Eυχαριστώ ιδιαίτερα τον υποψήφιο διδάκτορα Γιώργη Μεσσήνη για την καθοδήγηση, στήριξη και καθοριστική βοήθεια που μου παρείχε.

Τέλος, θα ήθελα να ευχαριστήσω την οικογένειά μου και τους φίλους μου που παρέχουν πάντοτε ένα χέρι βοήθειας σε ό,τι χρειαστώ.

\end{acknowledgements}


\begin{abstract}
Οι εταιρίες παροχής ηλεκτρισμού αντιμετωπίζουν το ολοένα και αυξανόμενο πρόβλημα της διείσδυσης μη τεχνικών απωλειών στις καταναλώσεις των πελατών τους. Το γεγονός αυτό πλήττει σημαντικά τις εταιρίες, μειώνοντας το εισόδημά τους και θέτει σε κίνδυνο τους ανειδίκευτους καταναλωτές που επεμβαίνουν στις υποδομές του παρόχου. Η προσέγγιση αυτού του προβλήματος έγινε με προσομοίωση ρευματοκλοπών σε ετήσιες χρονοσειρές καταναλωτών και δοκιμάστηκαν πληθώρα αλγορίθμων επιβλεπόμενης, μη επιβλεπόμενης και ημι-επιβλεπόμενης μηχανικής μάθησης για την ανίχνευση των καταναλωτών με διείσδυση μη τεχνικών απωλειών. Τα αποτελέσματα αναδεικνύουν τις δυνατότητες των συστημάτων μη επιβλεπόμενης και ημι-επιβλεπόμενης μάθησης σε σχέση με τη δεδομένη επιτυχία των αλγορίθμων επιβλεπόμενης μάθησης. Τα συστήματα που δημιουργήθηκαν έχουν ικανοποιητική απόδοση που δεν αποκλίνει σημαντικά από τους αλγορίθμους αναφοράς της επιβλεπόμενης μάθησης. Καθίσταται λοιπόν σαφές πως η ανίχνευση μη τεχνικών απωλειών είναι εφικτή με συστήματα μηχανικής μάθησης.

\begin{keywords}
  Μη τεχνικές απώλειες, Ρευματοκλοπές, Χρονοσειρές, Μηχανική μάθηση, Επιβλεπόμενοι αλγόριθμοι, Μη επιβλεπόμενοι αλγόριθμοι, Ημι-επιβλεπόμενοι αλγόριθμοι.
 
\end{keywords}

\end{abstract}



\begin{abstracteng}
\tl{Power companies face the problem of increasing intrusion of non-technical losses on consumptions of their clients. That fact hurts significantly power companies by reducing their economical growth and sets on danger unskilled consumers who intervene with the power infastracture. This problem was approached by simulating frauds on yearly timeseries and by testing  many different algorithms of supervised, unsupervised and semi-supervised  machine learning in order to detect consumers with non-technical loss intrusion. The results show the potencial of the unsupervised and semi-supervised learning in relation with the given success of supervised algorithms. The created systems have satisfactory performance which does not diverge significantly from the reference algorithms of supervised learning. Concluding the detection of non-technical losses is achievable with machine learning systems.}
\begin{keywordseng}
  \tl{Non-technical losses, power fraud, Timeseries, Machine learning, Supervised algorithms, Unsupervised algorithms, Semi-supervised algorithms.}
\end{keywordseng}

\end{abstracteng}
