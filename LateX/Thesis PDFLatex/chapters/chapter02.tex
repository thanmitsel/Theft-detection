Η ηλεκτρική ενέργεια είναι ζωτικής σημασίας για την καθημερινότητά μας αλλά και ο ακρογωνιαίος λίθος της βιομηχανίας. Για αυτό τον λόγο, η έννοια των μελλοντικών δικτύων (έξυπνα δίκτυα) στοχεύει στην αύξηση της αξιοπιστίας, της ποιότητας και της ασφάλειας της μελλοντικής παροχής ενέργειας. Για να συμβεί αυτό, απαιτούνται περαιτέρω πληροφορίες για τη λειτουργία και την κατάσταση των δικτύων διανομής. Μια από τις σημαντικότερες προκλήσεις στα μελλοντικά δίκτυα διανομής είναι η αυξανόμενη διείσδυση διεσπαρμένης παραγωγής (\en{Distributed Generation}) και η μετάβαση από την έννοια της παραδοσιακής παραγωγής ενέργειας με κυρίαρχους μεγάλους σταθμούς παραγωγής και ροές ενέργειας μονής κατεύθυνσης σε κατανεμημένα μοντέλα. Οι πληροφορίες λειτουργίας θα είναι καίριας σημασίας για τη λειτουργικότητα των μελλοντικών δικτύων διανομής και για τους διαχειριστές του δικτύου (\en{Distribution Network Operators}). Μια από τις πηγές πληροφορίας θα είναι οι προηγμένες υποδομές μέτρησης. Εκτός των άλλων, οι έξυπνοι μετρητές πρέπει να διευρύνουν τους γνωστικούς ορίζοντες των καταναλωτών για την ηλεκτρική ενέργεια. Η έννοια αυτή θα παράξει ακόμη περισσότερη πληροφορία στους διαχειριστές δικτύου. Τα δεδομένα των έξυπνων μετρητών δίνουν τη δυνατότητα στον διαχειριστή του δικτύου να αναλύσει ροές ενέργειας και να εντοπίσει πιθανή κλοπή ρεύματος \cite{netherlands}. \\
\section{Έξυπνοι μετρητές}
Η διανομή είναι ένας τομέας στον οποίο η εξέλιξη είναι σταδιακή, τουλάχιστον όσον αφορά τα στοιχεία του δικτύου. Παρ' όλα αυτά, ο κλάδος των τηλεπικοινωνιών και της εξαγωγής και επεξεργασίας δεδομένων εξελίσσεται ραγδαία τα τελευταία χρόνια. Οι απομακρυσμένες μετρήσεις και η συνεχής καταγραφή και παρακολούθηση της κατανάλωσης θεωρούνται ως προηγμένη υποδομή μέτρησης (\en{Advanced Metering Infrastructure}). Η δραστική μείωση στις τιμές των μετρητών και στον εξοπλισμό τηλεπικοινωνιών καθιστά την απόκτησή τους οικονομικά βιώσιμη, ξεκινώντας με μεγάλους καταναλωτές και σταδιακά εγκαθιστώντας τους στους μέσους και μικρούς. Η αποτελεσματικότητα των εργαλείων στην αναγνώριση και αποθάρρυνση της κλοπής και άλλων τρόπων παράκαμψης μετρητών είναι τεράστια, όπως φαίνεται να συμβαίνει σε αναπτυσσόμενες χώρες (συμπεριλαμβανομένου της Δομινικανικής Δημοκρατίας, της Ονδούρας και της Βραζιλίας) \cite{reduceloss}.\par
Η ευρεία εφαρμογή \en{AMI} μπορεί να συμβάλλει σημαντικά στη συνεχή ανάπτυξη και την αποτελεσματική λειτουργία των ενεργειακών δομών. Τα \en{AMI} παρέχουν ισχυρά εργαλεία, ικανά να μειώσουν τις συνολικές απώλειες και να αυξήσουν τα έσοδα των εταιριών.
\subsection{Θετικά αντίκτυπα εφαρμογής \en{AMI}}
Η εφαρμογή των \en{AMI} θα έχει τα ακόλουθα θετικά αποτελέσματα:
\begin{enumerate}
\item Αίσθηση παρακολούθησης στους χρήστες. Οι καταναλωτές αντιλαμβάνονται πως ο πάροχος ενέργειας μπορεί να παρακολουθεί την κατανάλωση. Αυτό επιτρέπει στην εταιρία τη γρήγορη ανίχνευση οποιασδήποτε ανωμαλίας στην κατανάλωση, λόγω αλλοίωσης του μετρητή ή παράκαμψής του και της δίνει τη δυνατότητα να κάνει διορθωτικές κινήσεις. Το αποτέλεσμα είναι η πειθάρχηση των καταναλωτών.
\item Ενίσχυση της εταιρικής διακυβέρνησης και καταπολέμησης της διαφθοράς. Τα παραδείγματα κλοπής μεγάλων καταναλωτών συνήθως συμπεριλαμβάνουν συνεννόηση μεταξύ αυτών και των ελεγκτών των μετρητών. Η διαφθορά είναι επίσης πιθανό να παρατηρηθεί και στις ενέργειες που συσχετίζονται με την αποσύνδεση του μετρητή, λόγω απλήρωτων λογαριασμών. Η εγκατάσταση των \en{AMI} καθιστά τις πληροφορίες των έξυπνων μετρητών διαθέσιμες στους καταναλωτές και τους διαχειριστές, επιβάλλοντας διαφάνεια.
\item Υλοποίηση προπληρωμένων καταναλώσεων. Η προ-πλήρωση των λογαριασμών είναι γενικώς πολύ θετικό για τους καταναλωτές μικρού εισοδήματος. Τα \en{AMI} δίνουν τη δυνατότητα αντιγραφής του επιχειρηματικού μοντέλου των εταιριών κινητής τηλεφωνίας και στον τομέα της ενέργειας.
\item Ελαχιστοποίηση απωλειών σε δυσπρόσιτες και απομακρυσμένες περιοχές. Τα \en{AMI} διαδραματίζουν καθοριστικό ρόλο στην προσέγγιση της διανομής μέσης τάσης (\en{Medium-Voltage Distribution}), που χρησιμοποιείται για την κατασκευή και λειτουργία ηλεκτρικών δικτύων, για την παροχή ενέργειας σε περιοχές που η πρόσβαση της εταιρίας είναι περιορισμένη για λόγους ασφαλείας. Στα \en{MVD} δίκτυα κάθε σύνδεση καταναλωτή ξεκινάει απευθείας από το μετασχηματιστή μέσης σε χαμηλή τάση, με το δίκτυο χαμηλής τάσης να εκλείπει.
\item Διαχείριση από την πλευρά της ζήτησης για μεγιστοποίηση της αποτελεσματικότητας στην παροχή και κατανάλωση ενέργειας. Τα \en{AMI} μέσα σε έξυπνο δίκτυο επιτρέπουν την βελτιστοποίηση της κατανάλωσης ενέργειας, ενημερώνοντας τους χρήστες έγκαιρα για τις τιμές, την αρχή και το τέλος των περιόδων αιχμής της κατανάλωσης, το άθροισμα της κατανάλωσης, συναγερμούς κτλ \cite{reduceloss}.
\end{enumerate}
\section{Μηχανική μάθηση}
Υπάρχουν διαφορετικοί τρόποι που ένας αλγόριθμος μπορεί να μοντελοποιήσει ένα πρόβλημα βασισμένος στα δεδομένα εισόδου. Είναι δημοφιλές στα βιβλία μηχανικής μάθησης και τεχνητής νοημοσύνης να εξετάζεται ο τρόπος μάθησης που ένας αλγόριθμος μπορεί να υιοθετήσει. Υπάρχουν μόνο μερικοί βασικοί τρόποι εκμάθησης ή μοντέλα εκμάθησης που ένας αλγόριθμος μπορεί να χρησιμοποιήσει. Θα αναφερθεί κάθε μοντέλο εκμάθησης με λίγα παραδείγματα από αλγορίθμους και τύπους προβλημάτων που ταιριάζει στο καθένα. Αυτή η ταξινόμηση ή ο τρόπος οργάνωσης των αλγορίθμων είναι χρήσιμος, καθώς αναγκάζει τον χρήστη να σκεφτεί τον ρόλο των δεδομένων εισόδου και το μοντέλο επεξεργασίας και να επιλέξει τον κατάλληλο αλγόριθμο για το πρόβλημα, με στόχο τα βέλτιστα αποτελέσματα. Παρακάτω αναλύονται οι τρεις διαφορετικές κατηγορίες αλγορίθμων μηχανικής μάθησης με βάση τον τρόπο εκμάθησης.
\subsection{Επιβλεπόμενη μάθηση}
Τα δεδομένα εισόδου καλούνται δεδομένα εκπαίδευσης και είναι γνωστά τα αποτελέσματά τους (κλάσεις). Τέτοια προβλήματα ορίζονται, όταν ένα παράδειγμα ταξινομείται σε αρνητική κλάση ή θετική κλάση ή αναζητείται αριθμητικό αποτέλεσμα σε μια ορισμένη χρονική περίοδο (παλινδρόμηση), ενώ έχει προηγηθεί εκπαίδευση μοντέλου με ζευγάρια δεδομένων αποτελεσμάτων. Ένα μοντέλο χτίζεται στη φάση της εκπαίδευσης κατά την οποία απαιτείται να κάνει προβλέψεις και να τις διορθώσει όταν είναι λάθος. Η διαδικασία της εκπαίδευσης συνεχίζει μέχρι το μοντέλο να επιτύχει το επίπεδο ευστοχίας στα δεδομένα εκπαίδευσης. Τέτοια προβλήματα είναι τα προβλήματα ταξινόμησης και παλινδρόμησης. Κάποιοι από τους δημοφιλείς αλγορίθμους είναι η λογιστική παλινδρόμηση και τα νευρωνικά δίκτυα.
\subsection{Μη επιβλεπόμενη μάθηση}
Τα δεδομένα εισόδου σε αυτούς τους αλγορίθμους δεν έχουν έχουν γνωστά αποτελέσματα. Ένα μοντέλο προετοιμάζεται, εξάγοντας χαρακτηριστικά από τα δεδομένα εισόδου. Εν συνεχεία, εφαρμόζονται γενικοί κανόνες που βασίζονται στα υπάρχοντα χαρακτηριστικά. Αυτό συνήθως συμβαίνει μέσω κάποιας μαθηματικής διαδικασίας που μειώνει συστηματικά την επαναληψιμότητα του αλγορίθμου ή με οργάνωση των δεδομένων βάσει ομοιότητας. Τέτοιου είδους προβλήματα είναι η συσταδοποίηση, η μείωση διάστασης και η εκπαίδευση μέσω κανόνων συσχέτισης. Αντιπροσωπευτικοί αλγόριθμοι είναι το \en{K-Means} και το \en{Principal Component Analysis (PCA)}.
\subsection{Ημι-επιβλεπόμενη μάθηση}
Τα δεδομένα εισόδου είναι μια μείξη γνωστών και άγνωστων δυαδικών χαρακτηριστικών. Υπάρχει μια επιθυμητή πρόβλεψη τους προβλήματος, αλλά το μοντέλο πρέπει να μάθει τη δομή για να οργανώσει τα δεδομένα αλλά και να κάνει τις τελικές προβλέψεις. Τέτοια προβλήματα είναι η ταξινόμηση και η παλινδρόμηση. Οι αλγόριθμοι που χρησιμοποιούνται είναι επέκταση άλλων ευέλικτων μεθόδων που κάνουν υποθέσεις για το μοντέλο χωρίς τα δυαδικά χαρακτηριστικά \cite{learningstyle}.
\section{Μετρικές μηχανικής μάθησης}
Για να γίνει αξιολόγηση της ταξινόμησης, χρειάζεται να ληφθούν υπόψη κάποια κριτήρια και μετρικές. Ο ρυθμός ευστοχίας ή η μέση τιμή του λάθους αδυνατούν να μας περιγράψουν σαφώς τον ταξινομητή, οπότε εισάγεται η έννοια του \en{confusion matrix}. Σύμφωνα με τον πίνακα μετράμε τις εξής τιμές:\\

\begin{figure}[ht!]
\centering
\noindent
\renewcommand\arraystretch{1.5}
\setlength\tabcolsep{0pt}
\begin{tabular}{c >{\bfseries}r @{\hspace{0.7em}}c @{\hspace{0.4em}}c @{\hspace{0.7em}}l}
  \multirow{10}{*}{\parbox{1.1cm}{\bfseries\raggedleft Πραγματική\\ Τιμή}} & 
    & \multicolumn{2}{c}{\bfseries Πρόβλεψη} & \\
  & & \bfseries \en{p} & \bfseries \en{n} & \bfseries Συνολικά \\
  & \en{p}$'$ & \MyBox{\en{True}}{\en{Positive}} & \MyBox{\en{False}}{\en{Negative}} & \en{P}$'$ \\[2.4em]
  & \en{n}$'$ & \MyBox{\en{False}}{\en{Positive}} & \MyBox{\en{True}}{\en{Negative}} & \en{N}$'$ \\
  & Συνολικά & \en{P} & \en{N} &
\end{tabular}

\caption{\en{Confusion Matrix}}
\label{fig:confusion matrix}
\end{figure}

\begin{center}
$TP$=πλήθος των σωστών προβλέψεων στο θετικό αποτέλεσμα\\
$TN$=πλήθος των σωστών προβλέψεων στο αρνητικό αποτέλεσμα\\
$FN$=πλήθος των λανθασμένων προβλέψεων στο θετικό αποτέλεσμα (αρνητική πρόβλεψη)\\
$FP$=πλήθος των λανθασμένων προβλέψεων στο αρνητικό αποτέλεσμα (θετική πρόβλεψη)\\
\end{center}
\par Με τις παραπάνω τιμές είναι δυνατό να δομηθούν τα κριτήρια ευστοχίας του συστήματος. Οι τέσσερις βασικοί άξονες της μέτρησης είναι το ποσοστό αναγνώρισης \en{DR (Detection Rate)}, το ποσοστό λάθος συναγερμού \en{FPR(False Positive Rate)}, το ποσοστό της ευστοχίας \en{(Accuracy)} και το \en{F1 score}, που είναι ένας συνδυασμός μετρικών, για να αποκτηθεί μια γενικότερη εικόνα της ακρίβειας του συστήματος.\par
Με τα πλήθη προβλέψεων να έχουν οριστεί έχουμε τις θεμελιώδεις μετρικές:
\begin{center}
$DR=\frac{TP}{TP+FN}$, $FPR=\frac{FP}{FP+TN}$\\ 
$Accuracy=\frac{TP+TN}{TP+FP+FN+TN}$, $F1=2\frac{precision \cdotp recall}{precision + recall}$\\
δεδομένου $Precision=\frac{TP}{TP + FP}$, $Recall=DR=\frac{TP}{TP + FN}$.
\end{center}
\par Ακόμη θα χρησιμοποιηθεί το ποσοστό αναγνώρισης του \en{Bayes} και η αντίστοιχή του άρνηση για να μας δώσουν μια πιθανοτική σκοπιά για τη διάκριση απάτης από τη φυσιολογική κατανάλωση. Η $P(I)$ είναι η πιθανότητα να υπάρχει απάτη στα δεδομένα και αυτό σε πραγματικές συνθήκες δεν είναι εύκολο να υπολογιστεί με ακρίβεια. Το ενδεχόμενο $A$ αντιστοιχεί στον συναγερμό που ενεργοποιείται στην αναγνώριση απάτης. Μπορεί στα συγκεκριμένα δεδομένα να οριστεί ως η πιθανότητα μια τυπική μέρα να βρεθεί απάτη στις μετρήσεις.\par
Τεχνικά οι δύο πιθανότητες ορίζονται ως εξής:
\begin{itemize}
\item $P(I|A)-$ ένας συναγερμός πραγματικά ενδεικνύει απάτη
\item $P(\neg{I}|\neg{A})-$ η μη ενεργοποίση του συναγερμού υποδηλώνει απουσία απάτης
\end{itemize}
Αυτό που έχει σημασία είναι και οι δύο πιθανότητες να παραμείνουν όσο το δυνατόν μεγαλύτερες \cite{propab}.\par
Μπορούμε να αντιστοιχίσουμε τα βασικά κριτήρια με τις πιθανότητες στο ποσοστό αναγνώρισης του \en{Bayes}.
\begin{center}
$P(I|A)=\frac{P(I)P(A|I)}{P(I)P(A|I)+P(\neg{I}) \cdotp P(A|\neg{I})}$, 
$P(\neg{I}|\neg{A})=\frac{P(\neg{I}) \cdotp P(\neg{A}|\neg{I})}{P(\neg{I}) \cdotp P(\neg{A}|\neg{I})+P(I) \cdotp P(\neg{A}|I)}$\\
για $P(A|I)=DR$, $P(A|\neg{I})=FPR$, $P(\neg{A}|I)=1-P(A|I)$, $P(\neg{A}|\neg{I})=1-P(A|\neg{I})$\\
έχουμε
$BDR=\frac{P(I)DR}{P(I) \cdotp DR+P(\neg{I}) \cdotp FPR}$\\
\end{center}
