Στο παρόν κεφάλαιο γίνεται μια εξερεύνηση στους αλγορίθμους επιβλεπόμενης μάθησης. Αυτό επιτεύχθηκε με τη χρήση γραμμικών και μη-γραμμικών ταξινομητών διερευνώντας διαφορετικά δεδομένα εισόδου για κάθε περίπτωση. Η βιβλιοθήκη που χρησιμοποιήθηκε για τη γραμμική ταξινόμηση ονομάζεται \en{LIBLINEAR} και χαρακτηρίζεται με εξαιρετικές επιδόσεις σε προβλήματα με μεγάλα σετ δεδομένων. Αντίστοιχα για τη μη-γραμμική ταξινόμηση χρησιμοποιήθηκε η βιβλιοθήκη \en{LIBSVM}, η οποία αναγάγει τα δεδομένα εισόδου σε μεγαλύτερο χώρο διαστάσεων.
\section{Θεωρία γραμμικής ταξινόμησης}
Η βιβλιοθήκη \en{LIBLINEAR} υποστηρίζει δύο δημοφιλείς δυαδικά γραμμικούς ταξινομητές: τη λογιστική παλινδρόμηση (\en{Logistic Regression}) και τη γραμμική μηχανή υποστήριξης διανυσμάτων (\en{linear SVM}). Δεδομένου ενός σετ εκπαίδευσης $(\mathbf{x}_i, y_i)$, $i=1,...,l$, όπου $\mathbf{x}_i\in\R^n$ είναι ένα χαρακτηριστικό διάνυσμα και $y_i=\pm1$ είναι οι ετικέτες, ένας γραμμικός ταξινομητής βρίσκει ένα διάνυσμα βαρών $\mathbf{w}\in\R^n$ επιλύοντας το ακόλουθο πρόβλημα:
\begin{center}
$min_{\mathbf{w}}f(\mathbf{w})\equiv\frac{1}{2}\mathbf{w}^T\mathbf{w}+C\sum_{i=1}^{l}\xi(y_i\mathbf{w}^T x_i)$
\end{center}
όπου $\mathbf{w}^T\mathbf{w}/2$ είναι ο όρος ομαλοποίησης, $\xi(y_i\mathbf{w}^T x_i)$ είναι η συνάρτηση κόστους (\en{loss function}) και $C>0$ είναι η παράμετρος ομαλοποίησης. Θεωρούμε τις συναρτήσεις κόστους στη λογιστική παλινδρόμηση (\en{LR}), στο  \en{L1-SVM}, στο \en{L2-SVM}:
\begin{center}
$\xi_{LR}(y\mathbf{w}^T\mathbf{x})=log(1 + exp(-y\mathbf{w}^T\mathbf{x}))$\\
$\xi_{L1}(y\mathbf{w}^T\mathbf{x})=(max(0, 1 - y\mathbf{w}^T\mathbf{x}))$\\
$\xi_{L2}(y\mathbf{w}^T\mathbf{x})=(max(0, 1 - y\mathbf{w}^T\mathbf{x}))^2$
\end{center}
Σε μερικές περιπτώσεις, η συνάρτηση διακρίσεως του ταξινομητή περιλαμβάνει και ένα παράγοντα βάρους, $b$. Η \en{LIBLINEAR} χειρίζεται αυτό τον παράγοντα αυξάνοντας το διάνυσμα $\mathbf{w}$ και κάθε παράδειγμα $\mathbf{x}_i$ με μία επιπλέον διάσταση: $\mathbf{w}^T \leftarrow [\mathbf{w}^T, b]$, $\mathbf{x}_i^T \leftarrow [\mathbf{x}_i^T, B]$, όπου B είναι μια σταθερά που ορίζεται από το χρήστη. Η προσέγγιση για το \en{L1-SVM} και το  \en{L2-SVM} είναι μέσω της μεθόδου \en{coordinate descent}. Για το \en{LR} και το \en{L2-SVM}, η \en{LIBLINEAR} υλοποιεί μια μέθοδο περιοχής εμπιστοσύνης \en{Newton}. Στη φάση των δοκιμών, εκτιμάται ένα μέλος των δεδομένων $\mathbf{x}$ σαν θετικό εάν $\mathbf{w}^T\mathbf{x}>0$, και αρνητικό σε αντίθετη περίπτωση\cite{liblinearguide} \cite{liblinearreport}.\par
Η μηχανή διανυσμάτων υποστήριξης εντάσσεται στο γενικότερο πλαίσιο της βελτιστοποίησης κυρτών συναρτήσεων και έχει νόημα η προσέγγισή της για όλους τους γραμμικούς ταξινομητές. Σε αδρές γραμμές, η διαδικασία εξελίσσεται σε τέσσερα κύρια βήματα:
\begin{itemize}
\item Το πρόβλημα της εύρεσης του βελτίστου υπερεπιπέδου ξεκινά με μια δήλωση του προβλήματος στον πρωτεύοντα χώρο βαρών, ως ένα πρόβλημα βελτιστοποίησης με περιορισμούς.
\item Κατασκευάζεται η συνάρτηση \en{Lagrange} του προβλήματος.
\item Διατυπώνονται οι συνθήκες για τη βελτιστοποίηση της μηχανής.
\item Στήνεται το σκηνικό για την επίλυση του προβλήματος βελτιστοποίησης στο δυικό χώρο των πολλαπλασιαστών \en{Lagrange}.
\end{itemize}
Όπως προαναφέρθηκε, το πρωτεύον πρόβλημα ασχολείται με μια κυρτή συνάρτηση κόστους και γραμμικούς περιορισμούς. Δοθέντος ενός τέτοιου προβλήματος βελτιστοποίησης με περιορισμούς, είναι δυνατό να κατασκευάσουμε ένα άλλο πρόβλημα, το αποκαλούμενο δυικό του πρωτεύοντος. Αυτό το δεύτερο πρόβλημα έχει την ίδια βέλτιστη τιμή με το πρωτεύον πρόβλημα, αλλά με τους πολλαπλασιαστές \en{Lagrange} να παρέχουν τη βέλτιστη λύση\cite{haykin}.
\section{Εξερεύνηση γραμμικών ταξινομητών}
Αρχικά έγινε μια εξερεύνηση των μεθόδων που παρέχει η \en{LIBLINEAR} για την επίλυση του δυαδικού προβλήματος. Λαμβάνοντας υπόψη 2.000 καταναλώσεις πελατών με ωριαίες μετρήσεις, επιλέχθηκε 10\% ποσοστό ρευματοκλοπών για την προσομοίωση. Η βιβλιοθήκη που χρησιμοποιήθηκε περιλαμβάνει 7 διαφορετικούς συνδυασμούς ταξινομητών και συναρτήσεων κόστους για να μπορούν όσο το δυνατόν περισσότερα προβλήματα. Παρόλα αυτά οι μέθοδοι \en{L1} είναι παλαιότερες εκδόσεις των \en{L2} και αναμένεται να έχουν χειρότερα αποτελέσματα στις δοκιμές. Για την σφαιρική αντιμετώπιση του προβλήματος χρησιμοποιήθηκαν όλοι οι ταξινομητές που παρέχονται από τη βιβλιοθήκη σε κάθε τύπο απάτης. Παρακάτω παραθέτονται οι συνδυασμοί ταξινομητών και συναρτήσεων κόστους που δοκιμάστηκαν και τα αποτελέσματα σε κάθε τύπο απάτης.
\begin{enumerate}
\item \en{L2} ομαλοποιημένη λογιστική παλινδρόμηση (πρωτεύον)
\item \en{L2} oμαλοποιημένος ταξινομητής με \en{L2} συνάρτηση κόστους διανυσμάτων υποστήριξης (δυικό)
\item \en{L2} oμαλοποιημένος ταξινομητής με \en{L2} συνάρτηση κόστους διανυσμάτων υποστήριξης (πρωτεύον)
\item \en{L2} oμαλοποιημένη ταξινομητής με \en{L1} συνάρτηση κόστους διανυσμάτων υποστήριξης (δυικό)
\item Ταξινόμηση διανυσμάτων υποστήριξης από \en{Crammer} και \en{Singer}
\item \en{L1} oμαλοποιημένος ταξινομητής με \en{L2} συνάρτηση κόστους διανυσμάτων υποστήριξης
\item \en{L1} ομαλοποιημένη λογιστική παλινδρόμηση
\item \en{L2} ομαλοποιημένη λογιστική παλινδρόμηση (δυικό)
\end{enumerate}
Παρατηρώντας τους πίνακες αποτελεσμάτων εύκολα αποδεικνύεται η αρχική υπόθεση πως οι ταξινομητές και συναρτήσεις κόστους \en{L2} έχουν καλύτερη συμπεριφορά ως προς την αντιμετώπιση του προβλήματος αναγνώρισης χρονοσειρών.  Πιο συγκεκριμένα για την τελική επιλογή του συνδυασμού μεθόδων επιλέχθηκαν δύο μετρικές για να καθορίσουν την επιλογή του καλύτερου πακέτου. Λήφθηκε υπόψη η μεταβολή της ευστοχίας (accuracy) και παράχθηκε μέσος όρος για όλους τους τύπους. Παράλληλα, υπολογίστηκε μέσος όρος των δοκιμών με γνώμονα το καλύτερο \en{F1 score}, καθώς είναι μια αρκετά ζυγισμένη μετρική για τα προβλήματα ταξινόμησης. Βάση λοιπόν του Πίνακα \ref{tab:meanlinearmetrics} την καλύτερη επίδοση έχει το πρωτεύον πρόβλημα που αποτελείται από \en{L2} oμαλοποιημένο ταξινομητή με \en{L2} συνάρτηση κόστους διανυσμάτων υποστήριξης, καθώς όπως μπορεί και να φανεί στον Πίνακα \ref{tab:accuracytypes} του Παραρτήματος η μηχανή διανυσμάτων υποστήριξης \en{Crammer} και \en{Singer} έχει καλύτερη επίδοση στους τύπους 2, 3 και στον μικτό. Αλλά, στην παρούσα φάση θα ασχοληθούμε με την απάτη τύπου 1.

\begin{table}[ht!]
\centering
\begin{tabular}{ |c||c|c|c|c|c|c|c|c|  }
 \hline
 Συνδυασμός & 1 & 2 & 3 & 4 & 5 & 6 & 7& 8 \\
 \hline
 \en{F1 score} & 23.92 & 31.99 & 30.19 &  28.67& 32.66 & 29.28 & 20.43 &24.04\\
 \hline
\en{Accuracy} & 91.36 & 90.41 & 90.46 &  90.56 & 90.15 & 90.37 & 91.43 & 91.35\\
\hline
  Μέσος όρος & 57.64 & 61.2 & 60.33 & 59.61 & 61.4 & 59,83 & 55.93 & 57.7\\
\hline
\end{tabular}
\caption{Μέσος όρος \en{Accuracy} των δοκιμών}
\label{tab:meanlinearmetrics}
\end{table}

\begin{table}[ht!]
\centering
\begin{tabular}{ |c||c|c|c|c|c|  }
 \hline
 Συνδυασμός & \en{DR}  & \en{FPR} & \en{Accuracy} & \en{F1 score} & \en{BDR} \\
 \hline
1 & 77.44 & 1.56 & 96.37 & 80.78 & 0.85 \\
  \hline
2 & 79.70 & 1.81 & 96.37 & 81.23 & 0.83 \\
  \hline
3 &78.95 & 2.22 & 95.93 & 79.25 & 0.80 \\
  \hline
4 & 78.95 & 2.05 & 96.07 & 79.85 & 0.81\\
  \hline
5 & 78.20 & 1.81 & 96.22 & 80.31 & 0.83\\
 \hline
6 & 77.44 & 2.14 & 95.85 & 78.63 & 0.80 \\
 \hline
7 & 75.94 & 1.81 & 96.00 & 78.91 & 0.82\\
 \hline
8 & 79.70 & 1.81 & 96.37 & 81.23 & 0.83\\
 \hline
\end{tabular}
\caption{Αποτελέσματα δοκιμής τύπου 1 χωρίς κανονικοποίηση}
\label{tab:exploreclassifiers1nonorm}
\end{table}
\section{Εξερεύνηση διαφορετικών τρόπων κανονικοποίησης}
Το σκέλος της κανονικοποίησης των δεδομένων είναι ζωτικής σημασίας για κάθε σύστημα μηχανικής μάθησης. Η κανονικοποίηση των δεδομένων υλοποιείται, μειώνοντας το εύρος των τιμών σε οποιαδήποτε σχετικά μικρό εύρος. Συνηθέστερη πετυχημένη πρακτική είναι η αναγωγή των τιμών σε εύρος [0,1] ή [-1,1] με στόχο την βελτίωση της επίδοσης και της ταχύτητας του αλγορίθμου. Αυτό επιτυγχάνεται σε μεγάλο βαθμό στην συγκεκριμένη περίπτωση από την κανονικοποίηση στο εύρος [0,1], βελτιώνοντας σε μικρό βαθμό τις μετρικές και μειώνοντας σχεδόν 10 φορές τον χρόνο εκτέλεσης της εκπαίδευσης. Στον Πίνακα \ref{tab:explorenormalization} παραθέτονται τα αποτελέσματα των βέλτιστων ταξινομητών σε κάθε είδος κανονικοποίησης.
\begin{table}[ht!]
\centering
\begin{tabular}{ |c|c|c|c|c|c|c|c|  }
 \hline
 Συνδυασμός& Κανονικοποίηση & \en{DR}  & \en{FPR} & \en{Accuracy} & \en{F1 score} & \en{BDR}& χρόνος εκπαίδευσης (s) \\
 \hline
4 & [0,1] & 80.87 & 1.54 & 96.96 & 81.94 & 0.85 & 6.492741\\
  \hline
4 & [-1,1]& 91.67 & 21.23 & 80.15 & 49.62 & 0.32 & 551.264250\\
  \hline
2 & - &79.70 & 1.81 & 96.37 & 81.23 & 0.83 & 58.246916 \\
  \hline
\end{tabular}
\caption{Αποτελέσματα κανονικοποιήσεων}
\label{tab:explorenormalization}
\end{table}

\section{Εξερεύνηση χρονικής υποδιαίρεσης χρονοσειρών}
Ολοκληρώνοντας την εξερεύνηση των ταξινομητών απαιτείται να γίνει έλεγχος στις χρονικές υποδιαιρέσεις των χρονοσειρών. Για αυτό το σκοπό έγινε δοκιμή του πιο εύστοχου ταξινομητή σε 2.000 καταναλωτές με ποσοστό ρευματοκλοπών 10\% και μόνο απάτες τύπου 1. Στη δοκιμή οι χρονοσειρές διαιρέθηκαν σε ημερήσιες, ωριαίες και ημίωρες μετρήσεις λαμβάνοντας υπόψη όχι μόνο τις μετρικές ευστοχίας, αλλά και τον χρόνο εκτέλεσης της εκπαίδευσης κάθε ταξινομητή. Στον Πίνακα \ref{tab:timedivision} εμφανίζεται όπως αναμενόταν πως όσο αυξάνεται η συχνότητα των μετρήσεων τόσο πιο εύστοχος γίνεται ο ταξινομητής. Παρόλα αυτά ο χρόνος εκτέλεσης της εκπαίδευσης φαίνεται να επηρεάζεται έντονα από διαφορετικές χρονικές υποδιαιρέσεις με την ταξινόμηση με συχνότητα λήξης ανά ημέρα να είναι σημαντικά γρηγορότερη από τις υπόλοιπες, αλλά παρουσιάζει σχετική δυσκολία στην αναγνώριση της απάτης. 

\begin{table}
\centering
\begin{tabular}{ |c||c|c|c|c|c|c|  }
 \hline
 Συχνότητα & \en{DR}  & \en{FPR} & \en{Accuracy} & \en{F1 score} & \en{BDR} & χρόνος εκπαίδευσης \en{(s)}\\
 \hline
μέρες & 81.62 & 2.55 & 95.85 & 79.86 & 0.78 & 0.069182\\
 \hline
ώρες & 82.88 & 2.16 & 96.22 & 82.59 & 0.81& 4.152410\\
  \hline
ημίωρα & 81.08 & 1.66 & 96.44 & 83.33 & 0.84& 12.169304\\
  \hline
\end{tabular}
\caption{Αποτελέσματα δοκιμής χρονικής υποδιαίρεσης}
\label{tab:timedivision}
\end{table}

\section{Θεωρία Μηχανών Διανυσμάτων Υποστήριξης}
Για την ταξινόμηση με μηχανές διανυσμάτων υποστήριξης επιλέχθηκε η βιβλιοθήκη \en{LIBSVM}, η οποία προέρχεται τους ίδιους δημιουργούς της \en{LIBLINEAR}. Σκοπός του \en{SVM} είναι η παραγωγή μοντέλων (βαση των δεδομένων εκπαίδευσης), τα οποία προβλέπουν τα χαρακτηριστικά των δεδομένων δοκιμής βάση μόνο των πληροφοριών που αντλούνται από τις τιμές των δεδομένων.\par
Ξεκινώντας από τα δεδομένα εκπαίδευσης έχουμε ζευγάρια παραδειγμάτων-δυαδικών χαρακτηριστικών $(\mathbf{x}_i,y_i)$,$i=1,...,l$ όπου $\mathbf{x}_i\in\R^n$ και $y\in\{1,-1\}^l$, οι μηχανές διανυσμάτων  υποστήριξης \en{(SVM)} απαιτούν την λύση του παρακάτω προβλήματος βελτιστοποίησης:
\begin{center}
$min_{\mathbf{w},\mathbf{b},\mathbf{\xi}} \frac{1}{2}\mathbf{w}^T\mathbf{w}+C\sum_{i=1}^l\xi_i$\\
δεδομένου $y_i(\mathbf{w}^T\phi(\mathbf{x}_i)+b)\geq 1-\xi_i$,\\
$\xi_i \geq 0$
\end{center}
Εδώ τα διανύσματα εκπαίδευσης $\mathbf{x}_i$, ανάγονται σε μεγαλύτερο (ίσως άπειρο) χώρο διαστάσεων από τη συνάρτηση $\phi$. Τα \en{SVM} βρίσκουν ένα γραμμικά διαχωρίσιμο υπερεπίπεδο με μέγιστο περιθώριο σε αυτό χώρο ανώτερων διαστάσεων. $C>0$ είναι ο παράγοντας που θέτει ποινή στον παράγοντα λάθους (\en{error term}). Επιπροσθέτως, η σχέση $K(\mathbf{x}_i,\mathbf{x}_j)\equiv \phi(\mathbf{x}_i)^T\phi(\mathbf{x}_j)$ ονομάζεται συνάρτηση πυρήνα. Παρόλο που νέοι πυρήνες προτείνονται από ερευνητές, έχουν θεσπιστεί οι ακόλουθοι:
\begin{itemize}
\item Γραμμικός: $K(\mathbf{x}_i,\mathbf{x}_j)= \mathbf{x}_i^T\mathbf{x}_j$.
\item Πολυωνυμικός: $K(\mathbf{x}_i,\mathbf{x}_j)= (\gamma\mathbf{x}_i^T\mathbf{x}_j+r)^d$, $\gamma>0$.
\item \en{RBF}: $K(\mathbf{x}_i,\mathbf{x}_j)= exp(-\gamma\|\mathbf{x}_i-\mathbf{x}_j\|^2)$, $\gamma>0$.
\item Σιγμοειδής: $K(\mathbf{x}_i,\mathbf{x}_j)= tanh(\gamma\mathbf{x}_i^T\mathbf{x}_j +r)$.
\end{itemize}
Εδώ τα $\gamma$, $r$ και $d$ είναι παράμετροι των πυρήνων \cite{libsvmguide}.
\section{Δοκιμή ταξινόμησης με Μηχανές Διανυσμάτων Υποστήριξης}
Η προτεινόμενη διαδικασία που ακολουθείται από τους δημιουργούς του \en{LIBSVM} είναι η εξής:
\begin{itemize}
\item Μετατροπή των δεδομένων σε μορφή αναγνωρίσιμη μορφή με το πακέτο \en{SVM}
\item Κανονικοποίηση δεδομένων
\item Εξέταση του \en{RBF} πυρήνα
\item Χρήση \en{cross-validation} για την εύρεση των βέλτιστων παραμέτρων $C$ και $\gamma$
\item Χρήση των βέλτιστων παραμέτρων $C$ και $\gamma$ για την εκπαίδευση των δεδομένων εκπαίδευσης
\item Δοκιμή
\end{itemize}
Έχοντας τη διαδικασία αυτή υπόψη δοκιμάστηκαν επιτυχώς δύο διαφορετικά σενάρια ταξινόμησης. Στο πρώτο σενάριο ταξινομήθηκαν οι χρονοσειρές κάθε καταναλωτή βάση της ετήσιας κατανάλωσης τους και αναγνωρίζοντας κάθε τύπο κλοπής. Στο δεύτερο σενάριο χρησιμοποιήθηκε ο πυρήνας \en{RBF} και έγινε μια προσέγγιση στην αναγνώριση των ημερήσιων μη τεχνικών απωλειών ταξινομώντας σε πρώτη φάση τις ημέρες όλων των καταναλωτών και σε δεύτερη φάση κάθε καταναλωτή \cite{libsvmguide}.
\subsection{Δοκιμή χρονοσειρών χωρίς πυρήνα}
Δεδομένης της ευστοχίας των γραμμικών ταξινομητών θεωρήθηκε αναγκαία η δοκιμή του γραμμικού πυρήνα \en{SVM}. Παρόλα αυτά, η διαίσθηση δεν ήταν η μόνη κινητήριος δύναμη για την υλοποίηση αυτής της δοκιμής. Γενικότερα, αν ο αριθμός των μετρήσεων είναι μεγάλος δεν απαιτείται να αναχθούν τα δεδομένα σε χώρο ανώτερων διαστάσεων. Πρακτικά αυτό σημαίνει πως η μη-γραμμική αναγωγή δεν βελτιώνει την επίδοση του συστήματος. Ενώ, είναι γενικώς αποδεκτό ότι ο πυρήνας \en{RBF} είναι τουλάχιστον καλύτερος από το γραμμικό, αυτή η δήλωση είναι αληθής μόνο αφού έχουν επιλεχθεί οι παράμετροι ($C$,$\gamma$). Ένας γενικός κανόνας χρήσης του γραμμικού πυρήνα είναι η χρήση του όταν ο αριθμός των παραδειγμάτων (καταναλωτών) είναι μικρότερος ή σχετικός με τον αριθμό των χαρακτηριστικών (ωριαίες μετρήσεις έτους).
\subsubsection{Αποτελέσματα δοκιμής}
Η δοκιμή έγινε σε 4.500 καταναλωτές ελέγχοντας αρχικά την επίδοση του συστήματος σε κάθε τύπο απάτης με ποσοστό ρευματοκλοπής 10\%. Στον Πίνακα \ref{tab:linearSVMtypes} φαίνονται τα αποτελέσματα της δοκιμής. Γίνεται, λοιπόν σαφές πως ο ταξινομητής μπορεί να αναγνωρίσει με αξιοπιστία μόνο τις απάτες τύπου 1, όπως και οι αντίστοιχοι ταξινομητές της \en{LIBSVM}. Παρόλα αυτά ακόμα και στα χαμηλότερα αποτελέσματα έχουμε ικανοποιητικό \en{Accuracy} που δείχνει ότι ο ταξινομητής λειτουργεί όπως αναμενόταν.

\begin{table}
\centering
\begin{tabular}{|c||c|c|c|c|c|c|}
\hline
Τύπος & \en{DR}  & \en{FPR} & \en{Accuracy} & \en{F1 score} & \en{BDR} & χρόνος εκτέλεσης\\
\hline
1 & 81.43 & 1.24 & 96.96 & 84.76 & 0.88 & 10.188667\\ 
\hline
2 & 22.63 & 7.25 & 85.63 & 24.22 & 0.26 & 39.489221\\
\hline
3 & 23.78 & 10.36 & 82.67 & 22.52 & 0.20 & 39.648516\\
\hline
Μικτός & 27.13 & 7.37 & 86.37 & 27.56 & 0.29 & 36.836504\\
\hline
\end{tabular}
\caption{Αποτελέσματα Γραμμικού \en{SVM} σε όλους τους τύπους απάτης}
\label{tab:linearSVMtypes}
\end{table}

\subsection{Δοκιμή ημερήσιων χαρακτηριστικών με πυρήνα \en{RBF}}
Σε αυτή τη φάση, δημιουργήθηκε η ανάγκη για εξαγωγή χαρακτηριστικών, ώστε να μειωθούν οι διαστάσεις των πινάκων και να επιταχυνθεί η διαδικασία. Παράλληλα, παρέχει ένα επίπεδο αποπροσωποποίησης δημιουργώντας ένα αποτύπωμα της καταναλωτικής συνήθειας.\cite{giwrgis}. Μετρώντας τα αθροίσματα, τα ελάχιστα, τα μέγιστα και τους μέσους όρους των καθημερινών καταναλώσεων δημιουργείται ένας βασικός κορμός χαρακτηριστικών για κάθε καταναλωτή που μπορεί εύκολα να επεκταθεί και σε άλλα γραμμικά και μη εξαρτώμενα χαρακτηριστικά.
\begin{itemize}
\item \textit{Μέγιστο και ώρα μεγίστου}
\item \textit{Ελάχιστο και ώρα ελαχίστου}
\item \textit{Άθροισμα κατανάλωσης ανά ημέρα}
\item \textit{Μέσος όρος, διακύμανση και τυπική απόκλιση ανά ημέρα}
\item \textit{Παράγοντας φορτίου, ελάχιστο προς μέση τιμή, ελάχιστο προς μέγιστο}
\item \textit{Επίδραση βραδινής κατανάλωσης}
\item \textit{Λοξότητα και Κύρτωση}
\end{itemize}
Η πρώτη δοκιμή του \en{SVM} έγινε με επιλογή 300 τυχαίων καταναλωτών μιας περιοχής με σκοπό να εκπαιδευτεί το σύστημα ώστε να μπορεί να αναγνωρίζει ημέρες απάτης μέσα στο έτος. Η εκπαίδευση του ταξινομητή γινόταν με τα ημερήσια χαρακτηριστικά για κάθε καταναλωτή μαζί με τον \en{confusion matrix}. Τα δεδομένα διαχωρίζονται σε 2 κομμάτια, το κομμάτι της εκπαίδευσης που περιέχει ένα μεγάλο ποσοστό δεδομένων κάθε καταναλωτή και το κομμάτι της δοκιμής που περιέχει ένα ποσοστό της τάξης του $0.30$ από τους αντίστοιχους καταναλωτές.\par
Ο ταξινομητής λοιπόν, εκπαιδεύεται με ημερήσια χαρακτηριστικά κάθε καταναλωτή, αλλά θα πρέπει να αποφανθεί στο τέλος αν ο καταναλωτής έχει νοθεύσει τις μετρήσεις του ή όχι. Η λύση δόθηκε εισάγοντας ένα όριο ημερών που αν ο ταξινιμητής το προσπερνούσε τότε ο καταναλωτής θεωρείται πως έχει αλλοιώσει τα δεδομένα του. Για να βρούμε την βέλτιστη τιμή αυτού του ορίου χρησιμοποιήθηκαν \en{ROC} καμπύλες για να παρατηρηθεί η μεταβολή του \en{DR} και \en{FPR}, ενώ αλλάζει το όριο ημερών.\\
\subsubsection{Αποτελέσματα δοκιμής}
Ελέγχοντας τα αποτελέσματα του Πίνακα \ref{tab:ROC50data} παρατηρείται πως επιλέγοντας όριο στις 10 ημέρες επιτυγχάνεται ακρίβεια της τάξης τους 0.95 στην εύρεση της απάτης, αλλά με σχετικά υψηλό ποσοστό λάθος συναγερμού της τάξης του 0.15 για τις έντονες απάτες. Αν χρειαστεί να ελαχιστοποιηθεί το \en{FPR} θα πρέπει να επιλεχθεί μια μεγαλύτερη οριακή τιμή όπως το 14, που έχει ικανοποιητικό ποσοστό και στο \en{DR} που είναι της τάξης του 0.85 και του \en{FPR} που είναι της τάξης του 0.08. Οι απάτες που έγιναν με μικρότερη ένταση δεν γίνονται αντιληπτες από τον ταξινομητή που επιστρέφει καμπύλη με παρόμοια κλίση με της ευθείας αναφοράς.\par
Αντίστοιχα στον Πίνακα \ref{tab:ROC35data} φαίνεται πως η μείωση του \en{FR} επηρέασε το σύστημα, και ειδικότερα μείωσε το όριο στις 10 μέρες με \en{DR}=0.85 και \en{FPR}=0.09. Ουσιαστικά φαίνεται πως το σύστημα χρειάζεται και άλλους καταναλωτές ώστε να αποτυπωθούν και οι καμπύλες για χαμηλότερες εντάσεις διείσδυσης στα δεδομένα. 

\begin{figure}[ht!]
\centering
\includegraphics[width=140mm, height=100mm]{../../plots/ROC50_1.png}
\caption{Καμπύλη \en{ROC} για \en{FR}=0.50 \label{fig:ROC50}}

\begin{center}
\begin{tabular}{ |c|c|c|c|c|  }
 \hline
 \multicolumn{5}{|c|}{\en{300 IDs, 0.5 rate, 0-100 threshold}} \\
 \hline
 Όριο (Μέρες)    & \en{DR} (0.8) & \en{FPR} (0.8) & \en{DR} (0.5) & \en{FPR} (0.5) \\
 \hline
2 &	 97,917 &	40,385 &	76,712 &	35,0649\\
4 &	 97,143 &	25,625 & 	65,972 &	22,436\\
6 &	 95,683 &	22,360 &	54,225 &	13,924\\
8 &	 95,588 &	16,463 &	45,588 &	6,098\\
10 & 96,241 &	13,772 &	37,879 &	3,571\\
12 & 90,698 &	10,526 &	31,783 &	1,17\\
14 & 86,614 &	8,671  &	26,190 &	1,149\\
16 & 84,8	&	5,714  &	19,355 &	0\\
18 & 82,787 &	6,18   &	15,702 &	0\\
20 & 79,832 &	5,525  &	11,667 &	0\\
\hline
\end{tabular}
\end{center}
\caption{Πίνακας επιλογής ορίου \en{FR}=0.5 \label{tab:ROC50data}}
\end{figure}



\begin{figure}[ht!]
\centering
\includegraphics[width=140mm, height=100mm]{../../plots/ROC35.png}
\caption{Καμπύλη \en{ROC} για \en{FR}=0.35 \label{fig:ROC35}}


\begin{center}
\begin{tabular}{ |c|c|c| }
 \hline
 \multicolumn{3}{|c|}{\en{300 IDs, 0.35 rate, 0-100 threshold}} \\
 \hline
 Όριο (Μέρες)   & \en{DR} (0.8) & \en{FPR} (0.8)\\
 \hline
2 &	 95,192 &	30,612\\
4 &	 91,176 &	23,232\\
6 &	 89,691 &	17,734\\
8 &	 85,567 &	12,808\\
10 & 85,106 &	8,738\\
12 & 84,444 &	5,238\\
14 & 79,545 &	2,830\\
16 & 75		&	2,830\\
18 & 68,235 &	2,791\\
20 & 63,529 &	2,791\\
\hline
\end{tabular}
\end{center}
\caption{Πίνακας επιλογής ορίου \en{FR}=0.35 \label{tab:ROC35data}}
\end{figure}

\section{Σχόλια}