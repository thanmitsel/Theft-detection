Στην παρούσα διπλωματική αντιμετωπίστηκαν δυσκολίες που εν μέρει όρισαν τη μελλοντική πορεία του ζητήματος. Υπήρξαν δύο ειδών τεχνικές δυσκολίες στην ανίχνευση μη τεχνικών απωλειών σε ετήσια δεδομένα. Η πρώτη βασίζεται στο γεγονός ότι πρόκειται για χρονοσειρές διάρκειας ενός έτους, στις οποίες δεν μπορεί εύκολα να αποτυπωθεί μια αξιόπιστη καταναλωτική συμπεριφορά. Η δεύτερη σχετίζεται με την ευρεία χρήσης και δοκιμή πολλών ταξινομητών και την ανάγκη να λαμβάνονται υπόψη οι ιδιαίτερες απαιτήσεις του καθενός. Παράλληλα, πρέπει να καθοριστεί και ένα όριο στην αξιοπιστία των συστημάτων μηχανικής μάθησης, καθώς ένα αποτελεσματικό σύστημα πρέπει να έχει σιγουριά στον εντοπισμό του ζητούμενου συμβάντος και να ελαχιστοποιεί τα περιθώρια λάθους εκτίμησης.\par
Στην πιο ευρεία σφαίρα του ζητήματος, τίθενται θέματα προστασίας της ιδιωτικότητας των καταναλωτών. Από την άλλη, τους δίνεται η δυνατότητα ανωνυμοποίησης των δεδομένων τους \cite{anonymization}, γεγονός που δυσκολεύει σε μεγάλο βαθμό την εξόρυξη δεδομένων σε επόμενα στάδια. Με την ύπαρξη των έξυπνων μετρητών ανοίγεται ένα παράθυρο που εκθέτει τις προσωπικές δραστηριότητες σε οποιονδήποτε έχει πρόσβαση σε καταναλωτικές πληροφορίες. Οι τεράστιες δυνατότητες που ανοίγονται στην αναλυτική μελέτη χρονοσειρών εγείρουν ζητήματα προστασίας των προσωπικών δεδομένων.
\section{Τεχνικά εμπόδια}
Η αντιμετώπιση τεχνικών θεμάτων πάντα απαιτεί λεπτομερή ανάλυση της δυσκολίας και λήψεις αποφάσεων. Η έκταση των δεδομένων αποδείχθηκε σχετικά μικρή, καθώς τα συστήματα δεν είχαν τη δυνατότητα παρατήρησης των καταναλωτικών συνηθειών σε μεγάλο βάθος χρόνου. Το συγκεκριμένο πρόβλημα γεννά νέες δυσκολίες και μπορεί να προκαλέσει την αναξιοπιστία του συστήματος σε δεδομένα άλλων χρονικών περιόδων. Τέλος, αξίζει να ληφθεί υπόψη πως η διαδικασία εύρεσης και επεξεργασίας δεδομένων και χαρακτηρισμών τους είναι εξαιρετικά επίπονη και απαιτεί εμπιστοσύνη στην πηγή τους.
\subsection{Έλλειψη μακροχρόνιων δεδομένων}
Για να μπορέσει να αντιμετωπιστεί το ζήτημα των μη τεχνικών απωλειών με μακροπρόθεσμο ορίζοντα, απαιτείται η βαθιά κατανόηση της συχνότητας των προτύπων και των στιγμιοτύπων των χρονοσειρών. Με αυτό τον τρόπο, αναλύονται σε βάθος οι καταναλωτικές συνήθειες και γνωστοποιούνται οι μεταβλητές που τις επηρεάζουν. Τα δεδομένα της παρούσας εργασίας αφορούσαν χρονικό διάστημα που δεν ξεπερνούσε τα δύο έτη. Με τέτοιο εύρος μετρήσεων ήταν λοιπόν λογικό να περιοριστούν οι δοκιμές σε ενός έτους.\par
Εκεί που εγείρεται η σημαντική δυσκολία είναι το γεγονός ότι οι καταναλωτές ταξινομούνται με ένα και μόνο έτος αναφοράς. Ειδικότερα, τα συστήματα χρησιμοποιούν τις γενικές καταναλωτικές συνήθειες του έτους για να ταξινομήσουν κάθε καταναλωτή με αυτά τα κριτήρια. Η πιο ασφαλής προσέγγιση, για να κριθεί ένα έτος ύποπτο, θα απαιτούσε να υπάρχει μεγάλο χρονικό παράθυρο κατανάλωσης, ώστε να μπορεί εύκολα κάποιος να παρατηρήσει μια ασυνήθιστη τάση των δεδομένων. Έτσι, θα μπορούσαν να οργανωθούν ευκολότερα οι καταναλωτές σε ομάδες που θα είχαν μια γενικότερη ομοιότητα ως προς τις καταναλωτικές συνήθειες.
\subsection{Έλλειψη παραδειγμάτων}
Παράλληλα, έχει νόημα να παρατηρηθεί πως το δείγμα των καταναλωτών δεν είναι τελείως αντιπροσωπευτικό ως προς τη δυνατότητα γενίκευση σε μεγαλύτερο πληθυσμό. Ειδικότερα, οι 4.500 καταναλωτές θα μπορούσαν να είχαν πολύ διαφορετικές συνήθειες, αν ζούσαν σε διαφορετική τοποθεσία, άρα και διαφορετικές χρονοσειρές που θα εξετάζονταν διαφορετικά, αν απέκλιναν σημαντικά από τις υπάρχουσες. Το πρόβλημα εντείνεται, παρακολουθώντας την ομοιογένεια των τύπων των καταναλωτών. Εμφανίζεται, δηλαδή, μια κυρίαρχη ομάδα που έχει σχετική ομοιογένεια μεταξύ της και αποτελείται από νοικοκυριά και οικιακούς χρήστες. Στην ομάδα αυτή ανήκουν τουλάχιστον τα τρία τέταρτα του δείγματος, γεγονός που έχουν εκμεταλλευτεί  τα συστήματα ταξινόμησης, αλλά αίρονται ερωτήματα για το υπόλοιπο ένα τέταρτο του πληθυσμού, το οποίο στελεχώνεται από καταναλωτές με υψηλές ενεργειακές απαιτήσεις, δηλαδή από μικρομεσαίες επιχειρήσεις. Αυτό το μικρό δείγμα δεν μπορεί να εξάγει εύκολα μια γενικευμένη συμπεριφορά που να εκφράζει όλο το σύνολο, καθώς κάθε επιχείρηση ανάλογα με τις ανάγκες της προσαρμόζει τη λειτουργία της. Το αποτέλεσμα είναι να έχουμε ένα ικανοποιητικό πλήθος ομοιόμορφων καταναλωτών που εξάγουν όμοια χαρακτηριστικά και ένα μικρό υποσύνολο των δεδομένων με επιχειρήσεις, που έχουν μεγάλες και αδιευκρίνιστες ανάγκες.
\subsection{Δυσκολία επιλογής μετρικών}
Στην παρούσα διπλωματική χρησιμοποιήθηκε πλήθος αλγορίθμων μηχανικής μάθησης με καθένα να έχει τα δικά του ιδιαίτερα χαρακτηριστικά. Δημιουργήθηκε, λοιπόν, η ανάγκη σύγκρισης των αλγορίθμων βάσει κάποιων απόλυτων μετρικών για την τελική αξιολόγησή τους. Ειδικότερα, οι επιβλεπόμενοι αλγόριθμοι χρησιμοποιούν 70\% των δεδομένων για εκπαίδευση και το 30\% για προβλέψεις, οι μη επιβλεπόμενοι δεν χρησιμοποιούν εκπαίδευση για τη δημιουργία μοντέλου πρόβλεψης, ενώ οι ημι-επιβλεπόμενοι χρησιμοποιούν 70\% για την εξαγωγή του στατιστικού μοντέλου και την πρόβλεψη και 30\% για τη βελτιστοποίηση του μοντέλου. Όπως γίνεται αντιληπτό, οι προβλέψεις γίνονται σε διαφορετικά δείγματα των πληθυσμών, δημιουργώντας απαίτηση για αξιόπιστες μετρικές.\par
Τα \en{DR} και \en{FPR} μπορούν γρήγορα να δώσουν μια πρώτη αίσθηση για την ευστοχία του αλγορίθμου, αλλά λόγω της ευαισθησίας του προβλήματος δεν πρέπει να θεωρούνται οι κύριες μετρικές. Αυτό οφείλεται στο γεγονός ότι ένας αλγόριθμος με πολύ υψηλό \en{DR} μπορεί να αναγνωρίσει τις κλοπές, αλλά αν έχει \en{FPR} που ξεπερνά το 5\%, οι προβλέψεις δεν θεωρούνται εντελώς αξιόπιστες, καθώς εισάγεται μεγάλο περιθώριο λάθους. Ένας τρόπος να αποτυπωθεί η σχέση μεταξύ του \en{DR} και \en{FPR} είναι το \en{F1 score}, που έχει εξάρτηση και από τις δύο μετρικές και εξάγει ικανοποιητικά αποτελέσματα μόνο με χαμηλό \en{FPR}. Παράλληλα, ένας γενικότερος τρόπος να εξεταστεί η ταξινόμηση είναι με την ευστοχία \en{Accuracy} που πρέπει να βρίσκεται πάντα πάνω από το 90\% και περιγράφει τη γενικότερη πρόβλεψη του συστήματος. Όταν οι αλγόριθμοι έχουν παρόμοιες αυτές τις μετρικές, αξίζει να ελεγχθεί το \en{BDR} που προσφέρει μια πιθανοτική προσέγγιση, ορίζοντας την πιθανότητα πραγματικής κλοπής, δεδομένου ότι προβλέφθηκε.
\subsection{Εύρεση αξιόπιστων δυαδικών χαρακτηρισμών}
Ένα σημαντικός παράγοντας που δεν πρέπει να αμεληθεί είναι η αξιοπιστία και η προέλευση των δυαδικών χαρακτηριστικών των χρονοσειρών. Στην παρούσα εργασία δεν απαιτήθηκε να ευρεθούν τέτοια δεδομένα, καθώς προσομοιώθηκαν οι απάτες. Στην περίπτωση όμως που τα δεδομένα έρχονται με δυαδικούς χαρακτηρισμούς από ένα φορέα, απαιτείται έλεγχος στη μεθοδολογία εξαγωγής των χαρακτηριστικών. Η εγκυρότητα των δυαδικών αυτών διανυσμάτων είναι καίριας σημασίας για την εκπαίδευση και τον έλεγχο του συστήματος, καθώς αποτελέι τη βάση της υλοποίησης των αλγορίθμων και την κινητήριο δύναμη των αλγορίθμων βελτιστοποίησης. Επιπρόσθετα αξίζει να σημειωθεί πως θα μπορούσε να δημιουργηθεί ένα σύστημα με ανατροφοδότηση των φυσικών ελέγχων για τη δημιουργία αξιόπιστων δυαδικών χαρακτηριστικών.\par
\section{Ασφάλεια Καταναλωτών}
Η εισαγωγή των έξυπνων μετρητών στην καθημερινότητά μας δίνει τη δυνατότητα να διερευνηθούν σε βάθος οι καταναλώσεις ενέργειας και διευκολύνει την επικοινωνία των δεδομένων με εγκεκριμένους φορείς. Αυτή όμως η πραγματικότητα έχει και μια σκοτεινή πτυχή που αντιμετωπίζεται στις περισσότερες μελέτες μεγάλης κλίμακας δεδομένων. Οι προσωπικές πληροφορίες των πελατών είναι εκτεθειμένες σε ένα δίκτυο αμφίδρομης επικοινωνίας καταναλωτών και παρόχων, ενώ ανά πάσα στιγμή κάποιος εργαζόμενος μπορεί να ανατρέξει σε αυτές και να της εκμεταλλευτεί για προσωπικούς λόγους.\par
Η σημερινή τεχνολογία των έξυπνων μετρητών που βασίζονται στο \en{NALM} αλγόριθμο, παρέχει τρόπους να αναγνωρίζονται συσκευές σε λειτουργία ακόμη και όταν οι μετρήσεις αφορούν ένα σύνολο νοικοκυριών. Έτσι, κάποιος κακόβουλος χρήστης θα μπορούσε να αντλήσει δεδομένα για το πρόγραμμα των νοικοκυριών, τα είδη των συσκευών τους και τις ανάγκες τους. Ένας τρόπος να αντιμετωπιστεί αυτό το θέμα είναι η διαχείριση της ενεργειακής χρήσης μέσα στο σπίτι, πριν συλλεχθούν τα δεδομένα του μετρητή \cite{anonymization}.\par
Γίνεται λοιπόν σαφές πως οι έξυπνοι μετρητές χωρίς κάποιο σύστημα ανωνυμοποίησης πλήττουν την ιδιωτικότητα των καταναλωτών και εγείρουν θέματα ασφαλείας. Η έρευνα προς αυτή την κατεύθυνση υπερβαίνει το πλαίσιο αυτής τη διπλωματική εργασίας, αλλά ήδη προτείνονται νέοι αλγόριθμοι και δικτυακές δομές, για να μπορέσει να συμβαδίσει η προστασία της ιδιωτικότητας με την αποτελεσματικότητα των ερευνών.

