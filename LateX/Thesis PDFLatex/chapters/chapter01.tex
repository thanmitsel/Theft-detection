Είναι ευρέως διαδεδομένο πως η καθημερινότητα πολλών ανθρώπων συνδέεται άρρηκτα με τη χρήση ηλεκτρικών συσκευών, αλλά και με την ανάγκη ύπαρξης βιομηχανικών εγκαταστάσεων για την εκπλήρωση των καταναλωτικών τους επιθυμιών. Αυτό δημιουργεί μια αυξανόμενη ζήτηση στον τομέα της παραγωγής, της μεταφοράς και διανομής ηλεκτρικής ενέργεια, που με τη σειρά του οδηγεί στον συνεχή εκσυγχρονισμό των εγκαταστάσεων. Παράλληλα, διανύοντας την εποχή της Ψηφιακής Επανάστασης παρατηρείται η μετάβαση από τις αναλογικές τεχνολογίες στις ψηφιακές, γεγονός που δεν θα μπορούσε να αφήσει ανεπηρέαστο τον τομέα της ηλεκτρικής ενέργειας. Η μετάβαση αυτή στον τομέα που μελετάται σε αυτή τη διπλωματική εργασία σηματοδοτείται από την χρήση έξυπνων μετρητών, οι οποίοι έχουν τη δυνατότητα να παρέχουν σε πραγματικό χρόνο μεγάλο όγκο δεδομένων για τα επίπεδα της κατανάλωσης κάθε πελάτη.\par
Ανοίγεται, λοιπόν ένας νέος ορίζοντας εποπτείας και αναλυτικής μελέτης των χρονοσειρών που παράγονται από κάθε καταναλωτή. Η ταυτόχρονη και συνεχής αύξηση των ρευματοκλοπών στις περισσότερες περιοχές του κόσμου καθιστά επιτακτική ανάγκη την εύρεση μεθόδων εντοπισμού τους . Σύμφωνα με τα επίσημα στοιχεία του Διαχειριστή Δικτύου (ΔΕΔΔΗΕ), το 2016 εντοπίσηκαν 10.616 κρούσματα ρευματοκλοπών, μέγεθος που είναι ψηλότερο όλων των εποχών, έναντι 400 το 2006 \cite{energypress}. Άμεσο επακόλουθο της επίλυσης αυτού προβλήματος είναι η ομαλή λειτουργία των παροχέων ενέργειας και η βελτίωση της ποιότητας των υπηρεσιών που παρέχουν οι ίδιες. Στη συνέχεια θα αναπτυχθεί το βαθύτερο αίτιο της παρούσας διατριβής και μια επισκόπηση του περιεχομένου της \cite{theftanalysis}.
\section{Κίνητρο και υπόβαθρο διπλωματικής}
Το πρόβλημα της παράνομης αφαίρεσης ηλεκτρικής ενέργειας ενδιαφέρει τους διαχειριστές δικτύων. Οι χρήστες συχνά παραβιάζουν τους νόμους προσπαθώντας να αλλοιώσουν τα συστήματα μέτρησης. Σε κάποιες χώρες μόνο κάποιο κομμάτι της παραγωγής χρεώνεται, παραδείγματος χάριν στην Ινδία το 55\% της παραγωγής ηλεκτρικής ενέργειας χρεώνεται (και μόνο ένα μέρος της πληρωμής καταλήγει στον πάροχο). Παρόλα αυτά, η παράνομη χρήση ενέργειας λαμβάνει χώρα και σε Ευρωπαϊκές χώρες. Μια από τις κινητήριες δυνάμεις για το λανσάρισμα των αυτοματοποιημένων υποδομών ανάγνωσης μετρητών (\en{Automated Meter Reading}) για τον ιταλικό πάροχο ενέργειας (\en{ENEL}) ήταν η προσπάθεια ελαχιστοποίησης των μη τεχνικών απωλειών στο δίκτυα διανομής τους. Η μείωση των ρευματοκλοπών βοήθησε στην αιτιολόγηση μεγάλων επενδύσεων σε \en{AMR} και επί του παρόντος η Ιταλία πρωταγωνιστεί στην διείσδυση \en{AMR} \cite{india},\cite{regulation}.\\ 
Μερικοί μπορεί να υποστηρίζουν ότι οι εταιρίες παραγωγής και διανομής, οι οποίες έχουν σημαντικό έργο παρέχουν κακή εξυπηρέτηση, υπερχρεώνουν, κερδίζουν ανεξαρτήτως αρκετά χρήματα και ως εκ τούτου, ένα ποσοστό κλοπής δεν θα καταστρέψει την εταιρία ή θα επηρεάσει δραστικά τις λειτουργίες και την κερδοφορία της. Άλλοι παρατηρώντας την ίδια κατάσταση θα υποστήριζαν ότι η κλοπή είναι έγκλημα και δεν θα έπρεπε να επιτρέπεται. Η Διεθνής Εταιρία Προστασίας Εσόδων των Πάροχων (\en{ International Utilities Revenue Protection Association}) έχει καθιερωθεί για να προάγει τον εντοπισμό και την πρόληψη της κλοπής ρεύματος κυρίως για την οικονομική ασφάλεια των εταιριών παροχής ενέργειας.\par
Οι συνέπειες της κλοπής είναι εξαιρετικά σημαντικές και μπορούν να επηρεάσουν άμεσα τη βιωσιμότητα των υπηρεσιών που παρέχονται. Οι συνδιασμένες απώλειες (συμπεριλαμβάνοντας και τους απλήρωτους λογαριασμούς) σε μερικά συστήματα έχουν σοβαρές επιπτώσεις που έχουν ως αποτέλεσμα οι εγκαταστάσεις να λειτουργούν σε καθεστώς μεγάλων απωλειών και αναγκάζονται να αυξάνουν συνεχώς τα ηλεκτρικά φορτία. Απομονωμένες σε μια κουλτούρα αναποτελεσματικότητας και διαφθοράς, οι εταιρίες έχουν μεγάλη δυσκολία να παρέχουν αξιόπιστες υπηρεσίες. Ακόμη και σε αποτελεσματικά συστήματα ισχύος, όπως η \en{Tenaga} της Μαλαισίας, η κλοπή ρεύματος ανέρχεται στα \$132 εκατομμύρια ετησίως \cite{malaysia}. Αντίστοιχα στην Ελλάδα η συνολική εγχεόμενη ενέργεια στο Δίκτυα Διανομής ανήλθε το 2016 σε 47.655.372 \en{MWh}, το σύνολο των ρευματοκλοπών εκτιμάται σε 1.525.292 \en{MWh}. Στην πραγματικότητα όμως το μέγεθος των ρευματοκλοπών είναι αρκετά μεγαλύτερο, επιβαρύνει δε κατά κύριο λόγο τη Δημόσια Επιχείρηση Ηλεκτρισμού (ΔΕΗ). Ωστόσο παίρνοντας ως δεδομένη την ποσότητα, που αναγνωρίζει η Ρυθμιστική Αρχή Ενέργειας (ΡΑΕ), τα έσοδα που διαφεύγουν κάθε χρόνο λόγω των ρευματοκλοπών με βάση τις μοναδιαίες τιμές του 2016 έχουν ως εξής \cite{dehfraud}:

\begin{table}[ht!]
\centering
\begin{tabular}{ |c||c|  }
 \hline
 Εταιρίες & εκατ. \euro\\
 \hline
 ΔΕΗ & 120-125\\
 Υπηρεσίες Κοινής Ωφέλειας (ΥΚΩ) & 21\\
 ΕΤΜΕΑΡ & 32\\
 ΑΔΜΗΕ 4 & 7,3\\
 ΔΕΔΔΗΕ 5 & 26,5\\
 \hline
 Σύνολο& 206,8 έως 211,8\\
 \hline
\end{tabular}
\caption{Διαφεύγοντα έσοδα Ελληνικών πάροχων}
\label{tab:lostearnings}
\end{table}
\subsection{Ορίζοντας τις ρευματοκλοπές}
Υπάρχουν τέσσερα επικρατούντα είδη "κλοπής" σε όλα τα συστήματα ενέργειας. Η έκταση της κλοπής εξαρτάται από πλήθος παραγόντων από πολιτιστικές μέχρι τον τρόπο που διαχειρίζεται η ενέργεια.

\subsection{Επέμβαση στο μετρητή}
Επέμβαση στο μετρητή ορίζεται όταν ο καταναλωτής σκοπίμως προσπαθεί να εξαπατήσει τον πάροχο. Μια συνήθης πρακτική είναι να παραβιάζει το μετρητή ώστε να καταγράφει χαμηλότερα ποσά ενέργειας από τα πραγματικά. Αυτό εν γένει είναι μια  επικίνδυνη διαδικασία για ένα ερασιτέχνη, και σε πολλές περιπτώσεις έχουν καταγραφεί ηλεκτροπληξίες.
\subsubsection{Απευθείας Σύνδεση}
Η κλοπή ενέργειας επιτευχθεί τραβώντας μια γραμμή από την από το δίκτυο διανομής μέχρι το επιθυμητό σημείο παρακάμπτοντας το μετρητή. Ένας καθιερωμένος τρόπος κλοπή ενέργειας στην Ελλάδα είναι η απευθείας σύνδεση με αγκίστρωση στους αγωγούς του εναέριου δικτύου, απουσία μετρητικής διάταξης ή παροχής ή νομίμως υφιστάμενου κτίσματος \cite{rae}.
\subsubsection{Ακανόνιστες χρεώσεις}
Οι ακανόνιστες χρεώσεις μπορούν να συμβούν από πολλές πηγές. Κάποιο οργανισμοί παροχής ενέργειας μπορεί να μην είναι αρκετά αποτελεσματικοί στη μέτρηση της ενέργειας που έχει καταναλωθεί και ακούσια μπορεί να δώσουν υψηλότερη ή χαμηλότερη μέτρηση από την ακριβή. Αυτές οι ακανόνιστες χρεώσεις μπορεί να ισοζυγιστούν με την πάροδο του χρόνου. Παρόλα αυτά, είναι πολύ εύκολο σε μερικά συστήματα να κανονιστούν πολύ χαμηλότεροι λογαριασμοί από τους ρεαλιστικούς. Εργαζόμενοι μπορεί να δωροδοκηθούν για να καταγράψουν το μετρητή με μικρότερο νούμερο από αυτό που ενδεικνύεται. Ο καταναλωτής πληρώνει μικρότερο λογαριασμό και ο εργαζόμενος που καταγράφει τις μετρήσεις αποκτά ανεπίσημο μισθό. Οικιακοί ή επιχειρηματικοί καταναλωτές  μπορεί να έχουν φύγει από την πόλη ή την εγκατάσταση λόγω χρεωκοπίας. .
\subsubsection{Απλήρωτοι λογαριασμοί}
Κάποια άτομα και κάποιοι οργανισμοί δεν πληρώνουν αυτά που οφείλουν για ηλεκτρική ενέργεια. 
\section{Δομή Διπλωματικής}