Είναι ευρέως διαδεδομένο πως η καθημερινότητα πολλών ανθρώπων συνδέεται άρρηκτα με τη χρήση ηλεκτρικών συσκευών, αλλά και με την ανάγκη ύπαρξης βιομηχανικών εγκαταστάσεων για την εκπλήρωση των καταναλωτικών τους επιθυμιών. Αυτό δημιουργεί μια αυξανόμενη ζήτηση στον τομέα της παραγωγής, της μεταφοράς και διανομής ηλεκτρικής ενέργειας, που με τη σειρά του οδηγεί στον συνεχή εκσυγχρονισμό των εγκαταστάσεων. Παράλληλα, διανύοντας την εποχή της Ψηφιακής Επανάστασης παρατηρείται η μετάβαση από τις αναλογικές τεχνολογίες στις ψηφιακές, γεγονός που δεν θα μπορούσε να αφήσει ανεπηρέαστο τον τομέα της ηλεκτρικής ενέργειας. Η μετάβαση αυτή στον τομέα που μελετάται σε αυτή τη διπλωματική εργασία σηματοδοτείται από την χρήση έξυπνων μετρητών, οι οποίοι έχουν τη δυνατότητα να παρέχουν μεγάλο όγκο δεδομένων για τα επίπεδα της κατανάλωσης κάθε πελάτη.\par
Ανοίγεται, λοιπόν ένας νέος ορίζοντας εποπτείας και αναλυτικής μελέτης των χρονοσειρών που παράγονται από κάθε καταναλωτή. Η ταυτόχρονη και συνεχής αύξηση των ρευματοκλοπών στις περισσότερες περιοχές του κόσμου καθιστά επιτακτική ανάγκη την εύρεση μεθόδων εντοπισμού τους . Σύμφωνα με τα επίσημα στοιχεία του Διαχειριστή Δικτύου (ΔΕΔΔΗΕ), το 2016 εντοπίσηκαν 10.616 κρούσματα ρευματοκλοπών, μέγεθος που είναι ψηλότερο όλων των εποχών, έναντι 400 το 2006 \cite{energypress}. Άμεσο επακόλουθο της επίλυσης αυτού προβλήματος είναι η ομαλή λειτουργία των παροχέων ενέργειας και η βελτίωση της ποιότητας των υπηρεσιών που παρέχουν οι ίδιες. Στη συνέχεια θα αναπτυχθεί το βαθύτερο αίτιο της παρούσας διατριβής και μια επισκόπηση του περιεχομένου της \cite{theftanalysis}.
\section{Κίνητρο και υπόβαθρο διπλωματικής}
Το πρόβλημα της παράνομης αφαίρεσης ηλεκτρικής ενέργειας ενδιαφέρει τους διαχειριστές δικτύων. Οι χρήστες συχνά παραβιάζουν τους νόμους προσπαθώντας να αλλοιώσουν τα συστήματα μέτρησης. Σε κάποιες χώρες μόνο κάποιο κομμάτι της παραγωγής χρεώνεται, παραδείγματος χάριν στην Ινδία το 55\% της παραγωγής ηλεκτρικής ενέργειας χρεώνεται και το υπόλοιπο καταναλώνεται χωρίς να περάσει από μετρητικές συσκευές \cite{india}. Παράλληλα, μόνο ένα μέρος της πληρωμής καταλήγει στον πάροχο, λόγο απλήρωτων λογαριασμών και περιφερειακών χρεώσεων. Παρόλα αυτά, η παράνομη χρήση ενέργειας λαμβάνει χώρα και σε Ευρωπαϊκές χώρες. Μια από τις κινητήριες δυνάμεις για το λανσάρισμα των αυτοματοποιημένων υποδομών ανάγνωσης μετρητών (\en{Automated Meter Reading}) για τον ιταλικό πάροχο ενέργειας (\en{ENEL}) ήταν η προσπάθεια ελαχιστοποίησης των μη τεχνικών απωλειών στο δίκτυα διανομής τους. Η μείωση των ρευματοκλοπών βοήθησε στην αιτιολόγηση μεγάλων επενδύσεων σε \en{AMR} και επί του παρόντος η Ιταλία πρωταγωνιστεί στην διείσδυση \en{AMR} \cite{regulation}.\\ 
Οι εταιρίες παραγωγής, μεταφοράς και διανομής αναλαμβάνουν την ευθύνη της κάλυψη των ενεργειακών αναγκών των πελατών. Μερικοί μπορεί να υποστηρίζουν ότι οι αυτές οι εταιρίες παρέχουν κακή εξυπηρέτηση, υπερχρεώνουν, κερδίζουν ανεξαρτήτως αρκετά χρήματα και ως εκ τούτου, ένα ποσοστό κλοπής δεν θα καταστρέψει την εταιρία ή θα επηρεάσει δραστικά τις λειτουργίες και την κερδοφορία της. Άλλοι παρατηρώντας την ίδια κατάσταση θα υποστήριζαν ότι η κλοπή είναι έγκλημα και δεν θα έπρεπε να επιτρέπεται. Η Διεθνής Εταιρία Προστασίας Εσόδων των Πάροχων (\en{ International Utilities Revenue Protection Association}) έχει καθιερωθεί για να προάγει τον εντοπισμό και την πρόληψη της κλοπής ρεύματος κυρίως για την οικονομική ασφάλεια των εταιριών παροχής ενέργειας.\par
Οι συνέπειες της κλοπής είναι εξαιρετικά σημαντικές και μπορούν να επηρεάσουν άμεσα τη βιωσιμότητα των υπηρεσιών που παρέχονται. Οι συνδιασμένες απώλειες (συμπεριλαμβάνοντας και τους απλήρωτους λογαριασμούς) σε μερικά συστήματα έχουν σοβαρές επιπτώσεις που έχουν ως αποτέλεσμα οι εγκαταστάσεις να λειτουργούν σε καθεστώς μεγάλων απωλειών. Όταν οι εταιρίες παραγωγής, μεταφοράς και διανομής λειτουργούν σε καθεστός αναποτελεσματικότητας και διαφθοράς η παροχή αξιόπιστων υπηρεσιών επιτυγχάνεται με μεγάλη δυσκολία. Ακόμη και σε αποτελεσματικά συστήματα ισχύος, όπως η \en{Tenaga} της Μαλαισίας, η κλοπή ρεύματος ανέρχεται στα \$132 εκατομμύρια ετησίως \cite{malaysia}. Αντίστοιχα στην Ελλάδα η συνολική εγχεόμενη ενέργεια στο Δίκτυα Διανομής ανήλθε το 2016 σε 47.655.372 \en{MWh}, το σύνολο των ρευματοκλοπών εκτιμάται σε 1.525.292 \en{MWh}. Στην πραγματικότητα όμως το μέγεθος των ρευματοκλοπών είναι αρκετά μεγαλύτερο, επιβαρύνει δε κατά κύριο λόγο τη Δημόσια Επιχείρηση Ηλεκτρισμού (ΔΕΗ). Ωστόσο παίρνοντας ως δεδομένη την ποσότητα, που αναγνωρίζει η Ρυθμιστική Αρχή Ενέργειας (ΡΑΕ), τα έσοδα που διαφεύγουν κάθε χρόνο λόγω των ρευματοκλοπών με βάση τις μοναδιαίες τιμές του 2016 έχουν ως εξής \cite{dehfraud}:

\begin{table}[ht!]
\centering
\begin{tabular}{ |c||c|  }
 \hline
 Εταιρίες & εκατ. \euro\\
 \hline
 ΔΕΗ & 120-125\\
 Υπηρεσίες Κοινής Ωφέλειας (ΥΚΩ) & 21\\
 ΕΤΜΕΑΡ & 32\\
 ΑΔΜΗΕ 4 & 7,3\\
 ΔΕΔΔΗΕ 5 & 26,5\\
 \hline
 Σύνολο& 206,8 έως 211,8\\
 \hline
\end{tabular}
\caption{Διαφεύγοντα έσοδα Ελληνικών εταιριών λόγω ρευματοκλοπών}
\label{tab:lostearnings}
\end{table}
\subsection{Ορίζοντας τις ρευματοκλοπές}
Σύμφωνα με το εγχειρίδιο ρευματοκλοπών της ΡΑΕ ρευματοκλοπή ορίζεται εν γένει η αυθαίρετη και με δόλο επέμβαση σε εξoπλισμό ή εγκαταστάσεις του Δικτύου, με σκοπό την κατανάλωση ηλεκτρικής ενέργειας χωρίς αυτή να καταγράφεται, ή χωρίς να αντιστοιχίζεται με Εκπρόσωπο Φορτίου, και να μην τιμολογείται \cite{rae}. Υπάρχουν τέσσερις επικρατούντες μέθοδοι «κλοπής» σε όλα τα συστήματα ενέργειας. Η έκταση της κλοπής εξαρτάται από πλήθος παραγόντων από πολιτιστικούς μέχρι τον τρόπο που διαχειρίζεται η ενέργεια. 

\subsubsection{Επέμβαση στο μετρητή}
Επέμβαση στο μετρητή ορίζεται όταν ο καταναλωτής σκοπίμως επεμβαίνει στη μετρικτική διάταξη με σκοπό τη χαμηλότερη χρέωση. Μια συνήθης πρακτική είναι να παραβιάζει το μετρητή ώστε να καταγράφει χαμηλότερα ποσά ενέργειας από τα πραγματικά. Αυτό εν γένει είναι μια  επικίνδυνη διαδικασία για ένα ερασιτέχνη, και σε πολλές περιπτώσεις έχουν καταγραφεί ηλεκτροπληξίες. Στην Ελλάδα πρόκειται για τη συνηθέστερη περίπτωση ρευματοκλοπής \cite{rae}.
\subsubsection{Απευθείας Σύνδεση}
Η κλοπή ενέργειας επιτευχθεί τραβώντας μια γραμμή από το δίκτυο διανομής μέχρι το επιθυμητό σημείο παρακάμπτοντας το μετρητή. Ένας καθιερωμένος τρόπος κλοπής ενέργειας στην Ελλάδα είναι η απευθείας σύνδεση με αγκίστρωση στους αγωγούς του εναέριου δικτύου, απουσία μετρητικής διάταξης ή παροχής ή νομίμως υφιστάμενου κτίσματος \cite{rae}.
\subsubsection{Ακανόνιστες χρεώσεις}
Οι ακανόνιστες χρεώσεις μπορούν να συμβούν από πολλές πηγές. Κάποιοι οργανισμοί παροχής ενέργειας μπορεί να μην είναι αρκετά αποτελεσματικοί στη μέτρηση της ενέργειας που έχει καταναλωθεί και ακούσια μπορεί να δώσουν υψηλότερη ή χαμηλότερη μέτρηση από την πραγματική. Αυτές οι ακανόνιστες χρεώσεις μπορεί να ισοζυγιστούν με την πάροδο του χρόνου. Παρόλα αυτά, είναι πολύ εύκολο σε μερικά συστήματα να έρθει σε επαφή εργαζόμενος με καταναλωτή για να ορίσουν  πολύ χαμηλότερους λογαριασμούς από τους ρεαλιστικούς. Εργαζόμενοι μπορεί να δωροδοκηθούν για να καταγράψουν το μετρητή με μικρότερο νούμερο από αυτό που ενδεικνύεται. Ο καταναλωτής πληρώνει μικρότερο λογαριασμό και ο εργαζόμενος που καταγράφει τις μετρήσεις αποκτά ανεπίσημο μισθό.
\subsubsection{Απλήρωτοι λογαριασμοί}
Κάποια άτομα και κάποιοι οργανισμοί δεν πληρώνουν αυτά που οφείλουν για ηλεκτρική ενέργεια. Οικιακοί ή επιχειρηματικοί καταναλωτές  μπορεί να έχουν φύγει από την πόλη ή την εγκατάσταση λόγω χρεωκοπίας. Στη Νότιο Αμερική, υπάρχει«καθεστώς μη πληρωμής» \cite{mkhwanazi}. Στην Αρμενία, τα επίπεδα μη πληρωμής είναι της τάξης του 80-90\% για τον οικιακό τομέα. Οι απώλειες των μετασχηματιστών και της διανομής είναι άνω του 40\% \cite{tacis}.\par
Σε όλες τις χώρες, καθώς η τιμή της ηλεκτρικής ενέργειας αυξάνεται, κάποιοι άνθρωποι αδυνατούν να πληρώσουν τους λογαριασμούς τους με συνέπεια. Αυτό τους ενθαρρύνει να βρουν τρόπους να μειώσουν τους λογαριασμούς, όπως να επεμβαίνουν τους μετρητές.
\section{Δομή Διπλωματικής}
Στην παρούσα διπλωματική γίνεται μια διεξοδική αναζήτηση μεθόδων ανίχνευσης απάτης με μια πληθώρα διαφορετικών αλγορίθμων από την σκοπιά της μηχανικής μάθησης. Δεδομένου του εύρους των δυνατοτήτων της μηχανικής μάθησης γίνεται προσπάθεια για αντιμετώπιση του προβλήματος από διαφορετικές οπτικές γωνίες, προσπαθώντας να επιτευχθεί η βέλτιστη αντιστάθμιση μεταξύ απόδοσης και πρακτικότητας. Η εξισορρόπηση αυτών των παραγόντων είναι κύριο μέλημα κάθε μηχανικού. Ειδικότερα, συνοψίζοντας κάθε κεφάλαιο εξάγεται η παρακάτω δομή:%\cite{artoftradeoff}
\subsubsection{Κεφάλαιο 1}
Γνωστοποιείται η κινητήριος δύναμη αυτής της διπλωματικής, κάνοντας ένα σαφή ορισμό του προβλήματος προς αντιμετώπιση.
\subsubsection{Κεφάλαιο 2}
Γίνεται μια εισαγωγή στα εργαλεία που χρησιμοποιούνται για την λήψη των αρχικών χρονοσειρών, την επεξεργασία τους και ταξινόμηση των καταναλωτών, αλλά και για τις συνιστώσες που λαμβάνονται υπόψιν για τα τελικά αποτελέσματα.
\subsubsection{Κεφάλαιο 3}
Αναπτύσσεται η μορφή και φύση των δεδομένων, αλλά και η μεθοδολογία προεπεξεργασίας τους. Παράλληλα, διευκρινίζεται ο τρόπος προσομοίωσης και μοντελοποίησης της ρευματοκλοπής.
\subsubsection{Κεφάλαιο 4}
Δημιουργείται ένας άξονας αναφοράς για τα αποτελέσματα με τη χρήση αλγορίθμων επιβλεπόμενης μάθησης που φημίζονται για την μεγάλη ευστοχία τους, αλλά και την δυσκολία εφαρμογής τους σε πραγματικά προβλήματα.
\subsubsection{Κεφάλαιο 5}
Εξετάζονται λεπτομερώς τα συστατικά των αλγορίθμων μη-επιβλεπόμενης μάθησης, ενώ παράλληλα διεξάγεται δοκιμές για την εξερεύνηση των διαφορετικών μεθόδων επίλυσης του θέματος.
\subsubsection{Κεφάλαιο 6}
Επεξηγούνται οι δυσκολίες που αντιμετωπίστηκαν από το διαφορετικά πρίσματα. Αναλυτικότερα γνωστοποιούνται τα τεχνικά εμπόδια που αντιμετωπίστηκαν, αλλά και τα εμπόδια που θα αντιμετωπίσου οι καταναλωτές, προσπαθώντας να οριστεί ένα μονοπάτι αποφυγής τους και αρμονικής συνύπαρξης των δύο πλευρών.
\subsubsection{Κεφάλαιο 7}
Γίνεται σφαιρική εποπτεία των αποτελεσμάτων με γνώμονες τη φύση κάθε αλγορίθμου και την ευστοχία στην ταξινόμηση των καταναλωτών.