                %%
\documentclass[11pt,a4paper,english,greek,twoside]{ceid-thesis}

\usepackage{graphicx}
\usepackage{epstopdf}
\usepackage{indentfirst}
\usepackage{verbatim}
\usepackage{amsmath}
\usepackage{amsthm}
\usepackage{amssymb}
\usepackage{latexsym}
\bibliographystyle{static/hellas}
\usepackage{hyphenat}
\usepackage{makeidx}
\addto\captionsgreek{%
  \renewcommand{\indexname}{Ευρετήριο όρων}%
}
\makeindex

% 1.5 spacing
\renewcommand{\baselinestretch}{1.2}

% latin text (and greek text)
\newcommand{\tl}[1]{\textlatin{#1}}
\newcommand{\tg}[1]{\textgreek{#1}}

% typeset short english phrases
\newcommand{\en}[1]{\foreignlanguage{english}{#1}}

% typeset source code
\newcommand{\src}[1]{{\tt\en{#1}}}

% typeset a backslash
\newcommand{\bkslash}{\en{\symbol{92}}}

\newtheorem{definition}{Ορισμός}
\newtheorem{proposition}{Πρόταση}
\newtheorem{theorem}{Θεώρημα}
\newtheorem{corollary}{Συμπέρασμα}
\newtheorem{lemma}{Λήμμα}
\newtheorem{example}{Παράδειγμα}
\newtheorem{remark}{Σημείωση}
\newtheorem{notation}{Συμβολισμός}
\newtheorem{law}{Νόμος}
\renewcommand{\thedefinition}{\arabic{chapter}.\arabic{definition}}
\renewcommand{\theproposition}{\arabic{chapter}.\arabic{proposition}}
\renewcommand{\thetheorem}{\arabic{chapter}.\arabic{theorem}}
\renewcommand{\thecorollary}{\arabic{chapter}.\arabic{corollary}}
\renewcommand{\thelemma}{\arabic{chapter}.\arabic{lemma}}
\renewcommand{\theexample}{\arabic{chapter}.\arabic{example}}
\newcommand{\set}[1]{\left\{#1\right\}}
\newcommand{\To}{\Longrightarrow}
\newcommand{\xml}{\en{XML}}

\selectlanguage{greek}

\hyphenation{ο-ποί-α}

%%%%%%%%%%%%%%%%%%%%%%%%%%%%%%%%%%%%%%%%%%%%%%%%%%%%%
%% THESIS INFO 
%%
%
% Τίτλος Πτυχιακής Εργασίας
    \title{Χρονοεξαρτώμενες Εναλλακτικές Διαδρομές σε Οδικά Δίκτυα}
% "του" ή "της", ανάλογα με το φύλο του σπουδαστή
    \edef\toutis{του}
% Ονοματεπώνυμο σπουδαστή (ΚΕΦΑΛΑΙΑ, γενική πτώση)
    \edef\authorNameCapital{ΠΑΣΧΑΛΙΔΗ ΧΡΗΣΤΟΥ}
% Ονοματεπώνυμο σπουδαστή (πεζά, ονομαστική πτώση)
    \author{Χρήστος Πασχαλίδης}
% Ονοματεπώνυμο Επιβλέποντα Καθηγητή
    \supervisor{Χρήστος Ζαρολιάγκης}
    \edef\supervisorTitle{**** Τιτλος}
% Ονοματεπώνυμο Επιβλέποντα Καθηγητή
    \supervisorSecond{Σπύρος Κοντογιάννης}
    \edef\supervisorSecondTitle{**** Τιτλος}
% "Επιβλέπων" ή "Επιβλέπουσα", ανάλογα με το φύλο του Επιβλέποντα Καθηγητή
    \edef\supervisorMaleFemale{Επιβλέποντες}
% Τόπος, μήνας και έτος
    \edef\thesisPlaceDate{Πατρα, Δεκέμβριος 2016}
% Ημερομηνία Εξέτασης
    \edef\examinationDate{19η Δεκεμβρίου 2016}
% Έτος Copyright
    \edef\copyrightYear{2016}
% Ονοματεπώνυμο 1ου εξεταστή
    \epitropiF{****}
% Τίτλος 1ου εξεταστή
    \edef\epitropiFTitle{Επίκουρος Καθηγητής}
% Ονοματεπώνυμο 2ου εξεταστή
    \epitropiS{****}
% τίτλος 2ου εξεταστή
    \edef\epitropiSTitle{Επιστ. Συνεργάτης}
%%%%%%%%%%%%%%%%%%%%%%%%%%%%%%%%%%%%%%%%%%%%%%%%%%%%%


\begin{document}
\selectlanguage{greek}
\maketitle

\frontmatter
% Περίληψη
    \begin{acknowledgements}

Θα ήθελα να ευχαριστήσω τον επιβλέποντα καθηγητή κ. Νικόλαο Χατζηαργυρίου για την ευκαιρία που μου έδωσε να εκπονήσω τη παρούσα διπλωματική και την υποστήριξή του σε όλη την πορεία της.

Επίσης, θα ήθελα να ευχαριστήσω  τους καθηγητές κ. Σταύρο Παπαθανασίου και κ. Παύλο Γεωργιλάκη για την τιμή που μου έκαναν να συμμετάσχουν στην επιτροπή εξέτασης της διπλωματικής.

Eυχαριστώ ιδιαίτερα τον υποψήφιο διδάκτορα Γιώργη Μεσσήνη για την καθοδήγηση, στήριξη και καθοριστική βοήθεια που μου παρείχε.

Τέλος, θα ήθελα να ευχαριστήσω την οικογένειά μου και τους φίλους μου που παρέχουν πάντοτε ένα χέρι βοήθειας σε ό,τι χρειαστώ.

\end{acknowledgements}


\begin{abstract}
Οι εταιρίες παροχής ηλεκτρισμού αντιμετωπίζουν το ολοένα και αυξανόμενο πρόβλημα της διείσδυσης μη τεχνικών απωλειών στις καταναλώσεις των πελατών τους. Το γεγονός αυτό πλήττει σημαντικά τις εταιρίες, μειώνοντας το εισόδημά τους και θέτει σε κίνδυνο τους ανειδίκευτους καταναλωτές που επεμβαίνουν στις υποδομές του παρόχου. Η προσέγγιση αυτού του προβλήματος έγινε με προσομοίωση ρευματοκλοπών σε ετήσιες χρονοσειρές καταναλωτών και δοκιμάστηκαν πληθώρα αλγορίθμων επιβλεπόμενης, μη επιβλεπόμενης και ημι-επιβλεπόμενης μηχανικής μάθησης για την ανίχνευση των καταναλωτών με διείσδυση μη τεχνικών απωλειών. Τα αποτελέσματα αναδεικνύουν τις δυνατότητες των συστημάτων μη επιβλεπόμενης και ημι-επιβλεπόμενης μάθησης σε σχέση με τη δεδομένη επιτυχία των αλγορίθμων επιβλεπόμενης μάθησης. Τα συστήματα που δημιουργήθηκαν έχουν ικανοποιητική απόδοση που δεν αποκλίνει σημαντικά από τους αλγορίθμους αναφοράς της επιβλεπόμενης μάθησης. Καθίσταται λοιπόν σαφές πως η ανίχνευση μη τεχνικών απωλειών είναι εφικτή με συστήματα μηχανικής μάθησης.

\begin{keywords}
  Μη τεχνικές απώλειες, Ρευματοκλοπές, Χρονοσειρές, Μηχανική μάθηση, Επιβλεπόμενοι αλγόριθμοι, Μη επιβλεπόμενοι αλγόριθμοι, Ημι-επιβλεπόμενοι αλγόριθμοι.
 
\end{keywords}

\end{abstract}



\begin{abstracteng}
\tl{Power companies face the problem of increasing intrusion of non-technical losses on consumptions of their clients. That fact hurts significantly power companies by reducing their economical growth and sets on danger unskilled consumers who intervene with the power infastracture. This problem was approached by simulating frauds on yearly timeseries and by testing  many different algorithms of supervised, unsupervised and semi-supervised  machine learning in order to detect consumers with non-technical loss intrusion. The results show the potencial of the unsupervised and semi-supervised learning in relation with the given success of supervised algorithms. The created systems have satisfactory performance which does not diverge significantly from the reference algorithms of supervised learning. Concluding the detection of non-technical losses is achievable with machine learning systems.}
\begin{keywordseng}
  \tl{Non-technical losses, power fraud, Timeseries, Machine learning, Supervised algorithms, Unsupervised algorithms, Semi-supervised algorithms.}
\end{keywordseng}

\end{abstracteng}

% Αφιέρωση
    \thesisDedication{στους γονείς μου}
% Ευχαριστίες
    \include{front_matter/acknowledgements}
% Πίνακας Περιεχομένων
    \tableofcontents
% Κατάλογος Σχημάτων
    \listoffigures
% Κατάλογος Πινάκων
    \listoftables

%%%%%%%%%%%%%%%%%%%%%%%%%%%%%%%%%%%%%%%%%%%%%%%%%%%%%
%% INCLUDE YOUR CHAPTERS/SECTIONS HERE
%%
\mainmatter
% Εισαγωγή
    \include{body_matter/chap1}
% Κεφάλαια
    \chapter{\selectlanguage{greek}��������� ��������}
��� �������� ���� ��������������  ��������� �� �����
������� ����������� ��� ����� ����� �� ��� ������� ����, ������ ��
��������� �������� ������, �� ������� \en{RDF} ��� �� �������
��������� ��� \en{RDF}.

\section{��������� �������� ������}
\subsection{�� ����� �� ��������� �������� ������}
��� ������ ������������ ��������� ���� ����� � ���������� �����,
�������� ������ �� ���������� ��� ������������ ��������
������/�����������: �� ����� ����������� ���������� ��������� ��
������ ������� (������������) ����� ������� ���������� ���� ������
������� \cite{elli05}.

�� ����� ��� ������� �� ��������� �������� ������ ����� �� ����:
\begin{itemize}
\item � ���� ��� ����������� ��� �����.
\item � ���� ��� �������������.
\end{itemize}

������� �� �� ���������� ����, �� ���������� ��� ������������� ���
�����~\ref{figure2.2} �������� �� ����: \src{
\begin{tabbing}
1.<?x\=ml\= v\=ersion="1.0"?> \\
2.<rdf:RDFxmlns:rdf="http://www.w3.org/1999/02/22-rdf-syntax-ns\#" \\
3.\>\>\>xmlns:dc="http://purl.org/dc/elements/1.1/" \\
4.\>\>\>xmlns:exterms="http://www.example.org/terms/"> \\
5.\><rdf:Description
rdf:about="http://www.example.org/index.html"> \\
6.\>\><exterms:creation-date>August 16, 1999</exterms:creation-date> \\
7.\>\><dc:language>en</dc:language> \\
8.\>\><dc:creator rdf:resource="http://www.example.org/staffid/85740"/> \\
9.\></rdf:Description> \\
10.</rdf:RDF> \\
\end{tabbing}
}

    \chapter{��������� �������}
��� �������� ���� ������ ������� ��� ��������� ��� ����������
�������� ������ ��� ����� ��������� �� ������� \en{(schema-based
peer-to-peer systems)}. ��� �������� ������������� ���� ������
��������� ��� ������� �� ���� ��� ���������, ����� ��� ��� �������
��� �� ���������� \en{RDF} ��������, ��� ����� ��������� � ������
��� �������� ��������.

\section{�������� ��������}
�� ������ ��������� ��������� ��� ����� ����������� ��� ���������
��� ��������� ��� ��������� ��� ����� \cite{neidl03issues}. ��
����� ��������� �������� ������ ��� ����������� ���� ����������
2.1.2 ������ ������ ������� ���� ������������� ��� ���������� ���
��� ����������� ��� ��������� ��� �������� ���� �����
������������� ��� ��� ����������� ����������. � ��������� �� ����
�� ��������� �������� ������ ������� �� ���� ��������������
�������������� - �������, � �� ���������� ������������� ���� �����
������.

� ������ ������ ��� ��� ����������� ����������� ������� ���
��������� �������� ������ �� ����� ����� ��������� �� �������
(\en{schema based peer-to-peer systems}). ��������� ��� ��������
�������� ����������� ��������� ��� ���� ��������� ��� ��
�������������� ��� ���������� �������� ������.
............................
    \chapter{������� ��� ��������}
��� �������� ���� ������������� � ������ ��� ����� ��� ���
��������� ��� ����������. ������ ������������ � ������������� ���
���������� ��� ������� � ����������� ��� ��� ���������
������������, ��� ��� �������� ������������� �� ��������� ���
����������.

\section{������� - ��������� ��������������}
���� ������� ���� ������������� � ������� ��� ���������� ��� �
�������� ��� �� ������������ ���� ����� ��� �������������.

\subsection{����������� �������������}
�� ������� ����������� ��� ���� ������ ������� ��� ��� �����
�����������. ��� ������ ���� ��������� �� ������� ���� �����
������, �� ����� ����������� ��� �� ���� ������������:

\begin{itemize}
\item ���������� ����������� ��������.
\item ���������� ����������� ��������� ��� �����.
\item ���������� ������������ ������.
\end{itemize}


\begin{figure}[!ht] \center\leavevmode
\epsfxsize=12cm \epsfysize=14cm
\epsfbox{figures/peerArchitecture.eps} \caption{������������ �����
������}\label{figure4.1}
\end{figure}

�� �����~\ref{figure4.1} ����������� .................


\subsection{��������� �������������}
�������� ������� ���������� ��������� ��� ������ ��� �� ���������
��� ���������. � ��������� ���� ������� �� ���� �� �����������
���� ���������.

\subsubsection{���������� ����������� ��������}
�� ���������� ���� ...............

    \chapter{\selectlanguage{greek}���������}
��� �������� ���� ������������ � ��������� ��� ����������, �� ����
�� ������ ��� ������������� ��� ����������� ��������. ������
������������� � ��������� ��� �� ���������������� �������� ���
����������������. ��� �������� �������� �� ������������ ����������
��� ���� �������� ����������� ��� ���������� ����� ��� � ���� ���
������.

\section{\selectlanguage{greek}������������ ����������}
���� ������� ���� �������������� �� ������� ���������� ���
������������ ����� ��� ������������ ������� �� ��� ��������� ���
������������ ��� ������.

\subsection{����������}

\subsubsection{���������� ��������� ���������}
���� ���� ������ ���������� ��� ����� ���� ��� �������, ������
���������� �� ����� ��� ����� ��������������� �� \en{RDFSculpt}.
��� ��������................

\noindent\texttt{��������� ��� ����������� \en{groupedMapping}. \\
�������� ������������� ��} \texttt{��������} \texttt{��� \en{mapping} ��� �������
���� ���� �����. \\ �� �������� \en{groupedMapping} ���� �� �����: \\
$[[[[$�����1,������������1$]$,��������������1$]$,$[[$�����1,������������2$]$
\\,��������������2$]$,...$]$,$[[[$�����2,������������3$]$,��������������3$]$,
$[[$�����2,\\������������4$]$,��������������4$]$,...$]]$ }
\texttt{
\begin{tabbing}
��\=� �\=��� \=��\=��\=��\=� \\
����������� ��������� ��� \en{groupedMapping}, ��� ���������� \en{imapping} \\
��� �� \en{imapping} ���� �������� \\
\>���� �� ����� �������� ��� ����������� ���� \en{classMapping} \\
\>\>���� ��� ����� ��� ������ ��� ����� ��������, ��� �������� \\
\>\>\en{classesToWrite} \\
\>\>��� �� �������� \en{classesToWrite} ���� �������� \\
\>\>\>���� �� ��������-����� ��� ��������� ���� ���� ��� ����������� \\
\end{tabbing}
}

\noindent\textbf{����������} \\

���� ��� � ������ ���� �������� �� ���������� ��� ������� �� �� \en{RDF} ����� ��� ��������
��� �����. ���� ������ ��� ��� �� \en{SQL} ������� ��� ���� ����� ��� ��������
����, ���� �������� � ��� ��� �������� ���� ������ \ref{data}. ��� ��� ������� ��� ������������� ��������
��� � ��� ���� �������� ���� ��� �������.

...........................

\section{\selectlanguage{greek}��������� �������}
���� ������� ���� ������� ��� ������� ��������� ��� �������,
��� ������ ��� ��� ������� ��� ��� ����������.

\subsection{\selectlanguage{greek}\en{public class FirstUi}}
\noindent � ����� ���� ������������ ��� ����� ��������� ��� ������ ��� �������.\\

\noindent\textbf{�����}

\begin{itemize}
\item\src{private GridBagLayout blayout} \\
�� \en{layout} ��� ��� �� \en{Panel}.
\item\src{private GridBagConstraints con} \\
�� \en{constraints} ��� �� \en{layout}.
\item\src{private Icon arrowR} \\
��������� ��� �� ������ \en{Next}.
\end{itemize}

\noindent\textbf{�������}

\begin{itemize}
\item\src{public FirstUi()}\\
� ������������� ��� ������ � ������ ����� ��� \en{createEntryFrame()}.
\item\src{private void createEntryFrame()}\\
������� ��� ������������ �� en{frame}.
\end{itemize}



    \chapter{�������}
��� �������� ���� ������� � ������� ����� ����������� ���
����������.

\section{����������� �������}
� ������� ��� ���������� ����� ���������������� �� �� ����� ����
�������� �����������. ������� �� �� ������� ���� �������� ��� ���
������� �������� ����� ������ (\en{peer1,peer2,peer3}). ��������
������ ��� �� ������ \en{peer2} ��� \en{peer3} ����� ��� ����� ���
��������. �� ����� ��� \en{peer2} �������� ���
�����~\ref{figure6.1}.



������ � ��������� ��� ���������� ���� �� ����: � \en{peer2} �����
�������� ��� \en{peer1} ��� � \en{peer3} �������� ��� \en{peer2}.

������ ������ �� �������������� ����� ��� ��� ����� \en{peer1} ���
��� �������� �� ��������� �� ���� �������� ����������� ���� ���
���� ���������� ��� ������������� ����������� �������� ��� ���
������������� ��������� ���������. ��� �������� ��� ��� ����� ����
��������� ��������� ����� ���������� ��� ��� ������ ���
������������� ��������� ��������� ��� ������������ ������.

\section{��������� ���������� �������}
���� ������� ���� ������������� ��������� ��� ������ ���
���������� ������� �� �� ������� ��� ����������� ���� �����������
�������.

���� �������� �� ���������� � ������ \en{peer1} ������ �����������
� ����� ��� �������� ��� �����~\ref{figure6.3}. � ����� ���� �����
� ����� ��������� ��� ������� ���� ����� � ������� ������� ��
����� ��� ����� ���� ����� ����� ������������ �� �������� ���
����� ��� �� ����� ��� ����� ���� ����� �� ���������� �� \en{RDF}
��������.

����� � ������� ������� �� ������ \en{Next} ���� ��������� ���
������� ........
    \chapter{��������}

\section{������������}
�� ��������� �������� ������, ����������� �� ������������ ���
����������� ����������� ������������� ��� ���������� ���������,
����������� ��� ��������� �������� ������ �� ����� ���������� ����
����������� ��� �������������� ����� ��� ��� ������������ ���
��������� ���� �������� ��� �� ����������� (\en{Schema-based
peer-to-peer systems}).

��� ��������� ���� ���� ������ ������������ ��� ����� ��� ���
������������ ��� ��������� ���. ���� �� ��� ������� ��������
������, ���� ������ ���� ������������ ���������� �������������
���������. �������� ������ �� ������� �������� ���� �������
��������. �� ��������� ��� ����� �������� ����� ���� ��� ��������
���� ��� ��������, ��� �� ����� ������ � ��������� �����������,
�������� ��� ������ ������� ������������� ������ ��� �������� ��
���� ���� ������� �� ����������������� �� ���������. ���� ���
������������� ����� �������� ���������� ��� �������� �������� ���
�������, ��� ����� ���������� ��� �� ��������� �������� ������.

� ���������� ��� �������� ������������ �������� ���� ��� �����. ��
����� ����� �� ���������� ���� ������� ���������� �������� ������
��������� �� ������� \en{RDF} �� ����� �������: (�) ��� �������
��� ��� ����������� ��� ������,(�) ��������� ����������� ��������,
(�) ��������� ����������� ��������� ��������� ��� ����� �� ��
����� �������������� ��� ���������� � ������� �� �� �������
������� �������������, (�) �������� ������������ ������ ��� ��
��������� ���������� ��� (�) ��������� ��������� ��� ������������
���������.

�� ������� ������ ����� �� ������� ��� �� ������������ �������
��������� ��� ������� �������� �� ���� ��� ������� ��� ��������
��� ��� ���� �����, ��� ���������� ����� �� ����������
��������������� ��������� ����� �� ����� ������� �������������.
������������, �� ������� ��� ������ ���������
���������$-$�����$($\en{views}) ���� ������� �������� ���
���������� �������� �����. ���������������� ������ �� ������� ���
�� ������� ���� ����� ������� ������ ����, ������ �� ����������
������� ��� ������������������ ���� �������� ��� ���������� ���
���� ����������, ��������������� ���� �� ����� ��� ������ ��� ���
�� �������� �����.

�������������� �� ������� ��� ����������� ��� ������� ����� ���
������������ ����� ��� ������ ������� �������� ������ ��������� ��
�������, �� ����� ������� ������ ��� ��������� ��� ����������� ��
��� ����������� ����� ��' ��� �� ������������  ���������.

\section{����������� ����������}
�� ������� ��� ����������� ��� ������� ����� ��� ������������
�������� �� �������� �� ��������� ��� �� ��������� ���������,
����������� �� ���� ����� ������������. ������������, �����������
�� ��������:

\begin{itemize}
\item ���������� ����������� �������� �������� �� ���� �� ����� �
������ �� ���������� ��� �������. ���� ���� ���� ���������� ��
�������, ���� ������ ���� �� ���������� �� ������������ �����
������� ��� �� ����������� �������� �� ����������� ��� ���. ��
����� ��� ������ (�� ���� �� ����� �������� ��� ���������),
��������� �� ��������� ��� ����� ���������� ��������. � ����������
�������� �� ��� ������� ����������� ��������.
\item ���������� ������������� ��������� �� ����� �� ��� �����
������������ �� ���� ��������� ���� �� ������. � ����������� ���
�� ���� ��������� �� ����� �� ������� ��� ������ ���� �����������
��� �� �����.
\item ���������� ��� ���������� �� ���� �� ����������� ��� ��
���������� �� ���� ������� ������� ������ \en{(scalability
testing)} ��� �� �������������� ��� ���� ������ �������� �����. H
���������� ���� ����� ��� �������� �� ��� ����� ���� ������
������� ���������� �� ��� ������� ����� ��� ��� �������� ���
����������.
\end{itemize}

    \include{body_matter/chap8}
    \include{body_matter/chap9}
% Παραρτήματα
    \appendix
    \include{back_matter/appA}
    \include{back_matter/appB}  
    \cleardoublepage
% Βιβλιογραφία - Αναφορές
    \bibliography{back_matter/references}
% Συντομογραφίες - Αρκτικόλεξα - Ακρωνύμια
    \include{back_matter/abbreviations}
% Γλωσσάριο
    \newcommand{\gloss}[2]{#1 \> \en{#2}\\ }

\chapter{����������� ����� ����}

\begin{tabbing}
%ta 'a' rythmizoun to platos ton dyo stilon
  aaaaaaaaaaaaaaaaaaaaaaaaaaaaaaaaaaa \= aaaa\kill
  \Large\textbf{���������} \> \Large\textbf{�������� ����} \\
  \gloss{�������}{sibling}
  \gloss{��������������}{idempotency}
  %\gloss{������������}{identifier}
  \gloss{�������� �����������}{information retrieval}
  \gloss{������������������}{commutativity}
  \gloss{��������}{descedant}
  \gloss{����������}{absorption}
  \gloss{���� ���������}{database}
  \gloss{��������}{attribute}
  \gloss{������������}{interface}
  \gloss{�������}{difference}
  \gloss{��������� ���������}{portal catalog}
  \gloss{�������� ����}{lattice}
  \gloss{������� �����������}{structural queries}
  \gloss{������� �������}{structural relationships}
  \gloss{������ �����}{schema}
  \gloss{����������}{validity}
  \gloss{�����}{union}
\end{tabbing}

%%%%%%%%%%%%%%%%%%%%%%%%%%%%%%%%%%%%%%%%%%%%%%%%%%%%
\backmatter
% Ευρετήριο Όρων
    \printindex
    \cleardoublepage

%%%%%%%%%%%%%%%%%%
%%%%%%%%%%%%%%%%%%

%% Δημιουργία ετικετών CD:

    \definecdlabeloffsets{0}{-0.65}{0}{0.55} % upper label x offset [cm] (default=0) /  upper label y offset [cm] (default=0) /  lower label x offset [cm] (default=0) /  lower  label y offset [cm] (default=0) -- For Q-Connect KF01579 labels use the following offset values: {0}{-0.65}{0}{0.55}

    \createcdlabel{Πρότυπο Σύστημα Ομότιμων \\ Κόμβων Βασισμένο σε Σχήματα \en{RDF}}{Κωνσταντίνος Δ. Δημητρίου}{ΟΚΤΩΒΡΙΟΣ}{2014}{8} % τίτλος πτυχιακής / όνομα συγγραφέα / μήνας / έτος / εύρος περιοχής τίτλου σε cm (προτεινόμενη τιμή: 8) 

%%
%% Δημιουργία εξωφύλλου θήκης CD:

    \createcdcover{Πρότυπο Σύστημα Ομότιμων \\ Κόμβων Βασισμένο σε Σχήματα \en{RDF}}{Κωνσταντίνος Δ. Δημητρίου}{ΟΚΤΩΒΡΙΟΣ}{2014}{10} % τίτλος πτυχιακής / όνομα συγγραφέα / μήνας / έτος / εύρος περιοχής τίτλου σε cm (προτεινόμενη τιμή: 10) 

%%
    \pagebreak
    \thispagestyle{empty}
\end{document}
              